% !TeX encoding = UTF-8
%% !TeX program = XeLaTeX

% LaTeX 命令和环境
%
% \command_name[optional_arguments...]{mandatory_arguments...}
%
% \begin{environment_name}[optional_arguments...]{manadatory_arguments...}
% \end{enviroment_name}

% 指定文档类
\documentclass{ctexbook}

% 导言区:调用宏包、全局设置
\usepackage{amsmath}   % https://ctan.org/pkg/amsmath   数学符号支持
\usepackage{amssymb}   % https://ctan.org/pkg/amsmath   数学符号支持
\usepackage{hyperref}  % https://ctan.org/pkg/hyperref  超链接引用支持
\usepackage{latexsym}  % https://ctan.org/pkg/latexsym  符号支持
\usepackage{syntonly}  % https://ctan.org/pkg/syntonly  快速语法检查支持
\usepackage{ulem}      % https://ctan.org/pkg/ulem      下划线支持
\usepackage{xcolor}    % https://ctan.org/pkg/color     彩色字符支持

\hypersetup{colorlinks=true, unicode=true}
%\syntaxonly                                 % 仅进行语法检查而不生成文档,加快编译速度

% 指示文档内容
\begin{document}

%===============================================================================
% 标题
\title{rrMathematics}
\author{zhengrr}
\date{\today}
\maketitle

%===============================================================================
% 目录
\tableofcontents

%===============================================================================
\chapter{\TeX / \LaTeX / \LaTeXe}

墨色命令不依赖额外宏包,\textcolor{brown}{棕色}命令依赖\texttt{latexsym}宏包,\textcolor{blue}{蓝色}命令依赖\texttt{amsmath}宏包,\textcolor{teal}{凫色}命令依赖\texttt{amssymb}宏包。

%-------------------------------------------------------------------------------
\section{转义字符}

\begin{table}[h]
\centering
\begin{tabular}{r c p{15ex} l p{5ex} r c p{15ex} l}
	\hline

	 35 & \#             & \verb|\#|             & 井号 &&
	 94 & \^{}           & \verb|\^{}|           & 脱字符 \\

	 36 & \$             & \verb|\$|             & 美元符 &&
	 95 & \_             & \verb|\_|             & 下划线 \\

	 37 & \%             & \verb|\%|             & 百分号 &&
	123 & \{             & \verb|\{|             & 左花括号 \\

	 38 & \&             & \verb|\&|             & 和号 &&
	125 & \}             & \verb|\}|             & 右花括号 \\

	 92 & \textbackslash & \verb|\textbackslash| & 反斜线 &&
	126 & \~{}           & \verb|\~{}|           & 波浪号 \\

	\hline
\end{tabular}
\end{table}

%-------------------------------------------------------------------------------
\section{标点符号}

\begin{table}[h]
\centering
\begin{tabular}{c p{15ex} l p{5ex} c p{15ex} l}
	\hline

	-          & \verb|-|          & 连字符 &&
	\S         & \verb|\S|         & 分节符 \\

	--         & \verb|--|         & 连接号 &&
	\copyright & \verb|\copyright| & 版权符 \\

	---        & \verb|---|        & 破折号 &&
	\dag       & \verb|\dag|       & 剑标 \\

	`'         & \verb|`'|         & 单引号 &&
	\ddag      & \verb|\ddag|      & 双剑标 \\

	``''       & \verb|``''|       & 双引号 &&
	\dots      & \verb|\dots|      & 省略号 \\

	\P         & \verb|\P|         & 段落符 &&
	\pounds    & \verb|\pounds|    & 英镑符 \\

	\hline
\end{tabular}
\end{table}

%-------------------------------------------------------------------------------
\newpage
\section{希腊字母}

\begin{table}[h]
\centering
\begin{tabular}{c p{15ex} p{5ex} c p{15ex} p{5ex} c p{15ex}}
	\hline

	$A$           & \verb|$A$|                                               &&
	$\mathrm{A}$  & \verb|$\mathrm{A}$|                                      &&
	$\alpha$      & \verb|$\alpha$|                                          \\

	$B$           & \verb|$B$|                                               &&
	$\mathrm{B}$  & \verb|$\mathrm{B}$|                                      &&
	$\beta$       & \verb|$\beta$|                                           \\

	$\varGamma$   & \texttt{\$\textcolor{blue}{\textbackslash varGamma}\$}   &&
	$\Gamma$      & \verb|$\Gamma$|                                          &&
	$\gamma$      & \verb|$\gamma$|                                          \\

	$\varDelta$   & \texttt{\$\textcolor{blue}{\textbackslash varDelta}\$}   &&
	$\Delta$      & \verb|$\Delta$|                                          &&
	$\delta$      & \verb|$\delta$|                                          \\

	$E$           & \verb|$E$|                                               &&
	$\mathrm{E}$  & \verb|$\mathrm{E}$|                                      &&
	$\epsilon$    & \verb|$\epsilon$|                                        \\

	$Z$           & \verb|$Z$|                                               &&
	$\mathrm{Z}$  & \verb|$\mathrm{Z}$|                                      &&
	$\zeta$       & \verb|$\zeta$|                                           \\

	$H$           & \verb|$H$|                                               &&
	$\mathrm{H}$  & \verb|$\mathrm{H}$|                                      &&
	$\eta$        & \verb|$\eta$|                                            \\

	$\varTheta$   & \texttt{\$\textcolor{blue}{\textbackslash varTheta}\$}   &&
	$\Theta$      & \verb|$\Theta$|                                          &&
	$\theta$      & \verb|$\theta$|                                          \\

	$I$           & \verb|$I$|                                               &&
	$\mathrm{I}$  & \verb|$\mathrm{I}$|                                      &&
	$\iota$       & \verb|$\iota$|                                           \\

	$K$           & \verb|$K$|                                               &&
	$\mathrm{K}$  & \verb|$\mathrm{K}$|                                      &&
	$\kappa$      & \verb|$\kappa$|                                          \\

	$\varLambda$  & \texttt{\$\textcolor{blue}{\textbackslash varLambda}\$}  &&
	$\Lambda$     & \verb|$\Lambda$|                                         &&
	$\lambda$     & \verb|$\lambda$|                                         \\

	$M$           & \verb|$M$|                                               &&
	$\mathrm{M}$  & \verb|$\mathrm{M}$|                                      &&
	$\mu$         & \verb|$\mu$|                                             \\

	$N$           & \verb|$N$|                                               &&
	$\mathrm{N}$  & \verb|$\mathrm{N}$|                                      &&
	$\nu$         & \verb|$\nu$|                                             \\

	$\varXi$      & \texttt{\$\textcolor{blue}{\textbackslash varXi}\$}      &&
	$\Xi$         & \verb|$\Xi$|                                             &&
	$\xi$         & \verb|$\xi$|                                             \\

	$O$           & \verb|$O$|                                               &&
	$\mathrm{O}$  & \verb|$\mathrm{O}$|                                      &&
	$\mathrm{o}$  & \verb|$\mathrm{o}$|                                      \\

	$\varPi$      & \texttt{\$\textcolor{blue}{\textbackslash varPi}\$}      &&
	$\Pi$         & \verb|$\Pi$|                                             &&
	$\pi$         & \verb|$\pi$|                                             \\

	$P$           & \verb|$P$|                                               &&
	$\mathrm{P}$  & \verb|$\mathrm{P}$|                                      &&
	$\rho$        & \verb|$\rho$|                                            \\

	$\varSigma$   & \texttt{\$\textcolor{blue}{\textbackslash varSigma}\$}   &&
	$\Sigma$      & \verb|$\Sigma$|                                          &&
	$\sigma$      & \verb|$\sigma$|                                          \\

	$T$           & \verb|$T$|                                               &&
	$\mathrm{T}$  & \verb|$\mathrm{T}$|                                      &&
	$\tau$        & \verb|$\tau$|                                            \\

	$\varUpsilon$ & \texttt{\$\textcolor{blue}{\textbackslash varUpsilon}\$} &&
	$\Upsilon$    & \verb|$\Upsilon$|                                        &&
	$\upsilon$    & \verb|$\upsilon$|                                        \\

	$\varPhi$     & \texttt{\$\textcolor{blue}{\textbackslash varPhi}\$}     &&
	$\Phi$        & \verb|$\Phi$|                                            &&
	$\phi$        & \verb|$\phi$|                                            \\

	$X$           & \verb|$X$|                                               &&
	$\mathrm{X}$  & \verb|$\mathrm{X}$|                                      &&
	$\chi$        & \verb|$\chi$|                                            \\

	$\varPsi$     & \texttt{\$\textcolor{blue}{\textbackslash varPsi}\$}     &&
	$\Psi$        & \verb|$\Psi$|                                            &&
	$\psi$        & \verb|$\psi$|                                            \\

	$\varOmega$   & \texttt{\$\textcolor{blue}{\textbackslash varOmega}\$}   &&
	$\Omega$      & \verb|$\Omega$|                                          &&
	$\omega$      & \verb|$\omega$|                                          \\

	\hline
\end{tabular}
\end{table}

%-------------------------------------------------------------------------------
\newpage
\section{文字强调}

\underline{underlined} % 下划线

\uline{underlined}     % 下划线

\emph{强调} 版本

%-------------------------------------------------------------------------------
\section{章节和目录}

\begin{table}[h]
\centering
\begin{tabular}{l l}
	\hline
	\texttt{\textbackslash chapter\{\textcolor{gray}{title}\}}       & 章 \\
	\texttt{\textbackslash section\{\textcolor{gray}{title}\}}       & 节 \\
	\texttt{\textbackslash subsection\{\textcolor{gray}{title}\}}    & 小节 \\
	\texttt{\textbackslash subsubsection\{\textcolor{gray}{title}\}} & 小小节 \\
	\texttt{\textbackslash paragraph\{\textcolor{gray}{title}\}}     & 段 \\
	\texttt{\textbackslash subparagraph\{\textcolor{gray}{title}\}}  & 小段 \\
	\verb|\tableofcontents|                                          & 目录 \\
	\hline
\end{tabular}
\end{table}

%===============================================================================
\chapter{集合、映射和二元运算}

%-------------------------------------------------------------------------------
\section{集合和集合的笛卡尔积}

一般的,\emph{集合}(\href{http://mathworld.wolfram.com/Set.html}{Set},简称\emph{集})是指具有某种特定性质的事物的总体,组成这个集合的事物称为该集合的\emph{元素}(\href{http://mathworld.wolfram.com/Element.html}{Element},简称\emph{元}).

%通常使用大写拉丁字母 A,B,C,\cdots{} 表示集合,用小写拉丁字母 a,b,c,\cdots{} 表示集合的元素


% 附录
\appendix

%===============================================================================
\chapter{参考书目}

\begin{enumerate}
	\item 全国信息与文献标准化技术委员会.信息与文献 参考文献著录规则:GB/T 7714—2015 [S/OL].北京:中国标准出版社,2015.\href{http://www.scal.edu.cn/dxtsgxb/201906120155}{超链接}.
	\item 同济大学数学系.高等数学[M].第六版.北京:高等教育出版社,2007.
\end{enumerate}

\end{document}
