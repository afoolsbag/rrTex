% !TeX encoding = UTF-8
% XeLaTeX

% LaTeX 命令和环境
%
% \command_name[optional_arguments...]{mandatory_arguments...}
%
% \begin{environment_name}[optional_arguments...]{manadatory_arguments...}
% \end{enviroment_name}

% 指定文档类
\documentclass{ctexbook}

% 导言区:调用宏包、全局设置
\usepackage{amsmath}   % https://ctan.org/pkg/amsmath   数学符号支持
\usepackage{amssymb}   % https://ctan.org/pkg/amsmath   数学符号支持
\usepackage{hyperref}  % https://ctan.org/pkg/hyperref  超链接引用支持
\usepackage{syntonly}  % https://ctan.org/pkg/syntonly  快速语法检查支持
\usepackage{ulem}      % https://ctan.org/pkg/ulem      下划线支持
\usepackage{xcolor}    % https://ctan.org/pkg/color     彩色字体支持

\hypersetup{colorlinks=true, unicode=true}
%\syntaxonly                                 % 仅进行语法检查而不生成文档,加快编译速度

% 指示文档内容
\begin{document}

%===============================================================================
% 标题
\title{rrMathematics}
\author{zhengrr}
\date{\today}
\maketitle

%===============================================================================
% 目录
\tableofcontents

%===============================================================================
\chapter{\TeX / \LaTeX / \LaTeXe}

%-------------------------------------------------------------------------------
\section{转义字符}

\begin{table}[h]
\caption{转义字符}
\begin{center}
\begin{tabular}{r c p{15ex} p{5ex} r c p{15ex}}
	\hline
	 35 & \#             & \texttt{\textbackslash \#}            &&  94 & \^{}           & \texttt{\textbackslash \^{}\{\}}      \\
	 36 & \$             & \texttt{\textbackslash \$}            &&  95 & \_             & \texttt{\textbackslash \_}            \\
	 37 & \%             & \texttt{\textbackslash \%}            && 123 & \{             & \texttt{\textbackslash \{}            \\
	 38 & \&             & \texttt{\textbackslash \&}            && 125 & \}             & \texttt{\textbackslash \}}            \\
	 92 & \textbackslash & \texttt{\textbackslash textbackslash} && 126 & \~{}           & \texttt{\textbackslash \~{}\{\}}      \\
	\hline
\end{tabular}
\end{center}
\end{table}

%-------------------------------------------------------------------------------
\section{标点符号}

\begin{table}[h]
\caption{标点符号}
\begin{center}
\begin{tabular}{c l l}
	\hline
	`'    & \texttt{`'}                  & 单引号 \\
	``''  & \texttt{``''}                & 双引号 \\
	-     & \texttt{-}                   & 连字符 \\
	--    & \texttt{--}                  & 连接号 \\
	---   & \texttt{---}                 & 破折号 \\
	\dots & \texttt{\textbackslash dots} & 省略号 \\
	\hline
\end{tabular}
\end{center}
\end{table}

%-------------------------------------------------------------------------------
\section{数学模式下的希腊字母}

\begin{table}[h]
\caption{数学模式下的希腊字母}
\begin{center}
\begin{tabular}{c p{15ex} p{5ex} c p{15ex} p{5ex} c p{15ex}}
	\hline
	$A$           & \texttt{A}                          && $\mathrm{A}$  & \texttt{\textbackslash mathrm\{A\}} && $\alpha$      & \texttt{\textbackslash alpha}       \\
	$B$           & \texttt{B}                          && $\mathrm{B}$  & \texttt{\textbackslash mathrm\{B\}} && $\beta$       & \texttt{\textbackslash beta}        \\
	$\varGamma$   & \texttt{\textbackslash varGamma}    && $\Gamma$      & \texttt{\textbackslash Gamma}       && $\gamma$      & \texttt{\textbackslash gamma}       \\
	$\varDelta$   & \texttt{\textbackslash varDelta}    && $\Delta$      & \texttt{\textbackslash Delta}       && $\delta$      & \texttt{\textbackslash delta}       \\
	$E$           & \texttt{E}                          && $\mathrm{E}$  & \texttt{\textbackslash mathrm\{E\}} && $\epsilon$    & \texttt{\textbackslash epsilon}     \\
	$Z$           & \texttt{Z}                          && $\mathrm{Z}$  & \texttt{\textbackslash mathrm\{Z\}} && $\zeta$       & \texttt{\textbackslash zeta}        \\
	$H$           & \texttt{H}                          && $\mathrm{H}$  & \texttt{\textbackslash mathrm\{H\}} && $\eta$        & \texttt{\textbackslash eta}         \\
	$\varTheta$   & \texttt{\textbackslash varTheta}    && $\Theta$      & \texttt{\textbackslash Theta}       && $\theta$      & \texttt{\textbackslash theta}       \\
	$I$           & \texttt{I}                          && $\mathrm{I}$  & \texttt{\textbackslash mathrm\{I\}} && $\iota$       & \texttt{\textbackslash iota}        \\
	$K$           & \texttt{K}                          && $\mathrm{K}$  & \texttt{\textbackslash mathrm\{K\}} && $\kappa$      & \texttt{\textbackslash kappa}       \\
	$\varLambda$  & \texttt{\textbackslash varLambda}   && $\Lambda$     & \texttt{\textbackslash Lambda}      && $\lambda$     & \texttt{\textbackslash lambda}      \\
	$M$           & \texttt{M}                          && $\mathrm{M}$  & \texttt{\textbackslash mathrm\{M\}} && $\mu$         & \texttt{\textbackslash mu}          \\
	$N$           & \texttt{N}                          && $\mathrm{N}$  & \texttt{\textbackslash mathrm\{N\}} && $\nu$         & \texttt{\textbackslash nu}          \\
	$\varXi$      & \texttt{\textbackslash varXi}       && $\Xi$         & \texttt{\textbackslash Xi}          && $\xi$         & \texttt{\textbackslash xi}          \\
	$O$           & \texttt{O}                          && $\mathrm{O}$  & \texttt{\textbackslash mathrm\{O\}} && $\mathrm{o}$  & \texttt{\textbackslash mathrm\{o\}} \\
	$\varPi$      & \texttt{\textbackslash varPi}       && $\Pi$         & \texttt{\textbackslash Pi}          && $\pi$         & \texttt{\textbackslash pi}          \\
	$P$           & \texttt{P}                          && $\mathrm{P}$  & \texttt{\textbackslash mathrm\{P\}} && $\rho$        & \texttt{\textbackslash rho}         \\
	$\varSigma$   & \texttt{\textbackslash varSigma}    && $\Sigma$      & \texttt{\textbackslash Sigma}       && $\sigma$      & \texttt{\textbackslash sigma}       \\
	$T$           & \texttt{T}                          && $\mathrm{T}$  & \texttt{\textbackslash mathrm\{T\}} && $\tau$        & \texttt{\textbackslash tau}         \\
	$\varUpsilon$ & \texttt{\textbackslash varUpsilon}  && $\Upsilon$    & \texttt{\textbackslash Upsilon}     && $\upsilon$    & \texttt{\textbackslash upsilon}     \\
	$\varPhi$     & \texttt{\textbackslash varPhi}      && $\Phi$        & \texttt{\textbackslash Phi}         && $\phi$        & \texttt{\textbackslash phi}         \\
	$X$           & \texttt{X}                          && $\mathrm{X}$  & \texttt{\textbackslash mathrm\{X\}} && $\chi$        & \texttt{\textbackslash chi}         \\
	$\varPsi$     & \texttt{\textbackslash varPsi}      && $\Psi$        & \texttt{\textbackslash Psi}         && $\psi$        & \texttt{\textbackslash psi}         \\
	$\varOmega$   & \texttt{\textbackslash varOmega}    && $\Omega$      & \texttt{\textbackslash Omega}       && $\omega$      & \texttt{\textbackslash omega}       \\
	\hline
\end{tabular}
\end{center}
\end{table}

%-------------------------------------------------------------------------------
\section{文字强调}

\underline{underlined} % 下划线

\uline{underlined}     % 下划线

\emph{强调} 版本

%-------------------------------------------------------------------------------
\section{章节和目录}

\begin{table}[h]
\caption{章节和目录}
\begin{center}
\begin{tabular}{l l}
	\hline
	\texttt{\textbackslash chapter\{\textcolor{gray}{title}\}}       & 章 \\
	\texttt{\textbackslash section\{\textcolor{gray}{title}\}}       & 节 \\
	\texttt{\textbackslash subsection\{\textcolor{gray}{title}\}}    & 小节 \\
	\texttt{\textbackslash subsubsection\{\textcolor{gray}{title}\}} & 小小节 \\
	\texttt{\textbackslash paragraph\{\textcolor{gray}{title}\}}     & 段 \\
	\texttt{\textbackslash subparagraph\{\textcolor{gray}{title}\}}  & 小段 \\
	\texttt{\textbackslash tableofcontents}                          & 目录 \\
	\hline
\end{tabular}
\end{center}
\end{table}

%===============================================================================
\chapter{集合、映射和二元运算}

%-------------------------------------------------------------------------------
\section{集合和集合的笛卡尔积}

一般的,\emph{集合}(\href{http://mathworld.wolfram.com/Set.html}{Set},简称\emph{集})是指具有某种特定性质的事物的总体,组成这个集合的事物称为该集合的\emph{元素}(\href{http://mathworld.wolfram.com/Element.html}{Element},简称\emph{元}).

%通常使用大写拉丁字母 A,B,C,\cdots{} 表示集合,用小写拉丁字母 a,b,c,\cdots{} 表示集合的元素


% 附录
\appendix

%===============================================================================
\chapter{参考书目}

\begin{enumerate}
	\item 全国信息与文献标准化技术委员会.信息与文献 参考文献著录规则:GB/T 7714—2015 [S/OL].北京:中国标准出版社,2015.\href{http://www.scal.edu.cn/dxtsgxb/201906120155}{超链接}.
	\item 同济大学数学系.高等数学[M].第六版.北京:高等教育出版社,2007.
\end{enumerate}

\end{document}
