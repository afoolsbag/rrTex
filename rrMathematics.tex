% !TeX encoding = UTF-8
%% !TeX program = XeLaTeX

\documentclass{ctexbook}
% 导言区开始

% 布局
\usepackage{geometry}
\geometry{margin=1in}
\usepackage{layout}

%\usepackage{amsmath}                     % 由 mathtools 替代
\usepackage{amssymb}                      % 数学符号支持
%\usepackage{color}                       % 由 xcolor 替代
\usepackage{enumitem}                     % 列表缩进支持
\usepackage{latexsym}                     % 数学符号支持
\usepackage{mathtools}                    % 数学支持
\usepackage{multirow}                     % 纵向合并单元格
\usepackage{syntonly}                     % 快速语法检查支持
\usepackage{ulem}                         % 下划线支持
\usepackage[dvipsnames,svgnames]{xcolor}  % 色彩支持

% 参考文献
\usepackage{gbt7714}
%\usepackage[backend=biber,style=gb7714-2015]{biblatex}
%\addbibresource[location=local]{rrMathematics.bib}

% 超链接
\usepackage{hyperref}

\hypersetup{colorlinks=true, unicode=true}  %
%\syntaxonly                                % 仅进行语法检查而不生成文档,加快编译速度

% zhengrr's color card
\newcommand{\rrcc}[1]{\fcolorbox{gray}{#1}{ } \textsl{#1}}

% zhengrr's unicode number
\newcommand{\rrun}[1]{\colorbox{Mulberry}{\href{https://unicode-table.com/#1/}{\color{White}\ttfamily{}\bfseries{}U+#1}}}

\providecommand{\Unicode}{
	\ttfamily
	Unicode\textregistered
}

\begin{document}
% 正文区开始

% 标题页
\title{rrMathematics}
\author{zhengrr}
\date{\today}
\maketitle

% 目录页
\tableofcontents

\chapter{\LaTeXe}

\section{排版文本}

\subsection{转义字符}

\begin{table}[h!]
	\centering
	\begin{minipage}[t]{0.46\textwidth}
		\centering
		\begin{tabular}{c p{7em} p{5em}}
			\hline
			\#             & \verb|\#|             & 井号 \\
			\$             & \verb|\$|             & 美刀符 \\
			\%             & \verb|\%|             & 百分号 \\
			\&             & \verb|\&|             & 和号 \\
			\textbackslash & \verb|\textbackslash| & 反斜线 \\
			\hline
		\end{tabular}
	\end{minipage}
	\qquad
	\begin{minipage}[t]{0.46\textwidth}
		\centering
		\begin{tabular}{c p{7em} p{5em}}
			\hline
			\^{}           & \verb|\^{}|           & 脱字符 \\
			\_             & \verb|\_|             & 下划线 \\
			\{             & \verb|\{|             & 左花括号 \\
			\}             & \verb|\}|             & 右花括号 \\
			\~{}           & \verb|\~{}|           & 波浪号 \\
			\hline
		\end{tabular}
	\end{minipage}
\end{table}

\subsection{标点符号}

\begin{table}[h!]
	\centering
	\begin{minipage}[t]{0.46\textwidth}
		\centering
		\begin{tabular}{c p{7em} p{5em}}
			\hline
			-          & \verb|-|          & 连字符 \\
			--         & \verb|--|         & 连接号 \\
			---        & \verb|---|        & 破折号 \\
			`'         & \verb|`'|         & 单引号 \\
			``''       & \verb|``''|       & 双引号 \\
			\P         & \verb|\P|         & 段落符 \\
			\hline
		\end{tabular}
	\end{minipage}
	\qquad
	\begin{minipage}[t]{0.46\textwidth}
		\centering
		\begin{tabular}{c p{7em} p{5em}}
			\hline
			\S         & \verb|\S|         & 分节符 \\
			\copyright & \verb|\copyright| & 版权符 \\
			\dag       & \verb|\dag|       & 剑标 \\
			\ddag      & \verb|\ddag|      & 双剑标 \\
			\dots      & \verb|\dots|      & 省略号 \\
			\pounds    & \verb|\pounds|    & 英镑符 \\
			\hline
		\end{tabular}
	\end{minipage}
\end{table}

\newpage
\subsection{文本样式}

\begin{table}[h!]
	\centering
	\begin{tabular}{l l l l}
		\hline
		\textrm{roman}           & \verb|\textrm{|{\color{gray}\verb|text|}\verb|}|     & \verb|\rmfamily|   & 罗马体(衬线字体) \\
		\textsf{sans serif}      & \verb|\textsf{|{\color{gray}\verb|text|}\verb|}|     & \verb|\sffamily|   & 无衬线字体 \\
		\texttt{typewriter}      & \verb|\texttt{|{\color{gray}\verb|text|}\verb|}|     & \verb|\ttfamily|   & 等宽字体 \\
		\hline
		\textbf{bold face}       & \verb|\textbf{|{\color{gray}\verb|text|}\verb|}|     & \verb|\bfseries|   & 粗体 \\
		\textmd{medium}          & \verb|\textmd{|{\color{gray}\verb|text|}\verb|}|     & \verb|\mdseries|   & 中等粗细 \\
		\hline
		\textit{italic}          & \verb|\textit{|{\color{gray}\verb|text|}\verb|}|     & \verb|\itshape|    & 意大利斜体 \\
		\textsl{slanted}         & \verb|\textsl{|{\color{gray}\verb|text|}\verb|}|     & \verb|\slshape|    & 倾斜体 \\
		\textsc{Small Caps}      & \verb|\textsc{|{\color{gray}\verb|text|}\verb|}|     & \verb|\scshape|    & 小写字母大写 \\
		\textup{upright}         & \verb|\textup{|{\color{gray}\verb|text|}\verb|}|     & \verb|\upshape|    & 直立体 \\
		\hline
		\emph{emphasized}        & \verb|\emph{|{\color{gray}\verb|text|}\verb|}|       & \verb|\em|         & 强调(默认为斜体) \\
		\textnormal{normal font} & \verb|\textnormal{|{\color{gray}\verb|text|}\verb|}| & \verb|\normalfont| & 默认字体 \\
		\uline{underlined}       & \verb|\uline{|{\color{gray}\verb|text|}\verb|}|      &                    & 下划线 \\
		\underline{underlined}   & \verb|\underline{|{\color{gray}\verb|text|}\verb|}|  &                    & 下划线 \\
		\hline
	\end{tabular}
\end{table}

\subsection{文本尺寸}

\begin{table}[h!]
	\centering
	\begin{tabular}{c c c c c | l}
		\verb|\tiny| & \verb|\scriptsize|       & \verb|\footnotesize|         & \verb|\small|  &                          & \verb|\normalsize| \\
		\hline
		{\tiny tiny} & {\scriptsize scriptsize} & {\footnotesize footnotesize} & {\small small} & {\normalsize normalsize} &                    \\
            &                          &                              &                & {\large large}           & \verb|\large|      \\
            &                          &                              &                & {\Large Large}           & \verb|\Large|      \\
            &                          &                              &                & {\LARGE LARGE}           & \verb|\LARGE|      \\
            &                          &                              &                & {\huge huge}             & \verb|\huge|       \\
            &                          &                              &                & {\Huge Huge}             & \verb|\Huge|       \\
	\end{tabular}
\end{table}

\newpage
\subsection{文本颜色}

依赖 \href{https://ctan.org/pkg/xcolor}{\texttt{xcolor}} 宏包提供颜色支持:

\begin{table}[h!]
	\centering
	{
		\slshape
		\begin{tabular}{l l l l l}
			\hline
			\rrcc{black}          & \rrcc{red}         & \rrcc{green}        & \rrcc{blue}        &                       \\
			\rrcc{white}          & \rrcc{cyan}        & \rrcc{magenta}      & \rrcc{yellow}      &                       \\
			\hline
			\rrcc{darkgray}       & \rrcc{gray}        & \rrcc{lightgray}    &                    &                       \\
			\rrcc{brown}          & \rrcc{olive}       & \rrcc{orange}       & \rrcc{lime}        &                       \\
			\rrcc{purple}         & \rrcc{teal}        & \rrcc{violet}       & \rrcc{pink}        &                       \\
			\hline
			\rrcc{Apricot}        & \rrcc{Cyan}        & \rrcc{Mahogany}     & \rrcc{ProcessBlue} & \rrcc{SpringGreen}    \\
			\rrcc{Aquamarine}     & \rrcc{Dandelion}   & \rrcc{Maroon}       & \rrcc{Purple}      & \rrcc{Tan}            \\
			\rrcc{Bittersweet}    & \rrcc{DarkOrchid}  & \rrcc{Melon}        & \rrcc{RawSienna}   & \rrcc{TealBlue}       \\
			\rrcc{Black}          & \rrcc{Emerald}     & \rrcc{MidnightBlue} & \rrcc{Red}         & \rrcc{Thistle}        \\
			\rrcc{Blue}           & \rrcc{ForestGreen} & \rrcc{Mulberry}     & \rrcc{RedOrange}   & \rrcc{Turquoise}      \\
			\rrcc{BlueGreen}      & \rrcc{Fuchsia}     & \rrcc{NavyBlue}     & \rrcc{RedViolet}   & \rrcc{Violet}         \\
			\rrcc{BlueViolet}     & \rrcc{Goldenrod}   & \rrcc{OliveGreen}   & \rrcc{Rhodamine}   & \rrcc{VioletRed}      \\
			\rrcc{BrickRed}       & \rrcc{Gray}        & \rrcc{Orange}       & \rrcc{RoyalBlue}   & \rrcc{White}          \\
			\rrcc{Brown}          & \rrcc{Green}       & \rrcc{OrangeRed}    & \rrcc{RoyalPurple} & \rrcc{WildStrawberry} \\
			\rrcc{BurntOrange}    & \rrcc{GreenYellow} & \rrcc{Orchid}       & \rrcc{RubineRed}   & \rrcc{Yellow}         \\
			\rrcc{CadetBlue}      & \rrcc{JungleGreen} & \rrcc{Peach}        & \rrcc{Salmon}      & \rrcc{YellowGreen}    \\
			\rrcc{CarnationPink}  & \rrcc{Lavender}    & \rrcc{Periwinkle}   & \rrcc{SeaGreen}    & \rrcc{YellowOrange}   \\
			\rrcc{Cerulean}       & \rrcc{LimeGreen}   & \rrcc{PineGreen}    & \rrcc{Sepia}       &                       \\
			\rrcc{CornflowerBlue} & \rrcc{Magenta}     & \rrcc{Plum}         & \rrcc{SkyBlue}     &                       \\
			\hline
		\end{tabular}
	}
\end{table}


\subsection{超链接}

依赖 \href{https://ctan.org/pkg/hyperref}{\texttt{hyperref}} 宏包提供超链接支持。


\newpage
\section{排版文档}

\begin{table}[h!]
	\centering
	\begin{minipage}[t]{0.46\textwidth}
		\centering
		\begin{tabular}{l l}
			\hline
			\verb|\chapter{|{\color{gray}\verb|title|}\verb|}|       & 章 \\
			\verb|\section{|{\color{gray}\verb|title|}\verb|}|       & 节 \\
			\verb|\subsection{|{\color{gray}\verb|title|}\verb|}|    & 小节 \\
			\verb|\subsubsection{|{\color{gray}\verb|title|}\verb|}| & 小小节 \\
			\verb|\paragraph{|{\color{gray}\verb|title|}\verb|}|     & 段 \\
			\verb|\subparagraph{|{\color{gray}\verb|title|}\verb|}|  & 小段 \\
			\verb|\tableofcontents|                                  & 目录 \\
			\hline
		\end{tabular}
	\end{minipage}
	\qquad
	\begin{minipage}[t]{0.46\textwidth}
		\centering
		\begin{tabular}{l l}
			\hline
			\hline
		\end{tabular}
	\end{minipage}
\end{table}

\newpage
\layout


\newpage
\section{排版公式}

美国数学协会(American Mathematical Society)提供  \AmS 宏集以扩展 \LaTeX 公式排版,其核心是 \href{https://ctan.org/pkg/amsmath}{\texttt{amsmath}} 宏包,对多行公式的排版提供了有力的支持.
此外,\href{https://ctan.org/pkg/amsfonts}{\texttt{amsfonts}} 宏包以及基于它的 \verb|amssymb| 宏包提供了丰富的数学符号;\href{https://ctan.org/pkg/amsthm}{\texttt{amsthm}} 宏包扩展了 \LaTeX 定理证明格式.

\href{https://ctan.org/tex-archive/info/symbols/comprehensive/}{此文档}提供了符号列表,\href{http://detexify.kirelabs.org/classify.html}{此工具}提供了符号搜索.
本节所述命令多用于 \LaTeX 数学模式,使用数学模式的方法有多种:

\begin{table}[h!]
	\centering
	\begin{tabular}{l p{6em} l}
		\verb|$|{\color{gray}\verb|formula|}\verb|$|   && 行内公式(\TeX) \\
		                                               && \\
		\verb|\(|{\color{gray}\verb|formula|}\verb|\)| && 行内公式(\LaTeX) \\
		                                               && \\
		\verb|$$|                                      && \\
		\verb|    |{\color{gray}\verb|formula|}        && 行间公式(\TeX) \\
		\verb|$$|                                      && \\
		                                               && \\
		\verb|\[|                                      && \\
		\verb|    |{\color{gray}\verb|formula|}        && 无序号行间公式(\LaTeX) \\
		\verb|\]|                                      && \\
		                                               && \\
		\verb|\begin{displaymath}|                     && \\
		\verb|    |{\color{gray}\verb|formula|}        && 无序号行间公式(\LaTeX) \\
		\verb|\end{displaymath}|                       && \\
		                                               && \\
		\verb|\begin{equation*}|                       && \\
		\verb|    |{\color{gray}\verb|formula|}        && 无序号行间公式(\AmS) \\
		\verb|\end{equation*}|                         && \\
		                                               && \\
		\verb|\begin{equation}|                        && \\
		\verb|    |{\color{gray}\verb|formula|}        && 有序号行间公式(\LaTeX) \\
		\verb|\end{equation}|                          && \\
	\end{tabular}
\end{table}

\newpage
\subsection{希腊字母}

\begin{table}[h!]
	\centering
	\begin{tabular}{c@{ }l c@{ }l c@{ }l c@{ }l c@{ }l c@{ }l}
		\hline
		\multicolumn{2}{c}{\Unicode} & \multicolumn{4}{c}{\LaTeX{}} & \multicolumn{2}{c}{\Unicode} & \multicolumn{4}{c}{\LaTeX{}} \\
		\hline
		Α & \rrun{0391} &        $A$ & \verb|A|        &               &                    & α & \rrun{03B1} &   $\alpha$ & \verb|\alpha|   &               &                    \\
		Β & \rrun{0392} &        $B$ & \verb|B|        &               &                    & β & \rrun{03B2} &    $\beta$ & \verb|\beta|    &               &                    \\
		Γ & \rrun{0393} &   $\Gamma$ & \verb|\Gamma|   &   $\varGamma$ & \verb|\varGamma|   & γ & \rrun{03B3} &   $\gamma$ & \verb|\gamma|   &               &                    \\
		Δ & \rrun{0394} &   $\Delta$ & \verb|\Delta|   &   $\varDelta$ & \verb|\varDelta|   & δ & \rrun{03B4} &   $\delta$ & \verb|\delta|   &               &                    \\
		Ε & \rrun{0395} &        $E$ & \verb|E|        &               &                    & ε & \rrun{03B5} & $\epsilon$ & \verb|\epsilon| & $\varepsilon$ & \verb|\varepsilon| \\
		Ζ & \rrun{0396} &        $Z$ & \verb|Z|        &               &                    & ζ & \rrun{03B6} &    $\zeta$ & \verb|\zeta|    &               &                    \\
		Η & \rrun{0397} &        $H$ & \verb|H|        &               &                    & η & \rrun{03B7} &     $\eta$ & \verb|\eta|     &               &                    \\
		Θ & \rrun{0398} &   $\Theta$ & \verb|\Theta|   &   $\varTheta$ & \verb|\varTheta|   & θ & \rrun{03B8} &   $\theta$ & \verb|\theta|   &   $\vartheta$ & \verb|\vartheta|   \\
		Ι & \rrun{0399} &        $I$ & \verb|I|        &               &                    & ι & \rrun{03B9} &    $\iota$ & \verb|\iota|    &               &                    \\
		Κ & \rrun{039A} &        $K$ & \verb|K|        &               &                    & κ & \rrun{03BA} &   $\kappa$ & \verb|\kappa|   &   $\varkappa$ & \verb|\varkappa|   \\
		Λ & \rrun{039B} &  $\Lambda$ & \verb|\Lambda|  &  $\varLambda$ & \verb|\varLambda|  & λ & \rrun{03BB} &  $\lambda$ & \verb|\lambda|  &               &                    \\
		Μ & \rrun{039C} &        $M$ & \verb|M|        &               &                    & μ & \rrun{03BC} &      $\mu$ & \verb|\mu|      &               &                    \\
		Ν & \rrun{039D} &        $N$ & \verb|N|        &               &                    & ν & \rrun{03BD} &      $\nu$ & \verb|\nu|      &               &                    \\
		Ξ & \rrun{039E} &      $\Xi$ & \verb|\Xi|      &      $\varXi$ & \verb|\varXi|      & ξ & \rrun{03BE} &      $\xi$ & \verb|\xi|      &               &                    \\
		Ο & \rrun{039F} &        $O$ & \verb|O|        &               &                    & ο & \rrun{03BF} &        $o$ & \verb|o|        &               &                    \\
		Π & \rrun{03A0} &      $\Pi$ & \verb|\Pi|      &      $\varPi$ & \verb|\varPi|      & π & \rrun{03C0} &      $\pi$ & \verb|\pi|      &      $\varpi$ & \verb|\varpi|      \\
		Ρ & \rrun{03A1} &        $P$ & \verb|P|        &               &                    & ρ & \rrun{03C1} &     $\rho$ & \verb|\rho|     &     $\varrho$ & \verb|\varrho|     \\
		Σ & \rrun{03A3} &   $\Sigma$ & \verb|\Sigma|   &   $\varSigma$ & \verb|\varSigma|   & σ & \rrun{03C2} &   $\sigma$ & \verb|\sigma|   &   $\varsigma$ & \verb|\varsigma|   \\
		Τ & \rrun{03A4} &        $T$ & \verb|T|        &               &                    & τ & \rrun{03C3} &     $\tau$ & \verb|\tau|     &               &                    \\
		Υ & \rrun{03A5} & $\Upsilon$ & \verb|\Upsilon| & $\varUpsilon$ & \verb|\varUpsilon| & υ & \rrun{03C4} & $\upsilon$ & \verb|\upsilon| &               &                    \\
		Φ & \rrun{03A6} &     $\Phi$ & \verb|\Phi|     &     $\varPhi$ & \verb|\varPhi|     & φ & \rrun{03C5} &     $\phi$ & \verb|\phi|     &     $\varphi$ & \verb|\varphi|     \\
		Χ & \rrun{03A7} &        $X$ & \verb|X|        &               &                    & χ & \rrun{03C6} &     $\chi$ & \verb|\chi|     &               &                    \\
		Ψ & \rrun{03A8} &     $\Psi$ & \verb|\Psi|     & $\varPsi$     & \verb|\varPsi|     & ψ & \rrun{03C7} &     $\psi$ & \verb|\psi|     &               &                    \\
		Ω & \rrun{03A9} &   $\Omega$ & \verb|\Omega|   & $\varOmega$   & \verb|\varOmega|   & ω & \rrun{03C8} &   $\omega$ & \verb|\omega|   &               &                    \\
		Ͱ & \rrun{0370} &            &                 &               &                    & ͱ & \rrun{0371} &            &                 &               &                    \\
		Ͳ & \rrun{0372} &            &                 &               &                    & ͳ & \rrun{0373} &            &                 &               &                    \\
		Ͷ & \rrun{0376} &            &                 &               &                    & ͷ & \rrun{0377} &            &                 &               &                    \\
		Ϙ & \rrun{03D8} &            &                 &               &                    & ϙ & \rrun{03D9} &            &                 &               &                    \\
		Ϛ & \rrun{03DA} &            &                 &               &                    & ϛ & \rrun{03DB} &            &                 &               &                    \\
		Ϝ & \rrun{03DC} &            &                 &               &                    & ϝ & \rrun{03DD} & $\digamma$ & \verb|\digamma| &               &                    \\
		Ϟ & \rrun{03DE} &            &                 &               &                    & ϟ & \rrun{03DF} &            &                 &               &                    \\
		Ϡ & \rrun{03E0} &            &                 &               &                    & ϡ & \rrun{03E1} &            &                 &               &                    \\
		\hline
	\end{tabular}
\end{table}

\newpage
\subsection{拉丁字母}

\begin{table}[th]
	\centering
	\renewcommand\arraystretch{0.99}
	\begin{tabular}{ c@{ }l c@{ }l c@{ }l c@{ }l c@{ }l c@{ }l}
		\hline
		$A$ & \verb|A| & $\mathrm{A}$ & \verb|\mathrm{A}| & $\mathbb{A}$ & \verb|\mathbb{A}| & $a$ & \verb|a| & $\mathrm{a}$ & \verb|\mathrm{a}| & $\mathbf{a}$ & \verb|\mathbf{a}| \\
		$B$ & \verb|B| & $\mathrm{B}$ & \verb|\mathrm{B}| & $\mathbb{B}$ & \verb|\mathbb{B}| & $b$ & \verb|b| & $\mathrm{b}$ & \verb|\mathrm{b}| & $\mathbf{b}$ & \verb|\mathbf{b}| \\
		$C$ & \verb|C| & $\mathrm{C}$ & \verb|\mathrm{C}| & $\mathbb{C}$ & \verb|\mathbb{C}| & $c$ & \verb|c| & $\mathrm{c}$ & \verb|\mathrm{c}| & $\mathbf{c}$ & \verb|\mathbf{c}| \\
		$D$ & \verb|D| & $\mathrm{D}$ & \verb|\mathrm{D}| & $\mathbb{D}$ & \verb|\mathbb{D}| & $d$ & \verb|d| & $\mathrm{d}$ & \verb|\mathrm{d}| & $\mathbf{d}$ & \verb|\mathbf{d}| \\
		$E$ & \verb|E| & $\mathrm{E}$ & \verb|\mathrm{E}| & $\mathbb{E}$ & \verb|\mathbb{E}| & $e$ & \verb|e| & $\mathrm{e}$ & \verb|\mathrm{e}| & $\mathbf{e}$ & \verb|\mathbf{e}| \\
		$F$ & \verb|F| & $\mathrm{F}$ & \verb|\mathrm{F}| & $\mathbb{F}$ & \verb|\mathbb{F}| & $f$ & \verb|f| & $\mathrm{f}$ & \verb|\mathrm{f}| & $\mathbf{f}$ & \verb|\mathbf{f}| \\
		$G$ & \verb|G| & $\mathrm{G}$ & \verb|\mathrm{G}| & $\mathbb{G}$ & \verb|\mathbb{G}| & $g$ & \verb|g| & $\mathrm{g}$ & \verb|\mathrm{g}| & $\mathbf{g}$ & \verb|\mathbf{g}| \\
		$H$ & \verb|H| & $\mathrm{H}$ & \verb|\mathrm{H}| & $\mathbb{H}$ & \verb|\mathbb{H}| & $h$ & \verb|h| & $\mathrm{h}$ & \verb|\mathrm{h}| & $\mathbf{h}$ & \verb|\mathbf{h}| \\
		$I$ & \verb|I| & $\mathrm{I}$ & \verb|\mathrm{I}| & $\mathbb{I}$ & \verb|\mathbb{I}| & $i$ & \verb|i| & $\mathrm{i}$ & \verb|\mathrm{i}| & $\mathbf{i}$ & \verb|\mathbf{i}| \\
		$J$ & \verb|J| & $\mathrm{J}$ & \verb|\mathrm{J}| & $\mathbb{J}$ & \verb|\mathbb{J}| & $j$ & \verb|j| & $\mathrm{j}$ & \verb|\mathrm{j}| & $\mathbf{j}$ & \verb|\mathbf{j}| \\
		$K$ & \verb|K| & $\mathrm{K}$ & \verb|\mathrm{K}| & $\mathbb{K}$ & \verb|\mathbb{K}| & $k$ & \verb|k| & $\mathrm{k}$ & \verb|\mathrm{k}| & $\mathbf{k}$ & \verb|\mathbf{k}| \\
		$L$ & \verb|L| & $\mathrm{L}$ & \verb|\mathrm{L}| & $\mathbb{L}$ & \verb|\mathbb{L}| & $l$ & \verb|l| & $\mathrm{l}$ & \verb|\mathrm{l}| & $\mathbf{l}$ & \verb|\mathbf{l}| \\
		$M$ & \verb|M| & $\mathrm{M}$ & \verb|\mathrm{M}| & $\mathbb{M}$ & \verb|\mathbb{M}| & $m$ & \verb|m| & $\mathrm{m}$ & \verb|\mathrm{m}| & $\mathbf{m}$ & \verb|\mathbf{m}| \\
		$N$ & \verb|N| & $\mathrm{N}$ & \verb|\mathrm{N}| & $\mathbb{N}$ & \verb|\mathbb{N}| & $n$ & \verb|n| & $\mathrm{n}$ & \verb|\mathrm{n}| & $\mathbf{n}$ & \verb|\mathbf{n}| \\
		$O$ & \verb|O| & $\mathrm{O}$ & \verb|\mathrm{O}| & $\mathbb{O}$ & \verb|\mathbb{O}| & $o$ & \verb|o| & $\mathrm{o}$ & \verb|\mathrm{o}| & $\mathbf{o}$ & \verb|\mathbf{o}| \\
		$P$ & \verb|P| & $\mathrm{P}$ & \verb|\mathrm{P}| & $\mathbb{P}$ & \verb|\mathbb{P}| & $p$ & \verb|p| & $\mathrm{p}$ & \verb|\mathrm{p}| & $\mathbf{p}$ & \verb|\mathbf{p}| \\
		$Q$ & \verb|Q| & $\mathrm{Q}$ & \verb|\mathrm{Q}| & $\mathbb{Q}$ & \verb|\mathbb{Q}| & $q$ & \verb|q| & $\mathrm{q}$ & \verb|\mathrm{q}| & $\mathbf{q}$ & \verb|\mathbf{q}| \\
		$R$ & \verb|R| & $\mathrm{R}$ & \verb|\mathrm{R}| & $\mathbb{R}$ & \verb|\mathbb{R}| & $r$ & \verb|r| & $\mathrm{r}$ & \verb|\mathrm{r}| & $\mathbf{r}$ & \verb|\mathbf{r}| \\
		$S$ & \verb|S| & $\mathrm{S}$ & \verb|\mathrm{S}| & $\mathbb{S}$ & \verb|\mathbb{S}| & $s$ & \verb|s| & $\mathrm{s}$ & \verb|\mathrm{s}| & $\mathbf{s}$ & \verb|\mathbf{s}| \\
		$T$ & \verb|T| & $\mathrm{T}$ & \verb|\mathrm{T}| & $\mathbb{T}$ & \verb|\mathbb{T}| & $t$ & \verb|t| & $\mathrm{t}$ & \verb|\mathrm{t}| & $\mathbf{t}$ & \verb|\mathbf{t}| \\
		$U$ & \verb|U| & $\mathrm{U}$ & \verb|\mathrm{U}| & $\mathbb{U}$ & \verb|\mathbb{U}| & $u$ & \verb|u| & $\mathrm{u}$ & \verb|\mathrm{u}| & $\mathbf{u}$ & \verb|\mathbf{u}| \\
		$V$ & \verb|V| & $\mathrm{V}$ & \verb|\mathrm{V}| & $\mathbb{V}$ & \verb|\mathbb{V}| & $v$ & \verb|v| & $\mathrm{v}$ & \verb|\mathrm{v}| & $\mathbf{v}$ & \verb|\mathbf{v}| \\
		$W$ & \verb|W| & $\mathrm{W}$ & \verb|\mathrm{W}| & $\mathbb{W}$ & \verb|\mathbb{W}| & $w$ & \verb|w| & $\mathrm{w}$ & \verb|\mathrm{w}| & $\mathbf{w}$ & \verb|\mathbf{w}| \\
		$X$ & \verb|X| & $\mathrm{X}$ & \verb|\mathrm{X}| & $\mathbb{X}$ & \verb|\mathbb{X}| & $x$ & \verb|x| & $\mathrm{x}$ & \verb|\mathrm{x}| & $\mathbf{x}$ & \verb|\mathbf{x}| \\
		$Y$ & \verb|Y| & $\mathrm{Y}$ & \verb|\mathrm{Y}| & $\mathbb{Y}$ & \verb|\mathbb{Y}| & $y$ & \verb|y| & $\mathrm{y}$ & \verb|\mathrm{y}| & $\mathbf{y}$ & \verb|\mathbf{y}| \\
		$Z$ & \verb|Z| & $\mathrm{Z}$ & \verb|\mathrm{Z}| & $\mathbb{Z}$ & \verb|\mathbb{Z}| & $z$ & \verb|z| & $\mathrm{z}$ & \verb|\mathrm{z}| & $\mathbf{z}$ & \verb|\mathbf{z}| \\
		\hline
	\end{tabular}
\end{table}

\newpage
\subsection{数学样式}

\begin{table}[h]
	\centering
	\begin{tabular}{l l l l}
		\hline
		$\mathrm{roman}$            & \verb|\mathrm{|{\color{gray}\verb|math|}\verb|}|     &                 & 罗马体(衬线字体),常用于单位和函数 \\
		$\mathsf{sans\ serif}$      & \verb|\mathsf{|{\color{gray}\verb|math|}\verb|}|     &                 & 无衬线字体 \\
		$\mathtt{typewriter}$       & \verb|\mathtt{|{\color{gray}\verb|math|}\verb|}|     &                 & 等宽字体 \\
		$\mathbf{bold face}$        & \verb|\mathbf{|{\color{gray}\verb|math|}\verb|}|     &                 & 粗体,常用于向量 \\
		$\mathcal{CALLIGRAPHY}$     & \verb|\mathcal{|{\color{gray}\verb|math|}\verb|}|    &                 & 书法体(仅大写),常用于层、概型和范畴 \\
		$\mathit{italic}$           & \verb|\mathit{|{\color{gray}\verb|math|}\verb|}|     &                 & 意大利斜体 \\
		$\mathnormal{normal\ font}$ & \verb|\mathnormal{|{\color{gray}\verb|math|}\verb|}| &                 & 默认字体 \\
		\hline
		$\mathbb{BLACKBOARD\ BOLD}$ & \verb|\mathbb{|{\color{gray}\verb|math|}\verb|}|     & \verb|amsfonts| & 黑板粗体(仅大写),常用于集合 \\
		$\mathfrak{fraktur}$        & \verb|\mathfrak{|{\color{gray}\verb|math|}\verb|}|   & \verb|amsfonts| & 一种德国风格粗体,常用于群和环 \\
		\hline
	\end{tabular}
\end{table}

\chapter{集合、映射和二元运算}

\section{集合}

\subsection{集合与元素间的关系}

一般的,\emph{集合}(\href{http://mathworld.wolfram.com/Set.html}{Set},简称\emph{集})是指具有某种特定性质的事物的总体,组成这个集合的事物称为该集合的\emph{元素}(\href{http://mathworld.wolfram.com/Element.html}{Element},简称\emph{元}).
通常用大写拉丁字母 $A$,$B$,$C$,$\cdots$ 表示集合,用小写拉丁字母 $a$,$b$,$c$,$\cdots$ 表示集合的元素.

如果元素 $e$ 是集合 $S$ 的元素,就说元素 $e$ \emph{属于}集合 $S$(记作 $e\in{}S$)或集合 $S$ \emph{拥有}元素 $e$(记作 $S\ni{}e$).
如果元素 $e$ 不是集合 $S$ 的元素,就说元素 $e$ \emph{不属于}集合 $S$(记作 $e\not\in{}S$) 或集合 $S$ \emph{不拥有}元素 $e$(记作 $S\not\ni{}e$).
一个集合,若它只拥有限个元素,则称为\emph{有限集};不是有限集的集合称为\emph{无限集}.

\begin{table}[h]
	\centering
	\begin{tabular}{l l l}
		\hline
		$e\in{}S$     & 元素 $e$ 属于集合 $S$   & \multirow{2}{*}{元素 $e$ 是集合 $S$ 的元素}   \\
		$S\ni{}e$     & 集合 $S$ 拥有元素 $e$   &                                               \\
		$e\not\in{}S$ & 元素 $e$ 不属于集合 $S$ & \multirow{2}{*}{元素 $e$ 不是集合 $S$ 的元素} \\
		$S\not\ni{}e$ & 集合 $S$ 不拥有元素 $e$ &                                               \\
		\hline
	\end{tabular}
\end{table}

通常使用\emph{列举法}、\emph{描述法}或\emph{文氏图}(\href{http://mathworld.wolfram.com/VennDiagram.html}{Venn Diagram})表示集合:
例如,由元素 $e_1$,$e_2$,$\cdots$,$e_n$ 组成的集合 $S_1$ 可表示为
\[ S_1 = \{ e_1, e_2, \cdots, e_n \} \text{;} \]
由具有某种性质 $P$ 的元素 $e$ 的全体组成的集合 $S_2$ 可表示为
\[ S_2 = \{ e | e \text{具有性质} P \} \text{.} \]

\subsection{集合与集合间的关系}

如何集合 $V$ 的元素都是集合 $W$ 的元素,则称集合 $V$ 是集合 $W$ 的\emph{子集}(\href{http://mathworld.wolfram.com/Subset.html}{Subset}),记作 $V\subset{}W$(读作集合 $V$ \emph{包含于}集合 $W$);
或称集合 $W$ 是集合 $V$ 的\emph{超集}(\href{http://mathworld.wolfram.com/Superset.html}{Superset}),记作 $W\supset{}V$(读作集合 $W$ \emph{包含}集合 $V$).

如果集合 $S_1$ 的元素都是集合 $S_2$ 的元素,且集合 $S_2$ 的元素也都是集合 $S_1$ 的元素,即集合 $S_1$ 与集合 $S_2$ 互相包含,则称集合 $S_1$ 与集合 $S_2$ \emph{相等},记作 $S_1=S_2$(读作集合 $S_1$ \emph{等于} $S_2$).

如果集合 $V$ 包含于集合 $W$,且它们不相等,则称集合 $V$ 是集合 $W$ 的\emph{真子集}(\href{http://mathworld.wolfram.com/ProperSubset.html}{Proper Subset}),记作 $V\subsetneqq{}W$(读作集合 $V$ \emph{真包含于}集合 $W$);
或称集合 $W$ 是集合 $V$ 的\emph{真超集}(\href{http://mathworld.wolfram.com/ProperSuperset.html}{Proper Superset}),记作 $W\supsetneqq{}V$(读作集合 $W$ \emph{真包含}集合 $V$).

\begin{table}[h]
	\centering
	\begin{tabular}{l l l}
		\hline
		$V\subset{}W$     & 集合 $V$ 包含于集合 $W$   & 集合 $V$ 是集合 $W$ 的子集   \\
		$W\supset{}V$     & 集合 $W$ 包含集合 $V$     & 集合 $W$ 是集合 $V$ 的超集   \\
		$S_1=S_2$         & 集合 $S_1$ 等于集合 $S_2$ & 集合 $S_1$ 与集合 $S_2$ 相等 \\
		$V\subsetneqq{}W$ & 集合 $V$ 真包含于集合 $W$ & 集合 $V$ 是集合 $W$ 的子集   \\
		$W\supsetneqq{}V$ & 集合 $W$ 真包含集合 $V$   & 集合 $W$ 是集合 $V$ 的超集   \\
		\hline
	\end{tabular}
\end{table}

不拥有任何元素的集合称为\emph{空集}(\href{http://mathworld.wolfram.com/EmptySet.html}{Empty Set}),记作 $\emptyset$,规定空集是任何集合的子集;
在指定上下文中,拥有所有元素的集合称作\emph{全集}(\href{http://mathworld.wolfram.com/UniversalSet.html}{Universe}),全集通常用 $U$ 表示.

\subsection{集合与集合间的运算}

% 附录
\appendix

\chapter{参考文献}

\begin{enumerate}
	\item Wikibooks.LaTeX [M/OL].\url{https://wikibooks.org/wiki/LaTeX}.
	\item Tobias Oetiker,Hubert Partl,Irene Hyna,Elisabeth Schlegl.一份(不太)简短的 \LaTeXe 介绍 [M/OL].C\TeX 小组,译.\newline
	      \url{http://mirrors.ctan.org/info/lshort/chinese/lshort-zh-cn.pdf}.
	\item CTEX : OnlineDocuments [EB/OL].\url{http://www.ctex.org/OnlineDocuments}.
	\item 全国信息与文献标准化技术委员会.信息与文献 参考文献著录规则:GB/T 7714—2015 [S/OL].北京:中国标准出版社,2015.\newline
	      \url{http://www.scal.edu.cn/dxtsgxb/201906120155}.
	\item 同济大学数学系.高等数学[M].第七版.北京:高等教育出版社,2014.
	\item 同济大学数学系.高等数学[M].第六版.北京:高等教育出版社,2007.
\end{enumerate}

\end{document}
