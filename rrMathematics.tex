% !TeX encoding = UTF-8
%% !TeX program = XeLaTeX

% LaTeX 命令和环境
%
% \command_name[optional_arguments...]{mandatory_arguments...}
%
% \begin{environment_name}[optional_arguments...]{manadatory_arguments...}
% \end{enviroment_name}

% 指定文档类
\documentclass{ctexbook}

% 导言区:调用宏包、全局设置
\usepackage{amsmath}   % https://ctan.org/pkg/amsmath   数学支持
\usepackage{amssymb}   % https://ctan.org/pkg/amsmath   数学支持
\usepackage{enumitem}  % https://ctan.org/pkg/enumitem  列表缩进支持
\usepackage{latexsym}  % https://ctan.org/pkg/latexsym  符号支持
\usepackage{syntonly}  % https://ctan.org/pkg/syntonly  快速语法检查支持
\usepackage{ulem}      % https://ctan.org/pkg/ulem      下划线支持
\usepackage{xcolor}    % https://ctan.org/pkg/color     彩色字符支持

\usepackage{hyperref}  % https://ctan.org/pkg/hyperref  超链接引用支持,为减少冲突,将此宏包置于其它宏包之后引入

\hypersetup{colorlinks=true, unicode=true}
%\syntaxonly                                 % 仅进行语法检查而不生成文档,加快编译速度

% 指示文档内容
\begin{document}

%===============================================================================
% 标题
\title{rrMathematics}
\author{zhengrr}
\date{\today}
\maketitle

%===============================================================================
% 目录
\tableofcontents

%===============================================================================
\chapter{\TeX / \LaTeX / \LaTeXe}

%-------------------------------------------------------------------------------
\section{排版文本}

\subsection{转义字符}
\begin{table}[h]
	\centering
	\begin{minipage}[t]{0.46\textwidth}
		\centering
		\begin{tabular}{c p{7em} p{5em}}
			\hline
			\#             & \verb|\#|             & 井号 \\
			\$             & \verb|\$|             & 美元符 \\
			\%             & \verb|\%|             & 百分号 \\
			\&             & \verb|\&|             & 和号 \\
			\textbackslash & \verb|\textbackslash| & 反斜线 \\
			\hline
		\end{tabular}
	\end{minipage}
	\qquad
	\begin{minipage}[t]{0.46\textwidth}
		\centering
		\begin{tabular}{c p{7em} p{5em}}
			\hline
			\^{}           & \verb|\^{}|           & 脱字符 \\
			\_             & \verb|\_|             & 下划线 \\
			\{             & \verb|\{|             & 左花括号 \\
			\}             & \verb|\}|             & 右花括号 \\
			\~{}           & \verb|\~{}|           & 波浪号 \\
			\hline
		\end{tabular}
	\end{minipage}
\end{table}

\subsection{标点符号}
\begin{table}[h]
	\centering
	\begin{minipage}[t]{0.46\textwidth}
		\centering
		\begin{tabular}{c p{7em} p{5em}}
			\hline
			-          & \verb|-|          & 连字符 \\
			--         & \verb|--|         & 连接号 \\
			---        & \verb|---|        & 破折号 \\
			`'         & \verb|`'|         & 单引号 \\
			``''       & \verb|``''|       & 双引号 \\
			\P         & \verb|\P|         & 段落符 \\
			\hline
		\end{tabular}
	\end{minipage}
	\qquad
	\begin{minipage}[t]{0.46\textwidth}
		\centering
		\begin{tabular}{c p{7em} p{5em}}
			\hline
			\S         & \verb|\S|         & 分节符 \\
			\copyright & \verb|\copyright| & 版权符 \\
			\dag       & \verb|\dag|       & 剑标 \\
			\ddag      & \verb|\ddag|      & 双剑标 \\
			\dots      & \verb|\dots|      & 省略号 \\
			\pounds    & \verb|\pounds|    & 英镑符 \\
			\hline
		\end{tabular}
	\end{minipage}
\end{table}

\newpage
\subsection{字体样式}
\begin{table}[h]
	\centering
	\begin{tabular}{l l l l}
		\hline
		\hline
		\textrm{roman}           & \verb|\textrm{|{\color{gray}\verb|text|}\verb|}|     & \verb|\rmfamily|   & 罗马体(衬线字体) \\
		\textsf{sans serif}      & \verb|\textsf{|{\color{gray}\verb|text|}\verb|}|     & \verb|\sffamily|   & 无衬线字体 \\
		\texttt{typewriter}      & \verb|\texttt{|{\color{gray}\verb|text|}\verb|}|     & \verb|\ttfamily|   & 等宽字体 \\
		\hline
		\textbf{bold face}       & \verb|\textbf{|{\color{gray}\verb|text|}\verb|}|     & \verb|\bfseries|   & 粗体 \\
		\textmd{medium}          & \verb|\textmd{|{\color{gray}\verb|text|}\verb|}|     & \verb|\mdseries|   & 中等粗细 \\
		\hline
		\textit{italic}          & \verb|\textit{|{\color{gray}\verb|text|}\verb|}|     & \verb|\itshape|    & 意大利斜体 \\
		\textsl{slanted}         & \verb|\textsl{|{\color{gray}\verb|text|}\verb|}|     & \verb|\slshape|    & 倾斜体 \\
		\textsc{Small Caps}      & \verb|\textsc{|{\color{gray}\verb|text|}\verb|}|     & \verb|\scshape|    & 小写字母大写 \\
		\textup{upright}         & \verb|\textup{|{\color{gray}\verb|text|}\verb|}|     & \verb|\upshape|    & 直立体 \\
		\hline
		\emph{emphasized}        & \verb|\emph{|{\color{gray}\verb|text|}\verb|}|       & \verb|\em|         & 强调(默认为斜体) \\
		\textnormal{normal font} & \verb|\textnormal{|{\color{gray}\verb|text|}\verb|}| & \verb|\normalfont| & 默认字体 \\
		\uline{underlined}       & \verb|\uline{|{\color{gray}\verb|text|}\verb|}|      &                    & 下划线 \\
		\underline{underlined}   & \verb|\underline{|{\color{gray}\verb|text|}\verb|}|  &                    & 下划线 \\
		\hline
		\hline
	\end{tabular}
\end{table}

\subsection{字号}
\begin{table}[h]
	\centering
	\begin{tabular}{c c c c c l}
		\verb|\tiny| & \verb|\scriptsize|       & \verb|\footnotesize|         & \verb|\small|  &                          & \verb|\normalsize| \\
		{\tiny tiny} & {\scriptsize scriptsize} & {\footnotesize footnotesize} & {\small small} & {\normalsize normalsize} &                    \\
		             &                          &                              &                & {\large large}           & \verb|\large|      \\
		             &                          &                              &                & {\Large Large}           & \verb|\Large|      \\
		             &                          &                              &                & {\LARGE LARGE}           & \verb|\LARGE|      \\
		             &                          &                              &                & {\huge huge}             & \verb|\huge|       \\
		             &                          &                              &                & {\Huge Huge}             & \verb|\Huge|       \\
	\end{tabular}
\end{table}

\newpage
\subsection{颜色}
依赖 \verb|xcolor| 宏包提供颜色支持:
\begin{table}[h]
	\centering
	\begin{tabular}{r r r r}
		\hline
		\verb|black| \fcolorbox{black}{black}{ }       & \verb|red| \fcolorbox{black}{red}{ }     & \verb|green| \fcolorbox{black}{green}{ }         & \verb|blue| \fcolorbox{black}{blue}{ }     \\
		\verb|white| \fcolorbox{black}{white}{ }       & \verb|cyan| \fcolorbox{black}{cyan}{ }   & \verb|magenta| \fcolorbox{black}{magenta}{ }     & \verb|yellow| \fcolorbox{black}{yellow}{ } \\
		\hline
		\verb|darkgray| \fcolorbox{black}{darkgray}{ } & \verb|gray| \fcolorbox{black}{gray}{ }   & \verb|lightgray| \fcolorbox{black}{lightgray}{ } &                                            \\
		\verb|brown| \fcolorbox{black}{brown}{ }       & \verb|olive| \fcolorbox{black}{olive}{ } & \verb|orange| \fcolorbox{black}{orange}{ }       & \verb|lime| \fcolorbox{black}{lime}{ }     \\
		\verb|purple| \fcolorbox{black}{purple}{ }     & \verb|teal| \fcolorbox{black}{teal}{ }   & \verb|violet| \fcolorbox{black}{violet}{ }       & \verb|pink| \fcolorbox{black}{pink}{ }     \\
		\hline
	\end{tabular}
\end{table}

%-------------------------------------------------------------------------------
\newpage
\section{排版文档}

\subsection{章节和目录}
\begin{table}[h]
	\centering
	\begin{tabular}{l l}
		\hline
		\verb|\chapter{|{\color{gray}\verb|title|}\verb|}|       & 章 \\
		\verb|\section{|{\color{gray}\verb|title|}\verb|}|       & 节 \\
		\verb|\subsection{|{\color{gray}\verb|title|}\verb|}|    & 小节 \\
		\verb|\subsubsection{|{\color{gray}\verb|title|}\verb|}| & 小小节 \\
		\verb|\paragraph{|{\color{gray}\verb|title|}\verb|}|     & 段 \\
		\verb|\subparagraph{|{\color{gray}\verb|title|}\verb|}|  & 小段 \\
		\verb|\tableofcontents|                                  & 目录 \\
		\hline
	\end{tabular}
\end{table}

%-------------------------------------------------------------------------------
\newpage
\section{排版公式}

美国数学协会(American Mathematical Society)提供 \AmS 宏集以扩展 \LaTeX 公式排版,其核心是 \verb|amsmath| 宏包,对多行公式的排版提供了有力的支持。此外,\verb|amsfonts| 宏包以及基于它的 \verb|amssymb| 宏包提供了丰富的数学符号;\verb|amsthm| 宏包扩展了 \LaTeX 定理证明格式。


本节中,墨色命令不依赖额外宏包,\textcolor{brown}{棕色}命令依赖 \verb|latexsym| 宏包,\textcolor{blue}{蓝色}命令依赖 \verb|amsmath| 宏包,\textcolor{teal}{凫色}命令依赖\verb|amssymb| 宏包。

本节所述命令多用于 \LaTeX 数学模式,使用数学模式的方法有多种:

\begin{table}[h]
	\centering
	\begin{tabular}{l l}
		\hline
		\verb|$|{\color{gray}\verb|formula|}\verb|$|                                             & 行内公式 \\
		\verb|\[|{\color{gray}\verb|formula|}\verb|\]|                                           & 无序号行间公式 \\
		\verb|\begin{displaymath}|{\color{gray}\verb|formula|}\verb|\end{displaymath}|           & 无序号行间公式 \\
		{\color{blue}\verb|\begin{equation*}|{\color{gray}\verb|formula|}\verb|\end{equation*}|} & 无序号行间公式 \\
		\verb|\begin{equation}|{\color{gray}\verb|formula|}\verb|\end{equation}|                 & 有序号行间公式 \\
		\hline
	\end{tabular}
\end{table}

\newpage
\subsection{希腊字母}
\begin{table}[h]
	\centering
	\begin{minipage}[t]{0.28\textwidth}
		\centering
		\begin{tabular}{c l}
			\hline
			$A$           & \verb|A|                         \\
			$B$           & \verb|B|                         \\
			$\varGamma$   & {\color{blue}\verb|\varGamma|}   \\
			$\varDelta$   & {\color{blue}\verb|\varDelta|}   \\
			$E$           & \verb|E|                         \\
			$Z$           & \verb|Z|                         \\
			$H$           & \verb|H|                         \\
			$\varTheta$   & {\color{blue}\verb|\varTheta|}   \\
			$I$           & \verb|I|                         \\
			$K$           & \verb|K|                         \\
			$\varLambda$  & {\color{blue}\verb|\varLambda|}  \\
			$M$           & \verb|M|                         \\
			$N$           & \verb|N|                         \\
			$\varXi$      & {\color{blue}\verb|\varXi|}      \\
			$O$           & \verb|O|                         \\
			$\varPi$      & {\color{blue}\verb|\varPi|}      \\
			$P$           & \verb|P|                         \\
			$\varSigma$   & {\color{blue}\verb|\varSigma|}   \\
			$T$           & \verb|T|                         \\
			$\varUpsilon$ & {\color{blue}\verb|\varUpsilon|} \\
			$\varPhi$     & {\color{blue}\verb|\varPhi|}     \\
			$X$           & \verb|X|                         \\
			$\varPsi$     & {\color{blue}\verb|\varPsi|}     \\
			$\varOmega$   & {\color{blue}\verb|\varOmega|}   \\
			\hline
		\end{tabular}
	\end{minipage}
	\qquad
	\begin{minipage}[t]{0.28\textwidth}
		\centering
		\begin{tabular}{c l}
			\hline
			$\mathrm{A}$  & \verb|\mathrm{A}|                \\
			$\mathrm{B}$  & \verb|\mathrm{B}|                \\
			$\Gamma$      & \verb|\Gamma|                    \\
			$\Delta$      & \verb|\Delta|                    \\
			$\mathrm{E}$  & \verb|\mathrm{E}|                \\
			$\mathrm{Z}$  & \verb|\mathrm{Z}|                \\
			$\mathrm{H}$  & \verb|\mathrm{H}|                \\
			$\Theta$      & \verb|\Theta|                    \\
			$\mathrm{I}$  & \verb|\mathrm{I}|                \\
			$\mathrm{K}$  & \verb|\mathrm{K}|                \\
			$\Lambda$     & \verb|\Lambda|                   \\
			$\mathrm{M}$  & \verb|\mathrm{M}|                \\
			$\mathrm{N}$  & \verb|\mathrm{N}|                \\
			$\Xi$         & \verb|\Xi|                       \\
			$\mathrm{O}$  & \verb|\mathrm{O}|                \\
			$\Pi$         & \verb|\Pi|                       \\
			$\mathrm{P}$  & \verb|\mathrm{P}|                \\
			$\Sigma$      & \verb|\Sigma|                    \\
			$\mathrm{T}$  & \verb|\mathrm{T}|                \\
			$\Upsilon$    & \verb|\Upsilon|                  \\
			$\Phi$        & \verb|\Phi|                      \\
			$\mathrm{X}$  & \verb|\mathrm{X}|                \\
			$\Psi$        & \verb|\Psi|                      \\
			$\Omega$      & \verb|\Omega|                    \\
			\hline
		\end{tabular}
	\end{minipage}
	\qquad
	\begin{minipage}[t]{0.28\textwidth}
		\centering
		\begin{tabular}{c l}
			\hline
			$\alpha$      & \verb|\alpha|                    \\
			$\beta$       & \verb|\beta|                     \\
			$\gamma$      & \verb|\gamma|                    \\
			$\delta$      & \verb|\delta|                    \\
			$\epsilon$    & \verb|\epsilon|                  \\
			$\zeta$       & \verb|\zeta|                     \\
			$\eta$        & \verb|\eta|                      \\
			$\theta$      & \verb|\theta|                    \\
			$\iota$       & \verb|\iota|                     \\
			$\kappa$      & \verb|\kappa|                    \\
			$\lambda$     & \verb|\lambda|                   \\
			$\mu$         & \verb|\mu|                       \\
			$\nu$         & \verb|\nu|                       \\
			$\xi$         & \verb|\xi|                       \\
			$\mathrm{o}$  & \verb|\mathrm{o}|                \\
			$\pi$         & \verb|\pi|                       \\
			$\rho$        & \verb|\rho|                      \\
			$\sigma$      & \verb|\sigma|                    \\
			$\tau$        & \verb|\tau|                      \\
			$\upsilon$    & \verb|\upsilon|                  \\
			$\phi$        & \verb|\phi|                      \\
			$\chi$        & \verb|\chi|                      \\
			$\psi$        & \verb|\psi|                      \\
			$\omega$      & \verb|\omega|                    \\
			\hline
		\end{tabular}
	\end{minipage}
\end{table}

%===============================================================================
\chapter{集合、映射和二元运算}

%-------------------------------------------------------------------------------
\section{集合和集合的笛卡尔积}

一般的,\emph{集合}(\href{http://mathworld.wolfram.com/Set.html}{Set},简称\emph{集})是指具有某种特定性质的事物的总体,组成这个集合的事物称为该集合的\emph{元素}(\href{http://mathworld.wolfram.com/Element.html}{Element},简称\emph{元}).

%通常使用大写拉丁字母 A,B,C,\cdots{} 表示集合,用小写拉丁字母 a,b,c,\cdots{} 表示集合的元素


% 附录
\appendix

%===============================================================================
\chapter{参考书目}

\begin{enumerate}
	\item Tobias Oetiker,Hubert Partl,Irene Hyna,Elisabeth Schlegl.一份(不太)简短的 \LaTeXe 介绍 [M/OL].C\TeX 小组,译.\newline
	      \url{http://mirrors.ctan.org/info/lshort/chinese/lshort-zh-cn.pdf}.
	\item 全国信息与文献标准化技术委员会.信息与文献 参考文献著录规则:GB/T 7714—2015 [S/OL].北京:中国标准出版社,2015.\newline
	      \url{http://www.scal.edu.cn/dxtsgxb/201906120155}.
	\item 同济大学数学系.高等数学[M].第六版.北京:高等教育出版社,2007.
\end{enumerate}

\end{document}
