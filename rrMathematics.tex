% !TeX encoding = UTF-8
% XeLaTeX

\documentclass{book}

% 导言区
%\usepackage{amsmath}     % https://ctan.org/pkg/amsmath
\usepackage{indentfirst} % https://ctan.org/pkg/indentfirst
\usepackage{syntonly}    % https://ctan.org/pkg/syntonly
\usepackage{ulem}        % https://ctan.org/pkg/ulem
\usepackage{xeCJK}       % https://ctan.org/pkg/xecjk

\usepackage{hyperref}    % https://ctan.org/pkg/hyperref

\hypersetup{colorlinks=true, unicode=true}
\setlength{\parindent}{2em}
%\syntaxonly


\begin{document}

%-------------------------------------------------------------------------------
% 前言
\frontmatter

%-------------------------------------------------------------------------------
% 标题
\title{rrMathematics}
\author{zhengrr}
\date{\today}
\maketitle

%-------------------------------------------------------------------------------
% 目录
\tableofcontents

%-------------------------------------------------------------------------------
% 正文
\mainmatter

\chapter{集合、映射和二元运算}

\section{集合和集合的笛卡尔积}

一般的,\emph{集合}(\href{http://mathworld.wolfram.com/Set.html}{Set},简称\emph{集})是指具有某种特定性质的事物的总体,组成这个集合的事物称为该集合的\emph{元素}(\href{http://mathworld.wolfram.com/Element.html}{Element},简称\emph{元}).

通常使用大写拉丁字母 A,B,C,\dots 表示集合,用小写拉丁字母 a,b,c,\dots 表示集合的元素

%-------------------------------------------------------------------------------
% 附录
\appendix

\chapter{参考书目}

\begin{enumerate}
	\item 全国信息与文献标准化技术委员会.信息与文献 参考文献著录规则:GB/T 7714—2015 [S/OL].北京:中国标准出版社,2015.\href{http://www.scal.edu.cn/dxtsgxb/201906120155}{超链接}.
	\item 同济大学数学系.高等数学[M].第六版.北京:高等教育出版社,2007.
\end{enumerate}

%-------------------------------------------------------------------------------
% 后记
\backmatter

\end{document}
