\chapter{数学基础}

本章主要参照\citeauthor{WangFt2001}先生的\citetitle{WangFt2001}\cite{WangFt2001}.

人类数学大体上经历了三个大的发展阶段:以几何数学为主体的初等数学阶段,以分析数学为主体的古典数学阶段,以集论数学为主体的现代数学阶段.

\section{数理逻辑}

\subsection{形式语言初步}

\rTermWref{形式语言}{FormalLanguage}{formal language}可以由\rTermWref{形式文法}{Grammar}{formal grammar}描述,形式文法是一种四元组$G=(N,\Sigma,P,S)$,其中:
\begin{itemize}
    \item 集$N$是由\rTerm{非终结符}{nonterminal symbols}组成的有限集;
    \item 集$\Sigma$是由\rTerm{终结符}{terminal symbols}组成的有限集,且$N\cap\Sigma=\emptyset$;集$V=N\cup\Sigma$称作\rTerm{词汇表}{vocabulary};
    \item 集$P$是由\rTerm{产生式规则}{production rules}组成的有限集;
    \item 元$S$是\rTerm{开始符号}{start symbol},且$S\in{}N$.
\end{itemize}

譬如,一种可以描述二进制整数的形式文法和它的一例推导示例:
\begin{multicols}{2}
    \begin{equation*}
        G_\text{binary} = (N, \Sigma, P, S), \begin{cases}
            N      = \{ S_\text{(start)}, D_\text{(digital)} \} \\
            \Sigma = \{ -, 0, 1 \}                              \\
            P      = \begin{cases}
                         \text{1.}\ S \to D  \\
                         \text{2.}\ S \to -D \\
                         \text{3.}\ D \to 0D \\
                         \text{4.}\ D \to 1D \\
                         \text{5.}\ D \to 0  \\
                         \text{6.}\ D \to 1  \\
                     \end{cases}                                \\
            S                                                   \\
        \end{cases}
    \end{equation*}
    \newline
    \begin{equation*}
        \begin{aligned}
            S &\underset{2}{\Rightarrow} -D    \\
              &\underset{4}{\Rightarrow} -1D   \\
              &\underset{3}{\Rightarrow} -10D  \\
              &\underset{4}{\Rightarrow} -101D \\
              &\underset{5}{\Rightarrow} -1010 \\
        \end{aligned}
    \end{equation*}
\end{multicols}

\subsection{命题演算初步}

\rTermWref{命题演算}{PropositionalCalculus}{propositional calculus}是一种\rTerm{形式系统}{formal system},可描述为$\mathcal{L}(\mathrm{A},\Omega,\mathrm{Z},\mathrm{I})$,其中:
\begin{itemize}
    \renewcommand{\labelitemii}{}
    \item 集$\rHatNote{\mathrm{A}}{``alpha''}$是由\rTerm{命题变量}{propositional variables}$p$、$q$、$r$、$\cdots$组成的集;
    \item 集$\rHatNote{\Omega}{``omega''}$是由\rTerm{逻辑联结词}{logical connectives}组成的集,可按元数将其划分为$\Omega_0$、$\Omega_1$和$\Omega_2$:
          \begin{itemize}
              \item $\Omega_0 = \{ \rHatNote{\bot}{恒假}, \rHatNote{\top}{恒真} \}$;
              \item $\Omega_1 = \{ \rHatNote{\lnot}{否定词} \}$;
              \item $\Omega_2 = \{ \rHatNote{\land}{合取词}, \rHatNote{\lor}{析取词}, \rHatNote{\to}{蕴含词}, \rHatNote{\leftrightarrow}{等价词}, \cdots \}$;
          \end{itemize}
    \item 集$\rHatNote{\mathrm{Z}}{``zeta''}$是由\rTerm{推理规则}{inference rules}组成的集:
          \begin{itemize}
              \item 1. 每个命题变量都是合式公式(下简称``公式'');
              \item 2. $\bot$和$\top$是公式;
              \item 3. 若$p$是公式,则$\lnot{}p$也是公式;
              \item 4. 若$p$和$q$是公式,则$p\land{}q$、$p\lor{}q$、$p\to{}q$、$p\leftrightarrow{}q$、$\cdots$也都是公式;
              \item 5. 除外都不是公式.
          \end{itemize}
    \item 集$\rHatNote{\mathrm{I}}{``iota''}$是由\rTerm{公理}{axioms}组成的集,即下表所述运算:
\end{itemize}

\begin{table}[h!]
    \centering
    \newcommand{\F}{{\color{red}$\bot$}}
    \newcommand{\T}{{\color{blue}$\top$}}
    \newcommand{\FoF}{\F$\circ$\F}
    \newcommand{\FoT}{\F$\circ$\T}
    \newcommand{\ToF}{\T$\circ$\F}
    \newcommand{\ToT}{\T$\circ$\T}
    \begin{tabular}{c c c c c c c c l}
        \hline
        \multicolumn{4}{c}{等价的公式}                                                                                                        & \multicolumn{4}{c}{真值表} & \multicolumn{1}{c}{助记} \\
        $p\circ{}q$               & $p\circ\lnot{}q$               & $\lnot{}p\circ{}q$               & $\lnot{}p\circ\lnot{}q$               & \FoF & \FoT & \ToF & \ToT  &                          \\
        \hline
        \multicolumn{4}{c}{$\top$}                                                                                                            & \T   & \T   & \T   & \T    & ``恒真''                 \\
        $p\uparrow{}q$            & $p\to\lnot{}q$                 & $\lnot{}p\gets{}q$               & $\lnot{}p\lor\lnot{}q$                & \T   & \T   & \T   & \F    & ``与非'',存假为真       \\
        $p\to{}q$                 & $p\uparrow\lnot{}q$            & $\lnot{}p\lor{}q$                & $\lnot{}p\gets\lnot{}q$               & \T   & \T   & \F   & \T    & ``蕴含''                 \\
        \multicolumn{4}{c}{$\lnot{}p$}                                                                                                        & \T   & \T   & \F   & \F    & ``非''                   \\
        $p\gets{}q$               & $p\lor\lnot{}q$                & $\lnot{}p\uparrow{}q$            & $\lnot{}p\to\lnot{}q$                 & \T   & \F   & \T   & \T    & ``蕴含于''               \\
        \multicolumn{4}{c}{$\lnot{}q$}                                                                                                        & \T   & \F   & \T   & \F    & ``非''                   \\
        $p\leftrightarrow{}q$     & $p\not\leftrightarrow\lnot{}q$ & $\lnot{}p\not\leftrightarrow{}q$ & $\lnot{}p\leftrightarrow\lnot{}q$     & \T   & \F   & \F   & \T    & ``同或'',相同为真       \\
        $p\downarrow{}q$          & $p\not\gets\lnot{}q$           & $\lnot{}p\not\to{}q$             & $\lnot{}p\land\lnot{}q$               & \T   & \F   & \F   & \F    & ``或非'',全假为真       \\
        $p\lor{}q$                & $p\gets\lnot{}q$               & $\lnot{}p\to{}q$                 & $\lnot{}p\uparrow\lnot{}q$            & \F   & \T   & \T   & \T    & ``或'',存真为真         \\
        $p\not\leftrightarrow{}q$ & $p\leftrightarrow\lnot{}q$     & $\lnot{}p\leftrightarrow{}q$     & $\lnot{}p\not\leftrightarrow\lnot{}q$ & \F   & \T   & \T   & \F    & ``异或'',相异为真       \\
        \multicolumn{4}{c}{$q$}                                                                                                               & \F   & \T   & \F   & \T    & ``公式''                 \\
        $p\not\gets{}q$           & $p\downarrow\lnot{}q$          & $\lnot{}p\land{}q$               & $\lnot{}p\not\to\lnot{}q$             & \F   & \T   & \F   & \F    & ``不蕴含于''             \\
        \multicolumn{4}{c}{$p$}                                                                                                               & \F   & \F   & \T   & \T    & ``公式''                 \\
        $p\not\to{}q$             & $p\land\lnot{}q$               & $\lnot{}p\downarrow{}q$          & $\lnot{}p\not\gets\lnot{}q$           & \F   & \F   & \T   & \F    & ``不蕴含''               \\
        $p\land{}q$               & $p\not\to\lnot{}q$             & $\lnot{}p\not\gets{}q$           & $\lnot{}p\downarrow\lnot{}q$          & \F   & \F   & \F   & \T    & ``与'',全真为真         \\
        \multicolumn{4}{c}{$\bot$}                                                                                                            & \F   & \F   & \F   & \F    & ``恒假''                 \\
        \hline
    \end{tabular}
\end{table}

\subsubsection{重言式和矛盾式}

若$p$、$q$、$r$是公式,可以证明下述等式恒成立:

\[ (p \to p)                                                               \equiv \top \tag{同一律} \]
\[ (\lnot{}p \land p)                                                      \equiv \bot \tag{矛盾律} \]
\[ (\lnot{}p \lor p)                                                       \equiv \top \tag{排中律} \]
\[ (\lnot\lnot{}p \leftrightarrow p)                                       \equiv \top \tag{双重否定律} \]

也可以证明下述合取运算规律恒成立:

\[ [(p \land q) \leftrightarrow (q \land p)]                               \equiv \top \tag{合取交换律} \]
\[ \{[(p \land q) \land r] \leftrightarrow [p \land (q \land r)]\}         \equiv \top \tag{合取结合律} \]
\[ \{[p \land (q \lor r)] \leftrightarrow [(p \land q) \lor (p \land r)]\} \equiv \top \tag{合取分配律} \]
\[ [\lnot(p \land q) \leftrightarrow (\lnot{}p \lor \lnot{}q)]             \equiv \top \tag{合取德·摩根律} \]

也可以证明下述析取运算规律恒成立:

\[ [(p \lor q) \leftrightarrow (q \lor p)]                                 \equiv \top \tag{析取交换律} \]
\[ \{[(p \lor q) \lor r] \leftrightarrow [p \lor (q \lor r)]\}             \equiv \top \tag{析取结合律} \]
\[ \{[p \lor (q \land r)] \leftrightarrow [(p \lor q) \land (p \lor r)]\}  \equiv \top \tag{析取分配律} \]
\[ [\lnot(p \lor q) \leftrightarrow (\lnot{}p \land \lnot{}q)]             \equiv \top \tag{析取德·摩根律} \]

也可以证明下述蕴含运算规律恒成立:

\[ [\lnot{}p \to (p \to q)]                                                \equiv \top \tag{否定前件律} \]
\[ [q \to (p \to q)]                                                       \equiv \top \tag{肯定后件律} \]
\[ \{[p \to (q \to r)] \to [(p \to q) \to (p \to r)]\}                     \equiv \top \tag{蕴含分配律} \]
\[ [(\lnot{}p \to \lnot{}q) \to (q \to p)]                                 \equiv \top \tag{蕴含换位律} \]
\[ [(\lnot{}p \to p) \to p]                                                \equiv \top \tag{否定肯定律} \]
\[ \{(p \to q) \to [(q \to r) \to (p \to r)]\}                             \equiv \top \tag{假设三段论} \]

\subsection{谓词演算初步}

\rTermWref{谓词演算}{PredicateCalculus}{predicate calculus}是一种形式系统,可描述为$\mathcal{L}(\mathrm{A},\Omega,\mathrm{Z},\mathrm{I})$,其中:
\begin{itemize}
    \renewcommand{\labelitemii}{}
    \item 集$\rHatNote{\mathrm{A}}{``alpha''}$是由代词$x$、$y$、$z$、$\cdots$组成的集;
    \item 集$\rHatNote{\Omega}{``omega''}$是由终结符组成的集,可将其划分为$\Omega_\text{量词}$、$\Omega_\text{名词}$、$\Omega_\text{映射}$、$\Omega_\text{谓词}$和$\Omega_\text{逻辑联结词}$:
          \begin{itemize}
              \item $\Omega_\text{量词} = \{ \rHatNote{\forall}{全称量词}, \rHatNote{\exists}{存在量词} \}$;
              \item $\Omega_\text{名词} = \{ a, b, c, \cdots \}$;
              \item $\Omega_\text{映射} = \{ f, g, h, \cdots \}$;
              \item $\Omega_\text{谓词} = \{ R, S, T, \cdots \}$;
              \item $\Omega_\text{逻辑联结词} = \{ \bot, \top, \lnot, \land, \lor, \to, \leftrightarrow, \cdots \}$;
          \end{itemize}
    \item 集$\rHatNote{\mathrm{Z}}{``zeta''}$是由推理规则组成的集:
          \begin{itemize}
              \item 1. 每个代词、名词都是项;
              \item 2. 若$f$是$n$元映射,且$t_1$、$\cdots$、$t_n$是项,则$f(t_1,\cdots,t_n)$也是项;
              \item 3. 除外都不是项;
              \item 4. 若$R$是$n$元谓词,且$t_1$、$\cdots$、$t_n$是项,则$R(t_1,\cdots,t_n)$是合式公式(下简称``公式'');
              \item 5. $\bot$和$\top$是公式;
              \item 6. 若$p$是公式,则$\lnot{}p$也是公式;
              \item 7. 若$p$和$q$是公式,则$p\land{}q$、$p\lor{}q$、$p\to{}q$、$p\leftrightarrow{}q$、$\cdots$也都是公式;
              \item 8. 若$x$是代词且$p$是公式,则$\forall{}x:p$、$\exists{}x:p$也都是公式;
              \item 9. 除外都不是公式.
           \end{itemize}
    \item 集$\rHatNote{\mathrm{I}}{``iota''}$是由\rTerm{逻辑公理}{logical axioms}组成的集,除继承命题演算公理外:
\end{itemize}

\newtheorem{LogicalAxioms}{LA}

\begin{LogicalAxioms}[\emph{等词公理}]\rMgnNote{axiom of equality}\label{LA:E}
    对于任一项$t$,公式
    \begin{equation*}
        t = t
    \end{equation*}
    \hfill 恒成立.
\end{LogicalAxioms}

\begin{LogicalAxioms}[\emph{全称例化公理模式}]\rMgnNote{axiom scheme for universal instantiation}\label{LA:UI}
    对于任一公式$\phi$、代词$x$和项$t$,公式
    \begin{equation*}
        (\forall x: \phi) \to \phi_{x\gets{}t}
    \end{equation*}
    \hfill 恒成立.
\end{LogicalAxioms}

\begin{LogicalAxioms}[\emph{存在泛化公理模式}]\rMgnNote{axiom scheme for existential generalization}\label{LA:EG}
    对于任一公式$\phi$、代词$x$和项$t$,公式
    \begin{equation*}
        \phi_{x\gets{}t} \to (\exists x: \phi)
    \end{equation*}
    \hfill 恒成立.
\end{LogicalAxioms}

\section{公理化集合论}

\rTermWref{策梅洛--弗兰克尔集论}{Zermelo-FraenkelSetTheory}{Zermelo-Fraenkel set theory}是一种谓词演算,
其项也被称作``\rTermWref{集合}{Set}{set}''(简称``集'')或``\rTermWref{元素}{Element}{element}''(简称``元''),
其谓词集$\Omega_\text{谓词}=\{\rHatNote{=}{等词},\rHatNote{\in}{``属于''}\}$,
其公理集$\mathrm{I}$有:

\newtheorem{ZermeloFraenkelAxioms}{ZF}      % 策梅洛–弗兰克尔公理
\setcounter{ZermeloFraenkelAxioms}{-1}
\newtheorem{ZermeloFraenkelTheorems}{定理}  % 策梅洛–弗兰克尔定理

\begin{ZermeloFraenkelAxioms}[\emph{集存在引理}]\rMgnNote{lemma of set exists}\label{ZF:SE}
    $ \exists S: S = S $.
\end{ZermeloFraenkelAxioms}

\hrule
\begin{proof}
    令假定集$ \Gamma = \{ \text{LA\ref{LA:E}}, \text{LA\ref{LA:EG}} \} $,有:
    \begin{enumerate}
        \item $ \phi_{x\gets{}t} \to (\exists x: \phi) $     \hfill LA\ref{LA:EG}
        \item $ (x = x)_{x\gets{}t} \to (\exists x: x = x) $ \hfill 令公式$\phi$为$ x = x $
        \item $ (t = t) \to (\exists x: x = x) $
        \item $ \top \to (\exists x: x = x) $                \hfill LA\ref{LA:E}
        \item $ \exists x: x = x $
        \item $ \exists S: S = S $                           \qedhere
    \end{enumerate}
\end{proof}
\hrule
\vspace{2ex}

集存在引理说明集合存在.

\begin{ZermeloFraenkelAxioms}[\emph{外延公理}]\rMgnNote{axiom of extensionality}\label{ZF:E}
    $ \forall A, \forall B: A = B \iff (\forall e: e \in A \leftrightarrow e \in B) $.
\end{ZermeloFraenkelAxioms}

外延公理说明一个集合完全由它所包含的元素确定:两个集合若拥有相同的元素,则这两个集合相等;反之亦然.

\begin{ZermeloFraenkelAxioms}[\emph{规范公理模式}]\rMgnNote{axiom schema of specification}\label{ZF:S}
    $ \forall U, \exists S, \forall e: e \in S \leftrightarrow e \in U \land P(e) $.
\end{ZermeloFraenkelAxioms}

规范公理模式说明给定一个集合,从该集合中选定符合条件的若干元素可以组成集合,即``\rTermWref{子集}{Subset}{subset}''.

由 ZF\ref{ZF:SE} 和 ZF\ref{ZF:S} 可以构造\rTermWref{空集}{EmptySet}{empty set}$ \emptyset = \{ e | e \in S \land (e \neq e) \} $——我们得到的第一个确定的集合.

\begin{ZermeloFraenkelAxioms}[\emph{配对公理}]\rMgnNote{axiom of pairing}\label{ZF:P}
    $ \forall a, \forall b, \exists S, \forall e: e \in S \leftrightarrow e = a \lor e = b $.
\end{ZermeloFraenkelAxioms}

配对公理说明给定两个元素$a$和$b$,则$ \{ a, b \} $是集合,称作``元素$a$和$b$的\rTermWref{对集}{Pair}{pair}''.
当给定的两个元素相等都为$e$时,$ \{ e, e \} $可以记作$ \{ e \} $,称作``元素$e$的\rTermWref{独集}{SingletonSet}{singleton}''.
形如$ \{ \{ a \}, \{ a, b \} \} $的对集可以记作$ (a, b) $称作``元素$a$和$b$的\rTermWref{有序对集}{OrderedPair}{ordered pair}''.

\begin{ZermeloFraenkelAxioms}[并集公理]\label{ZFA:4}
\end{ZermeloFraenkelAxioms}

\begin{ZermeloFraenkelAxioms}[幂集公理]\label{ZFA:5}
\end{ZermeloFraenkelAxioms}

\begin{ZermeloFraenkelAxioms}[无穷公理]\label{ZFA:6}
\end{ZermeloFraenkelAxioms}

\begin{ZermeloFraenkelAxioms}[替换公理]\label{ZFA:7}
\end{ZermeloFraenkelAxioms}

\subsection{正则公理、基础公理和选择公理}

\begin{ZermeloFraenkelAxioms}[正则公理]\label{ZFA:8}
    $ \forall S, \exists e: (\exists z: z \in S) \implies [e \in S \land \lnot(\exists y: y \in S \land y \in e)] $
\end{ZermeloFraenkelAxioms}
ZFA\ref{ZFA:8}说明所有非空集合中都存在这样的元素:该元素与该集合的交集为空集.


\subsection{集合概念}

一般的,\emph{集合}(\href{http://mathworld.wolfram.com/Set.html}{Set},简称\emph{集})是指具有某种特定性质的事物的总体,组成这个集合的事物称为该集合的\emph{元素}(\href{http://mathworld.wolfram.com/Element.html}{Element},简称\emph{元}).
通常用大写拉丁字母 $A$,$B$,$C$,$\cdots$ 表示集合,用小写拉丁字母 $a$,$b$,$c$,$\cdots$ 表示集合的元素.

如果元素 $e$ 是集合 $S$ 的元素,就说``元素 $e$ \emph{属于}集合 $S$''(记作 $e\in{}S$)或``集合 $S$ \emph{拥有}元素 $e$''(记作 $S\ni{}e$).
如果元素 $e$ 不是集合 $S$ 的元素,就说``元素 $e$ \emph{不属于}集合 $S$''(记作 $e\not\in{}S$)或``集合 $S$ \emph{不拥有}元素 $e$''(记作 $S\not\ni{}e$).
一个集合,若它只拥有限个元素,则称为\emph{有限集};不是有限集的集合称为\emph{无限集}.

通常使用\emph{列举法}、\emph{描述法}或\emph{文氏图}(\href{http://mathworld.wolfram.com/VennDiagram.html}{Venn Diagram})表示集合:
例如,由元素 $a_1$,$a_2$,$\cdots$,$a_n$ 组成的集合 $A$ 可表示为
\[
A = \{ a_1, a_2, \cdots, a_n \} \text{;}
\]
由具有某种性质 $P$ 的元素 $b$ 的全体组成的集合 $B$ 可表示为
\[
B = \{ b | b \text{具有性质} P \} \text{.}
\]

\subsection{集合间的关系}

如果集合 $S$ 的元素都是集合 $L$ 的元素,则称集合 $S$ 是集合 $L$ 的\emph{子集}(\href{http://mathworld.wolfram.com/Subset.html}{Subset}),记作 $S\subset{}L$(读作``集合 $S$ \emph{包含于}集合 $L$'');
或称集合 $L$ 是集合 $S$ 的\emph{超集}(\href{http://mathworld.wolfram.com/Superset.html}{Superset}),记作 $L\supset{}S$(读作``集合 $L$ \emph{包含}集合 $S$'').

如果集合 $A$ 的元素都是集合 $B$ 的元素,且集合 $B$ 的元素也都是集合 $A$ 的元素,即集合 $A$ 与集合 $B$ 互相包含,则称集合 $A$ 与集合 $B$ \emph{相等},记作 $A=B$(读作``集合 $A$ \emph{等于}集合 $B$'').

如果集合 $S$ 包含于集合 $L$,且它们不相等,则称集合 $S$ 是集合 $L$ 的\emph{真子集}(\href{http://mathworld.wolfram.com/ProperSubset.html}{Proper Subset}),记作 $S\subsetneqq{}L$(读作``集合 $S$ \emph{真包含于}集合 $L$'');
或称集合 $L$ 是集合 $S$ 的\emph{真超集}(\href{http://mathworld.wolfram.com/ProperSuperset.html}{Proper Superset}),记作 $L\supsetneqq{}S$(读作``集合 $L$ \emph{真包含}集合 $S$'').

不拥有任何元素的集合称为\emph{空集}(\href{http://mathworld.wolfram.com/EmptySet.html}{Empty Set}),记作 $\varnothing$,规定空集是任何集合的子集.
有时,在指定上下文中可以将所有研究对象组成一个集合 $U$,所研究的其它集合都是集合 $U$ 的子集,此时我们称集合 $U$ 为\emph{全集}(\href{http://mathworld.wolfram.com/UniversalSet.html}{Universe}).

\subsection{并集、交集、差集、对称差集、补集和余集}

由所有属于集合 $A$ 或者属于集合 $B$ 的元素组成的集合,称为集合 $A$ 与集合 $B$ 的\emph{并集}(\href{http://mathworld.wolfram.com/Union.html}{Union},简称\emph{并}),记作 $A\cup{}B$,该运算也称为\emph{并}(\href{http://mathworld.wolfram.com/Cup.html}{Cup}),即
\[
A \cup B = \{ e | e \in A \text{或} e \in B \} \text{;}
\]
由所有既属于集合 $A$ 又属于集合 $B$ 的元素组成的集合,称为集合 $A$ 与集合 $B$ 的\emph{交集}(\href{http://mathworld.wolfram.com/Intersection.html}{Intersection},简称\emph{交}),记作 $A\cap{}B$,该运算也称为\emph{交}(\href{http://mathworld.wolfram.com/Cap.html}{Cap}),即
\[
A \cap B = \{ e | e \in A \text{且} e \in B \} \text{;}
\]
由所有属于集合 $A$ 但不属于集合 $B$ 的元素组成的集合,称为集合 $A$ 与集合 $B$ 的\emph{差集}(\href{http://mathworld.wolfram.com/SetDifference.html}{Set Difference},简称\emph{差}),记作 $A\setminus{}B$,该运算称为\emph{减}(\href{http://mathworld.wolfram.com/SetMinus.html}{Set Minus}),即
\[
A \setminus B = \{ e | e \in A \text{且} e \not\in B \} \text{;}
\]
由所有属于集合 $A$ 或者属于集合 $B$ 但不同时属于两集合的元素组成的集合,称为集合 $A$ 与集合 $B$ 的\emph{对称差集}(\href{http://mathworld.wolfram.com/SymmetricDifference.html}{Symmetric Difference}),记作 $A\ominus{}B$,即
\[
A \ominus B = (A \cup B) \setminus (A \cap B) \text{;}
\]
设有超集 $L$ 包含子集 $S$,则称超集 $L$ 与子集 $S$ 的差集,为超集 $L$ 中子集 $S$ 的\emph{补集}(\href{http://mathworld.wolfram.com/ComplementSet.html}{Complement Set}),记作 $\complement_LS$,即
\[
\begin{aligned}
\text{存在超集$L$和其子集$S$,} \complement_LS & = L \setminus S \\
& = \{ e | e \in L \text{且} e \not\in S \} \text{;}
\end{aligned}
\]
特别的,若存在全集 $U$,则称全集 $U$ 中子集 $S$ 的补集,为集合 $S$ 的\emph{余集},记作 $S^c$,即
\[
\begin{aligned}
\text{存在全集$U$,} S^c &= \complement_US \\
&= U \setminus S \\
&= \{ e | e \in U \text{且} e \not\in S \} \text{.}
\end{aligned}
\]

\subsection{笛卡尔乘积}

由所有``属于集合 $A$ 的任一元素 $a$ 和属于集合 $B$ 的任一元素 $b$ 组成的有序对 $(a, b)$'' 组成的集合,称为集合 $A$ 与集合 $B$ 的\emph{笛卡尔乘积}(\href{http://mathworld.wolfram.com/CartesianProduct.html}{Cartesian Product}),记作 $A\times{}B$,即
\[
A \times B = \{ (a, b) | a \in A \text{且} b \in B \} \text{.}
\]

\newpage
\section{映射}

\begin{Definition}[\textbf{映射}]\label{Definition:Map}
    设 $X$ 和 $Y$ 是非空集合,对于集合 $X$ 中任一元素 $x$,依照某种对应规律 $f$,恒有集合 $Y$ 中唯一确定元素 $y$ 与之对应,则称对应规律 $f$ 为一个从集合 $X$ 到集合 $Y$ 的\rTermWref{映射}{Map}{map},记作
    \[
        \begin{aligned}
            f&: X \to Y \\
            \text{或} \quad f&: x \mapsto y \text{,}
        \end{aligned}
    \]
    其中集合 $X$ 称为映射 $f$ 的\rTermWref{定义域}{Domain}{domain},记作 $D_f$,
    集合 $Y$ 称为映射 $f$ 的\rTermWref{到达域}{Codomain}{codomain},记作 $C_f$,
    元素 $x$ 称为在映射 $f$ 下元素 $y$ 的一个\rTermWref{原像}{Preimage}{preimage},
    元素 $ y = f(x) $ 称为在映射 $f$ 下元素 $x$ 的\rTermWref{像}{Image}{image},
    集合 $ f(X) = \{ f(x) | x \in X \} $ 称为映射 $f$ 的\rTermWref{值域}{Range}{range},记作 $R_f$.
\end{Definition}

\begin{Definition}[\textbf{单射}]\label{Definition:Injection}
    设 $ f: D_f \to C_f $ 是映射,若
    \[ \rHatNote {\forall x_1, \forall x_2: x_1 \in D_f \land x_2 \in D_f \land x_1 \neq x_2 \implies f(x_1) \neq f(x_2)} {对任意原像 $x_1$ 和 $x_2$ 不等,像 $f(x_1)$ 和 $f(x_2)$ 恒不等} \text{,} \]
    则称映射 $f$ 是\rTermWref{单射}{Injection}{injection}.
\end{Definition}

\begin{Definition}[\textbf{满射}]\label{Definition:Surjection}
    设 $ f: D_f \to C_f $ 是映射,若
    \[ \rHatNote {\forall y, \exists x: y \in C_f \land x \in D_f \implies y = f(x)} {对于到达域中任一元素 y,都存在原像 $x$,使得 $ y = f(x) $} \text{,} \]
    则称映射 $f$ 是\rTermWref{满射}{Surjection}{surjection}.
\end{Definition}

\begin{Definition}[\textbf{双射}]\label{Definition:Bijection}
    设 $f$ 是映射,若映射 $f$ 既是单射,又是满射,则称映射 $f$ 是\rTermWref{双射}{Bijection}{bijection}.
\end{Definition}

\subsection{逆映射、复合映射和映射幂}

\begin{Definition}[\textbf{逆映射}]\label{Definition:InverseMap}
    设 $ f: X \to Y $ 是单射,则对于值域 $R_f$ 中任一像 $y$ 都存在唯一原像 $x$ 与之对应,于是可以定义一个从值域 $R_f$ 到定义域 $X$ 的新映射
    \[ f^{-1}: y \mapsto x \text{,} \]
    这个映射称为映射 $f$ 的\rTerm{逆映射}{inverse map},其定义域 $ D_{f^{-1}} = R_f $,值域 $ R_{f^{-1}} = X $.
\end{Definition}

\begin{Definition}[\textbf{复合映射}]\label{Definition:CompositeMap}
    设 $ g: X \to Y_S $ 和 $ f: Y_L \to Z $ 是映射,且 $ Y_S \subset Y_L $,则对于集合 $X$ 中任一元素 $x$ 都存在集合 $Z$ 中唯一确定元素 $ z = f(g(x)) $ 与之对应,于是可以定义一个从集合 $X$ 到集合 $Z$ 的新映射
    \[ f \circ g: x \mapsto f(g(x)) \text{,} \]
    这个映射称为映射 $g$ 和映射 $f$ 构成的\rTerm{复合映射}{composite map},其定义域 $ D_{f\circ{}g} = X $,到达域 $ C_{f\circ{}g} = Z $.
\end{Definition}

\subsection{二元运算}

\begin{Definition}[\textbf{二元运算}]\label{Definition:BinaryOperation}
    设 $S$ 是集合,$ S \times S $ 是集合 $S$ 的笛卡儿平方,则称从集合 $ S \times S $ 到集合 $S$ 的映射 $ f: S \times S \to S $ 为集合 $S$ 上的\rTermWref{二元运算}{BinaryOperation}{binary operation}.
    若元素 $a$、$b$ 是集合 $S$ 的元素,通常将二元运算的像 $ f(a, b) $ 记作 $ a f b $.
\end{Definition}

\begin{Definition}[\textbf{零元}]\label{Definition:ZeroElement}
    设 $S$ 是集合,$\circ$ 是集合 $S$ 上的二元运算,若
    \[ \exists 0_L, \forall e : 0_L \in S \land e \in S \implies 0_L \circ e = 0_L \]
    则称元素 $0_L$ 为二元运算 $\circ$ 的\rTerm{左零元}{left zero element}.
    若
    \[ \exists 0_R, \forall e : 0_R \in S \land e \in S \implies e \circ 0_R = 0_R \]
    则称元素 $0_R$ 为二元运算 $\circ$ 的\rTerm{右零元}{right zero element}.
    若
    \[ \exists 0, \forall e : 0 \in S \land e \in S \implies 0 \circ e = 0 \land e \circ 0 = 0 \]
    则称元素 $0$ 为二元运算 $\circ$ 的\rTermWref{零元}{ZeroElement}{zero element}.
\end{Definition}

\begin{Definition}[\textbf{幺元}]\label{Definition:IdentityElement}
    设 $S$ 是集合,$\circ$ 是集合 $S$ 上的二元运算,若
    \[ \exists 1_L, \forall e : 1_L \in S \land e \in S \implies 1_L \circ e = e \]
    则称元素 $1_L$ 是二元运算 $\circ$ 的\rTerm{左幺元}{left identity element}.
    若
    \[ \exists 1_R, \forall e : 1_R \in S \land e \in S \implies e \circ 1_R = 1_R \]
    则称元素 $1_R$ 是二元运算 $\circ$ 的\rTerm{右幺元}{right identity element}.
    若
    \[ \exists 1, \forall e : 1 \in S \land e \in S \implies 1 \circ e = e \land e \circ 1 = e \]
    则元素 $1$ 为二元运算 $\circ$ 的\rTermWref{幺元}{IdentityElement}{identity element}.
\end{Definition}

\begin{Definition}[\textbf{交换律}]\label{Definition:Commutative}
    设 $S$ 是集合,$\circ$ 是集合 $S$ 上的二元运算,若
    \[ \forall a, \forall b : a \in S \land b \in S \implies a \circ b = b \circ a \]
    则称二元运算 $\circ$ 满足\rTermWref{交换律}{Commutative}{commutative}.
\end{Definition}

\begin{Definition}[\textbf{幂等律}]\label{Definition:Idempotent}
    设 $S$ 是集合,$\circ$ 是集合 $S$ 上的二元运算,若
    \[ \forall e : e \in S \implies e \circ e = e \]
    则称二元运算 $\circ$ 满足\rTerm{幂等律}{idempotent}.
\end{Definition}

\begin{Definition}[\textbf{幂零律}]\label{Definition:Nilpotent}
    设 $S$ 是集合,$\circ$ 是集合 $S$ 上的二元运算,$0$ 是二元运算 $\circ$ 的零元,若
    \[ \forall e : e \in S \implies e \circ e = 0 \]
    则称二元运算 $\circ$ 满足\rTerm{幂零律}{nilpotent}.
\end{Definition}

\begin{Definition}[\textbf{幂幺律}]\label{Definition:Unipotent}
    设 $S$ 是集合,$\circ$ 是集合 $S$ 上的二元运算,$1$ 是二元运算 $\circ$ 的幺元,若
    \[ \forall e : e \in S \implies e \circ e = 1 \]
    则称二元运算 $\circ$ 满足\rTerm{幂幺律}{unipotent}.
\end{Definition}
