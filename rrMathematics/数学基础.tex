\chapter{数学基础}

本章主要参照\citeauthor{WangFt2001}先生的\citetitle{WangFt2001}\cite{WangFt2001}.

\section{历史概述}

人类数学大体上经历了三个大的发展阶段:以几何数学为主体的初等数学阶段,以分析数学为主体的古典数学阶段,以集论数学为主体的现代数学阶段.

\section{逻辑准备}

\subsection{形式语言初步}

\rTermWref{形式语言}{FormalLanguage}{formal language}可以由\rTermWref{形式文法}{Grammar}{formal grammar}描述,形式文法是一种四元组$G=(N,\Sigma,P,S)$,其中:
\begin{itemize}
    \item 集$N$是由\rTerm{非终结符}{nonterminal symbols}组成的有限集;
    \item 集$\Sigma$是由\rTerm{终结符}{terminal symbols}组成的有限集,且$N\cap\Sigma=\emptyset$;集$V=N\cup\Sigma$称作\rTerm{词汇表}{vocabulary};
    \item 集$P$是由\rTerm{产生式规则}{production rules}组成的有限集;
    \item 元$S$是\rTerm{开始符号}{start symbol},且$S\in{}N$.
\end{itemize}

譬如,一种可以描述二进制整数的形式文法和它的一例推导示例:
\begin{multicols}{2}
    \begin{equation*}
        G_\text{binary} = (N, \Sigma, P, S), \begin{cases}
            N      = \{ S_\text{(start)}, D_\text{(digital)} \} \\
            \Sigma = \{ -, 0, 1 \}                              \\
            P      = \begin{cases}
                         \text{1.}\ S \to D  \\
                         \text{2.}\ S \to -D \\
                         \text{3.}\ D \to 0D \\
                         \text{4.}\ D \to 1D \\
                         \text{5.}\ D \to 0  \\
                         \text{6.}\ D \to 1  \\
                     \end{cases}                                \\
            S                                                   \\
        \end{cases}
    \end{equation*}
    \newline
    \begin{equation*}
        \begin{aligned}
            S &\underset{2}{\Rightarrow} -D    \\
              &\underset{4}{\Rightarrow} -1D   \\
              &\underset{3}{\Rightarrow} -10D  \\
              &\underset{4}{\Rightarrow} -101D \\
              &\underset{5}{\Rightarrow} -1010 \\
        \end{aligned}
    \end{equation*}
\end{multicols}

\subsection{命题演算初步}

\rTermWref{命题演算}{PropositionalCalculus}{propositional calculus}是一种\rTerm{形式系统}{formal system},可描述为$\mathcal{L}(\mathrm{A},\Omega,\mathrm{Z},\mathrm{I})$,其中:
\begin{itemize}
    \renewcommand{\labelitemii}{}
    \item 集$\rHatNote{\mathrm{A}}{``alpha''}$是由\rTerm{命题变量}{propositional variables}$p$、$q$、$r$、$\cdots$组成的集;
    \item 集$\rHatNote{\Omega}{``omega''}$是由\rTerm{逻辑联结词}{logical connectives}组成的集,可按元数将其划分为$\Omega_0$、$\Omega_1$和$\Omega_2$:
          \begin{itemize}
              \item $\Omega_0 = \{ \rHatNote{\bot}{恒假}, \rHatNote{\top}{恒真} \}$;
              \item $\Omega_1 = \{ \rHatNote{\lnot}{否定词} \}$;
              \item $\Omega_2 = \{ \rHatNote{\land}{合取词}, \rHatNote{\lor}{析取词}, \rHatNote{\to}{蕴含词}, \rHatNote{\leftrightarrow}{等价词}, \cdots \}$;
          \end{itemize}
    \item 集$\rHatNote{\mathrm{Z}}{``zeta''}$是由\rTerm{推理规则}{inference rules}组成的集:
          \begin{itemize}
              \item 1. 每个命题变量都是合式公式(下简称``公式'');
              \item 2. $\bot$和$\top$是公式;
              \item 3. 若$p$是公式,则$\lnot{}p$也是公式;
              \item 4. 若$p$和$q$是公式,则$p\land{}q$、$p\lor{}q$、$p\to{}q$、$p\leftrightarrow{}q$、$\cdots$也都是公式;
              \item 5. 除外都不是公式.
          \end{itemize}
    \item 集$\rHatNote{\mathrm{I}}{``iota''}$是由\rTerm{公理}{axioms}组成的集,即下表所述运算:
\end{itemize}

\begin{table}[h!]
    \centering
    \newcommand{\F}{{\color{red}$\bot$}}
    \newcommand{\T}{{\color{blue}$\top$}}
    \newcommand{\FoF}{\F$\circ$\F}
    \newcommand{\FoT}{\F$\circ$\T}
    \newcommand{\ToF}{\T$\circ$\F}
    \newcommand{\ToT}{\T$\circ$\T}
    \begin{tabular}{c c c c c c c c l}
        \hline
        \multicolumn{4}{c}{等价的公式} & \multicolumn{4}{c}{真值表} & \multicolumn{1}{c}{助记} \\
        $p\circ{}q$      & $p\circ\lnot{}q$      & $\lnot{}p\circ{}q$      & $\lnot{}p\circ\lnot{}q$      & \FoF & \FoT & \ToF & \ToT &                    \\
        \hline
        \multicolumn{4}{c}{$\top$}                                                                        & \T   & \T   & \T   & \T   & ``恒真''           \\
        $p\uparrow{}q$   & $p\to\lnot{}q$        & $\lnot{}p\gets{}q$      & $\lnot{}p\lor\lnot{}q$       & \T   & \T   & \T   & \F   & ``与非'',存假为真 \\
        $p\to{}q$        & $p\uparrow\lnot{}q$   & $\lnot{}p\lor{}q$       & $\lnot{}p\gets\lnot{}q$      & \T   & \T   & \F   & \T   & ``蕴含''           \\
        \multicolumn{4}{c}{$\lnot{}p$}                                                                    & \T   & \T   & \F   & \F   & ``非''             \\
        $p\gets{}q$      & $p\lor\lnot{}q$       & $\lnot{}p\uparrow{}q$   & $\lnot{}p\to\lnot{}q$        & \T   & \F   & \T   & \T   & ``蕴含于''         \\
        \multicolumn{4}{c}{$\lnot{}q$}                                                                    & \T   & \F   & \T   & \F   & ``非''             \\
        $p\odot{}q$      & $p\oplus\lnot{}q$     & $\lnot{}p\oplus{}q$     & $\lnot{}p\odot\lnot{}q$      & \T   & \F   & \F   & \T   & ``同或'',相同为真 \\
        $p\downarrow{}q$ & $p\not\gets\lnot{}q$  & $\lnot{}p\not\to{}q$    & $\lnot{}p\land\lnot{}q$      & \T   & \F   & \F   & \F   & ``或非'',全假为真 \\
        $p\lor{}q$       & $p\gets\lnot{}q$      & $\lnot{}p\to{}q$        & $\lnot{}p\uparrow\lnot{}q$   & \F   & \T   & \T   & \T   & ``或'',存真为真   \\
        $p\oplus{}q$     & $p\odot\lnot{}q$      & $\lnot{}p\odot{}q$      & $\lnot{}p\oplus\lnot{}q$     & \F   & \T   & \T   & \F   & ``异或'',相异为真 \\
        \multicolumn{4}{c}{$q$}                                                                           & \F   & \T   & \F   & \T   & ``公式''           \\
        $p\not\gets{}q$  & $p\downarrow\lnot{}q$ & $\lnot{}p\land{}q$      & $\lnot{}p\not\to\lnot{}q$    & \F   & \T   & \F   & \F   & ``不蕴含于''       \\
        \multicolumn{4}{c}{$p$}                                                                           & \F   & \F   & \T   & \T   & ``公式''           \\
        $p\not\to{}q$    & $p\land\lnot{}q$      & $\lnot{}p\downarrow{}q$ & $\lnot{}p\not\gets\lnot{}q$  & \F   & \F   & \T   & \F   & ``不蕴含''         \\
        $p\land{}q$      & $p\not\to\lnot{}q$    & $\lnot{}p\not\gets{}q$  & $\lnot{}p\downarrow\lnot{}q$ & \F   & \F   & \F   & \T   & ``与'',全真为真   \\
        \multicolumn{4}{c}{$\bot$}                                                                        & \F   & \F   & \F   & \F   & ``恒假''           \\
        \hline
    \end{tabular}
\end{table}

\subsubsection{重言式和矛盾式}

若$p$、$q$、$r$是公式,可以证明下述等式恒成立:

\[ (p \to p)                                                               \equiv \top \tag{同一律} \]
\[ (\lnot{}p \land p)                                                      \equiv \bot \tag{矛盾律} \]
\[ (\lnot{}p \lor p)                                                       \equiv \top \tag{排中律} \]
\[ (\lnot\lnot{}p \leftrightarrow p)                                       \equiv \top \tag{双重否定律} \]

也可以证明下述合取运算规律恒成立:

\[ [(p \land q) \leftrightarrow (q \land p)]                               \equiv \top \tag{合取交换律} \]
\[ \{[(p \land q) \land r] \leftrightarrow [p \land (q \land r)]\}         \equiv \top \tag{合取结合律} \]
\[ \{[p \land (q \lor r)] \leftrightarrow [(p \land q) \lor (p \land r)]\} \equiv \top \tag{合取分配律} \]
\[ [\lnot(p \land q) \leftrightarrow (\lnot{}p \lor \lnot{}q)]             \equiv \top \tag{合取德·摩根律} \]

也可以证明下述析取运算规律恒成立:

\[ [(p \lor q) \leftrightarrow (q \lor p)]                                 \equiv \top \tag{析取交换律} \]
\[ \{[(p \lor q) \lor r] \leftrightarrow [p \lor (q \lor r)]\}             \equiv \top \tag{析取结合律} \]
\[ \{[p \lor (q \land r)] \leftrightarrow [(p \lor q) \land (p \lor r)]\}  \equiv \top \tag{析取分配律} \]
\[ [\lnot(p \lor q) \leftrightarrow (\lnot{}p \land \lnot{}q)]             \equiv \top \tag{析取德·摩根律} \]

也可以证明下述蕴含运算规律恒成立:

\[ [\lnot{}p \to (p \to q)]                                                \equiv \top \tag{否定前件律} \]
\[ [q \to (p \to q)]                                                       \equiv \top \tag{肯定后件律} \]
\[ \{[p \to (q \to r)] \to [(p \to q) \to (p \to r)]\}                     \equiv \top \tag{蕴含分配律} \]
\[ [(\lnot{}p \to \lnot{}q) \to (q \to p)]                                 \equiv \top \tag{蕴含换位律} \]
\[ [(\lnot{}p \to p) \to p]                                                \equiv \top \tag{否定肯定律} \]
\[ \{(p \to q) \to [(q \to r) \to (p \to r)]\}                             \equiv \top \tag{假设三段论} \]

\subsection{谓词演算初步}

\rTermWref{谓词演算}{PredicateCalculus}{predicate calculus}是一种形式系统,可描述为$\mathcal{L}(\mathrm{A},\Omega,\mathrm{Z},\mathrm{I})$,其中:
\begin{itemize}
    \renewcommand{\labelitemii}{}
    \item 集$\rHatNote{\mathrm{A}}{``alpha''}$是由代词$x$、$y$、$z$、$\cdots$组成的集;
    \item 集$\rHatNote{\Omega}{``omega''}$是由终结符组成的集,可将其划分为$\Omega_\text{量词}$、$\Omega_\text{名词}$、$\Omega_\text{映射}$、$\Omega_\text{谓词}$和$\Omega_\text{逻辑联结词}$:
          \begin{itemize}
              \item $\Omega_\text{量词} = \{ \rHatNote{\forall}{全称量词}, \rHatNote{\exists}{存在量词} \}$;
              \item $\Omega_\text{名词} = \{ a, b, c, \cdots \}$;
              \item $\Omega_\text{映射} = \{ f, g, h, \cdots \}$;
              \item $\Omega_\text{谓词} = \{ R, S, T, \cdots \}$;
              \item $\Omega_\text{逻辑联结词} = \{ \bot, \top, \lnot, \land, \lor, \to, \leftrightarrow, \cdots \}$;
          \end{itemize}
    \item 集$\rHatNote{\mathrm{Z}}{``zeta''}$是由推理规则组成的集:
          \begin{itemize}
              \item 1. 每个代词、名词都是项;
              \item 2. 若$f$是$n$元映射,且$t_1$、$\cdots$、$t_n$是项,则$f(t_1,\cdots,t_n)$也是项;
              \item 3. 除外都不是项;
              \item 4. 若$R$是$n$元谓词,且$t_1$、$\cdots$、$t_n$是项,则$R(t_1,\cdots,t_n)$是合式公式(下简称``公式'');
              \item 5. $\bot$和$\top$是公式;
              \item 6. 若$p$是公式,则$\lnot{}p$也是公式;
              \item 7. 若$p$和$q$是公式,则$p\land{}q$、$p\lor{}q$、$p\to{}q$、$p\leftrightarrow{}q$、$\cdots$也都是公式;
              \item 8. 若$x$是代词且$p$是公式,则$\forall{}x:p$、$\exists{}x:p$也都是公式;
              \item 9. 除外都不是公式.
           \end{itemize}
    \item 集$\rHatNote{\mathrm{I}}{``iota''}$是由\rTerm{逻辑公理}{logical axioms}组成的集,除继承命题演算公理外:
\end{itemize}

\newtheorem{LogicalAxioms}{LA}

\begin{LogicalAxioms}[等词公理]\label{LA:1}
    对于任一项$t$,公式$ t = t $普遍成立.
\end{LogicalAxioms}

\begin{LogicalAxioms}[全称量词例化公理]\label{LA:2}
    对于任一公式$\phi$、代词$x$和项$t$,公式$ (\forall x: \phi) \to \phi^x_t $普遍成立.
\end{LogicalAxioms}

\begin{LogicalAxioms}[存在量词推广公理]\label{LA:3}
    对于任一公式$\phi$、代词$x$和项$t$,公式$ \phi^x_t \to (\exists x: \phi) $普遍成立.
\end{LogicalAxioms}

上述$\phi^x_t$表示以项$t$替换公式$\phi$中代词$x$后得到的公式.

\section{集论概念}

\rTermWref{策梅洛--弗兰克尔集论}{Zermelo-FraenkelSetTheory}{Zermelo-Fraenkel set theory}是一种谓词演算,其代词也被称作``集合''(简称``集'')或``元素''(简称``元''),其谓词集$\Omega_\text{谓词}=\{\rHatNote{=}{等词},\rHatNote{\in}{``属于''}\}$,其公理集$\mathrm{I}$可划分为三组公理:

% http://www.sohu.com/a/43289279_107944
\newtheorem{ZermeloFraenkelAxioms}{ZFA}
\newtheorem{ZermeloFraenkelTheorems}{ZFT}

\subsection{外延公理}

\begin{ZermeloFraenkelAxioms}[\emph{外延公理}]\rMarginNote{axiom of extensionality}\label{ZFA:1}
    $ \forall A, \forall B: A = B \iff (\forall e: e \in A \leftrightarrow e \in B) $
\end{ZermeloFraenkelAxioms}
ZFA\ref{ZFA:1}说明一个集合完全由它所包含的元素(即``外延'')确定:若两个集合拥有相同的元素,则这两个集合相等;反之亦然.

\subsection{子集公理、无序对公理、并集公理、幂集公理、无穷公理和替换公理}

\begin{ZermeloFraenkelAxioms}[子集公理]\label{ZFA:2}
    $ \forall U, \exists S, \forall e: e \in S \iff e \in U \land P(e) $
\end{ZermeloFraenkelAxioms}
ZFA\ref{ZFA:2}说明给定一个集合,从该集合中选定符合条件的若干元素可以组成集合,即``子集是存在的''.

\begin{proof}
    设假定集$
        \Gamma = \begin{cases}
            t = t                                                            &\text{(LA\ref{LA:1})}   \\
            \phi^x_t \to (\exists x: \phi)                                   &\text{(LA\ref{LA:3})}   \\
            \forall U, \exists S, \forall e: e \in S \iff e \in U \land P(e) &\text{(ZFA\ref{ZFA:2})} \\
        \end{cases}
    $,有:
    
    \begin{align*}
        t = t &\text{(LA\ref{LA:1})}
    \end{align*}
    由逻辑公理L\ref{LA:1}和L\ref{LA:3}可得$ \exists S: S = S $,即``集合是存在的''.
    结合ZFA\ref{ZFA:2},令该存在的集合为$U$,
\end{proof}

\begin{ZermeloFraenkelAxioms}[无序对公理]\label{ZFA:3}
\end{ZermeloFraenkelAxioms}

\begin{ZermeloFraenkelAxioms}[并集公理]\label{ZFA:4}
\end{ZermeloFraenkelAxioms}

\begin{ZermeloFraenkelAxioms}[幂集公理]\label{ZFA:5}
\end{ZermeloFraenkelAxioms}

\begin{ZermeloFraenkelAxioms}[无穷公理]\label{ZFA:6}
\end{ZermeloFraenkelAxioms}

\begin{ZermeloFraenkelAxioms}[替换公理]\label{ZFA:7}
\end{ZermeloFraenkelAxioms}

\subsection{正则公理、基础公理和选择公理}

\begin{ZermeloFraenkelAxioms}[正则公理]\label{ZFA:8}
    $ \forall S, \exists e: (\exists z: z \in S) \implies [e \in S \land \lnot(\exists y: y \in S \land y \in e)] $
\end{ZermeloFraenkelAxioms}
ZFA\ref{ZFA:8}说明所有非空集合中都存在这样的元素:该元素与该集合的交集为空集.
