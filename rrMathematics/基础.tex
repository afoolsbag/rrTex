\chapter{基础}

本章主要参照\citeauthor{WangFt2001}先生的\citetitle{WangFt2001}\footfullcite{WangFt2001}\cite{WangFt2001}.

\section{历史概述}

人类数学大体上经历了三个大的发展阶段:以几何数学为主体的初等数学阶段,以分析数学为主体的古典数学阶段,以集论数学为主体的现代数学阶段.

\section{逻辑准备}


\rTermWref{形式语言}{FormalLanguage}{formal language}可以由\rTermWref{形式文法}{Grammar}{formal grammar}描述,形式文法是一种四元组$\mathrm{G}=(\mathrm{N},\Sigma,\mathrm{P},\mathrm{S})$,其中:
\begin{itemize}
	\item \rTerm{非终结符}{nonterminal symbols}有限集$\mathrm{N}$;
	\item \rTerm{终结符}{terminal symbols}有限集$\Sigma$,且$\mathrm{N}\cap\Sigma=\emptyset$;集$\mathrm{V}=\mathrm{N}\cup\Sigma$称为\rTerm{词汇表}{vocabulary};
	\item \rTerm{产生式规则}{production rules}有限集$\mathrm{P}$;
	\item \rTerm{开始符号}{start symbol}$\mathrm{S}$,且$\mathrm{S}\in\mathrm{N}$.
\end{itemize}

\subsection{命题演算初步}

\rTermWref{命题演算}{PropositionalCalculus}{propositional calculus}是一种\rTerm{形式系统}{formal system},其形式语言可描述为$\mathcal{L}(\mathrm{A},\Omega,\mathrm{Z},\mathrm{I})$,其中:
\begin{itemize}
	\item \rTerm{命题变元}{propositional variables}集$\rHatNote{\mathrm{A}}{``alpha''}=\{p,q,r,\cdots\}$;
	\item \rTerm{逻辑连接词}{logical connectives}集
	      $\rHatNote{\Omega}{``omega''} = \{ \rHatNote{\top}{真}, \rHatNote{\bot}{假} \} \cup \{ \rHatNote{\lnot}{否定词} \} \cup \{ \rHatNote{\land}{合取词}, \rHatNote{\lor}{析取词}, \rHatNote{\rightarrow}{蕴含词}, \rHatNote{\leftrightarrow}{等价词}, \cdots \} \cup \cdots$;
	\item \rTerm{推理规则}{inference rules}集
	      $\rHatNote{\mathrm{Z}}{``zeta''} = \begin{cases}
	           \text{每个命题变元都是} \mathcal{L} \text{的公式(下简称``公式'');} \\
	           \text{若} p \text{和} q \text{是公式,则} \lnot{}p \text{、} p\land{}q \text{、} p\lor{}q \text{、} p\rightarrow{}q \text{、} p\leftrightarrow{}q \text{、} \cdots \text{也都是公式;} \\
	           \text{除外都不是公式.} \\
	       \end{cases}$
	\item \rTerm{公理}{axioms}集$\rHatNote{\mathrm{I}}{``iota''}$.
\end{itemize}

\newpage
\subsubsection{逻辑连接词}

\begin{table}[h!]
	\centering
	\newcommand{\T}{{\color{blue}$\top$}}
	\newcommand{\F}{{\color{red}$\bot$}}
	\newcommand{\FoF}{\F$\circ$\F}
	\newcommand{\FoT}{\F$\circ$\T}
	\newcommand{\ToF}{\T$\circ$\F}
	\newcommand{\ToT}{\T$\circ$\T}
	\begin{tabular}{c c c c c c c c l}
		\multicolumn{4}{c}{等价公式} & \multicolumn{4}{c}{真值表} & 助记 \\
		\hline
		\multicolumn{4}{c}{$p\circ{}q$}                                                                                                                             & \FoF & \FoT & \ToF & \ToT & \\
		\hline
		\multicolumn{4}{c}{$\rHatNote{\top}{``真''}$}                                                                                                               & \T   & \T   & \T   & \T   & 恒真 \\
		$\rHatNote{p\uparrow{}q}{``公式$p$与非公式$q$''}$           & $p\rightarrow\lnot{}q$     & $\lnot{}p\leftarrow{}q$      & $\lnot{}p\lor\lnot{}q$            & \T   & \T   & \T   & \F   & 存假为真 \\
		$\rHatNote{p\rightarrow{}q}{``公式$p$蕴含公式$q$''}$        & $p\uparrow\lnot{}q$        & $\lnot{}p\lor{}q$            & $\lnot{}p\leftarrow\lnot{}q$      & \T   & \T   & \F   & \T   & \\
		\multicolumn{4}{c}{$\rHatNote{\lnot{}p}{``非公式$p$''}$}                                                                                                    & \T   & \T   & \F   & \F   & \\
		$\rHatNote{p\leftarrow{}q}{``公式$p$蕴含于公式$q$''}$       & $p\lor\lnot{}q$            & $\lnot{}p\uparrow{}q$        & $\lnot{}p\rightarrow\lnot{}q$     & \T   & \F   & \T   & \T   & \\
		\multicolumn{4}{c}{$\rHatNote{\lnot{}q}{``非公式$q$''}$}                                                                                                    & \T   & \F   & \T   & \F   & \\
		$\rHatNote{p\leftrightarrow{}q}{``公式$p$等价于公式$q$''}$  & $p\oplus\lnot{}q$          & $\lnot{}p\oplus{}q$          & $\lnot{}p\leftrightarrow\lnot{}q$ & \T   & \F   & \F   & \T   & 相同为真 \\
		$\rHatNote{p\downarrow{}q}{``公式$p$或非公式$q$''}$         & $p\not\leftarrow\lnot{}q$  & $\lnot{}p\not\rightarrow{}q$ & $\lnot{}p\land\lnot{}q$           & \T   & \F   & \F   & \F   & 全假为真 \\
		$\rHatNote{p\lor{}q}{``公式$p$或公式$q$''}$                 & $p\leftarrow\lnot{}q$      & $\lnot{}p\rightarrow{}q$     & $\lnot{}p\uparrow\lnot{}q$        & \F   & \T   & \T   & \T   & 存真为真 \\
		$\rHatNote{p\oplus{}q}{``公式$p$异或公式$q$''}$             & $p\leftrightarrow\lnot{}q$ & $\lnot{}p\leftrightarrow{}q$ & $\lnot{}p\oplus\lnot{}q$          & \F   & \T   & \T   & \F   & 相异为真 \\
		\multicolumn{4}{c}{$\rHatNote{q}{``公式$q$''}$}                                                                                                             & \F   & \T   & \F   & \T   & \\
		$\rHatNote{p\not\leftarrow{}q}{``公式$p$非蕴含于公式$q$''}$ & $p\downarrow\lnot{}q$      & $\lnot{}p\land{}q$           & $\lnot{}p\not\rightarrow\lnot{}q$ & \F   & \T   & \F   & \F   & \\
		\multicolumn{4}{c}{$\rHatNote{p}{``公式$p$''}$}                                                                                                             & \F   & \F   & \T   & \T   & \\
		$\rHatNote{p\not\rightarrow{}q}{``公式$p$非蕴含公式$q$''}$  & $p\land\lnot{}q$           & $\lnot{}p\downarrow{}q$      & $\lnot{}p\not\leftarrow\lnot{}q$  & \F   & \F   & \T   & \F   & \\
		$\rHatNote{p\land{}q}{``公式$p$与公式$q$''}$                & $p\not\rightarrow\lnot{}q$ & $\lnot{}p\not\leftarrow{}q$  & $\lnot{}p\downarrow\lnot{}q$      & \F   & \F   & \F   & \T   & 全真为真 \\
		\multicolumn{4}{c}{$\rHatNote{\bot}{``假''}$}                                                                                                               & \F   & \F   & \F   & \F   & 恒假 \\
	\end{tabular}
\end{table}

\subsubsection{永真式}

\[ p \rightarrow p                                                                                 \text{,} \tag{同一律} \]
\[ \lnot{}p \lor p                                                                                 \text{,} \tag{排中律} \]
\[ \lnot(\lnot{}p \land p)                                                                         \text{,} \tag{矛盾律} \]
\[ \lnot\lnot{}p \leftrightarrow p                                                                 \text{,} \tag{双重否定律} \]
\[ (p \land q) \leftrightarrow (q \land p)                                                         \text{,} \tag{合取交换律} \]
\[ ((p \land q) \land r) \leftrightarrow (p \land (q \land r))                                     \text{,} \tag{合取结合律} \]
\[ (p \land (q \lor r)) \leftrightarrow ((p \land q) \lor (p \land r))                             \text{,} \tag{分配律} \]
\[ \lnot(p \land q) \leftrightarrow (\lnot{}p \lor \lnot{}q)                                       \text{,} \tag{德·摩根律} \]
\[ (p \lor q) \leftrightarrow (q \lor p)                                                           \text{,} \tag{析取交换律} \]
\[ ((p \lor q) \lor r) \leftrightarrow (p \lor (q \lor r))                                         \text{,} \tag{析取结合律} \]
\[ (p \lor (q \land r)) \leftrightarrow ((p \lor q) \land (p \lor r))                              \text{,} \tag{分配律} \]
\[ \lnot(p \lor q) \leftrightarrow (\lnot{}p \land \lnot{}q)                                       \text{,} \tag{德·摩根律} \]
\[ \lnot{}p \rightarrow (p \rightarrow q)                                                          \text{,} \tag{否定前件律} \]
\[ q \rightarrow (p \rightarrow q)                                                                 \text{,} \tag{肯定后件律} \]
\[ (p \rightarrow (q \rightarrow r)) \rightarrow ((p \rightarrow q) \rightarrow (p \rightarrow r)) \text{,} \tag{蕴含词分配律} \]
\[ (\lnot{}p \rightarrow \lnot{}q) \rightarrow (q \rightarrow p)                                   \text{,} \tag{换位律} \]
\[ (\lnot{}p \rightarrow p) \rightarrow p                                                          \text{,} \tag{否定肯定律} \]
\[ (p \rightarrow q) \rightarrow ((q \rightarrow r) \rightarrow (p \rightarrow r))                 \text{.} \tag{假设三段论} \]

\subsection{谓词演算简介}

谓词演算是一种形式系统,其形式语言可描述为$\mathcal{L}(\mathrm{A},\Omega,\mathrm{Z},\mathrm{I})$,其中:
\begin{itemize}
	\item \rTerm{命题变元}{propositional variables}集$\rHatNote{\mathrm{A}}{``alpha''}=\{p,q,r,\cdots\}$;
	\item \rTerm{逻辑连接词}{logical connectives}集
	$\rHatNote{\Omega}{``omega''} = \{ \rHatNote{\top}{真}, \rHatNote{\bot}{假} \} \cup \{ \rHatNote{\lnot}{否定词} \} \cup \{ \rHatNote{\land}{合取词}, \rHatNote{\lor}{析取词}, \rHatNote{\rightarrow}{蕴含词}, \rHatNote{\leftrightarrow}{等价词}, \cdots \} \cup \cdots$;
	\item \rTerm{推理规则}{inference rules}集
	$\rHatNote{\mathrm{Z}}{``zeta''} = \begin{cases}
	\text{每个命题变元都是} \mathcal{L} \text{的公式(下简称``公式'');} \\
	\text{若} p \text{和} q \text{是公式,则} \lnot{}p \text{、} p\land{}q \text{、} p\lor{}q \text{、} p\rightarrow{}q \text{、} p\leftrightarrow{}q \text{、} \cdots \text{也都是公式;} \\
	\text{除外都不是公式.} \\
	\end{cases}$
	\item \rTerm{公理}{axioms}集$\rHatNote{\mathrm{I}}{``iota''}$.
\end{itemize}

