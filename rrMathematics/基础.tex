\chapter{基础}

本章主要参照\citeauthor{WangFt2001}先生的\citetitle{WangFt2001}\footfullcite{WangFt2001}\cite{WangFt2001}.

\section{历史概述}

人类数学大体上经历了三个大的发展阶段:以几何数学为主体的初等数学阶段,以分析数学为主体的古典数学阶段,以集论数学为主体的现代数学阶段.

\section{逻辑准备}


\rTermWref{形式语言}{FormalLanguage}{formal language}可以由\rTermWref{形式文法}{Grammar}{formal grammar}描述,形式文法是一个四元组$\mathrm{G}=(\mathrm{N},\Sigma,\mathrm{P},\mathrm{S})$,其中:
\begin{itemize}
	\item \rTerm{非终结符}{nonterminal symbols}有限集$\mathrm{N}$;
	\item \rTerm{终结符}{terminal symbols}有限集$\Sigma$,且$\mathrm{N}\cap\Sigma=\emptyset$;集$\mathrm{V}=\mathrm{N}\cup\Sigma$称为\rTerm{词汇表}{vocabulary};
	\item \rTerm{产生式规则}{production rules}有限集$\mathrm{P}$;
	\item \rTerm{开始符号}{start symbol}$\mathrm{S}$,且$\mathrm{S}\in\mathrm{N}$.
\end{itemize}

\subsection{命题演算初步}

\rTermWref{命题演算}{PropositionalCalculus}{propositional calculus}是一个\rTerm{形式系统}{formal system}$\mathcal{L}(\mathrm{A},\Omega,\mathrm{Z},\mathrm{I})$,其中:
\begin{itemize}
	\item \rTerm{命题变元}{propositional variables}集$\hatn{\mathrm{A}}{``alpha''}=\{p,q,r,\cdots\}$;
	\item \rTerm{逻辑连接词}{logical connectives}集
	      $\hatn{\Omega}{``omega''} = \{ \hatn{\top}{真}, \hatn{\bot}{假} \} \cup \{ \hatn{\lnot}{否定词} \} \cup \{ \hatn{\land}{合取词}, \hatn{\lor}{析取词}, \hatn{\rightarrow}{蕴含词}, \hatn{\leftrightarrow}{等价词} \} \cup \cdots$;
	\item \rTerm{推理规则}{inference rules}集
	      $\hatn{\mathrm{Z}}{``zeta''} = \begin{cases}
	           \text{每个命题变元都是} \mathcal{L} \text{的公式(下简称``公式'');} \\
	           \text{若} p \text{和} q \text{是公式,则} \lnot{}p \text{、} p\land{}q \text{、} p\lor{}q \text{、} p\rightarrow{}q \text{、} p\leftrightarrow{}q \text{也都是公式;} \\
	           \text{除外都不是公式.} \\
	       \end{cases}$
	\item \rTerm{公理}{axioms}集$\hatn{\mathrm{I}}{``iota''}$.
\end{itemize}

\subsubsection{逻辑连接词}

\begin{table}[h!]
	\centering
	\newcommand{\T}{\color{blue}$\top$}
	\newcommand{\F}{\color{red}$\bot$}
	\begin{tabular}{c c c c c c c c l}
		\multicolumn{4}{c}{等价公式} & \multicolumn{4}{c}{真值表} & 读法 \\
		\hline
		\multicolumn{4}{c}{$p$\qquad}                                                                      & \F & \F & \T & \T & \\
		\multicolumn{4}{c}{\qquad$q$}                                                                      & \F & \T & \F & \T & \\
		\hline
		\multicolumn{4}{c}{$\top$}                                                                         & \T & \T & \T & \T & ``永真'' \\
		$p\uparrow{}q$        & $\lnot{}p\leftarrow{}q$ &  $p\rightarrow\lnot{}q$ & $\lnot{}p\lor\lnot{}q$ & \T & \T & \T & \F & ``与非'' \\
		$p\rightarrow{}q$     &&&& \T & \T & \F & \T & ``公式$p$蕴含公式$q$'' \\
		\multicolumn{4}{c}{$\lnot{}p$}                                                                     & \T & \T & \F & \F & ``非公式$p$'' \\
		$p\leftarrow{}q$      &&&& \T & \F & \T & \T & ``反蕴含'' \\
		\multicolumn{4}{c}{$\lnot{}q$}                                                                     & \T & \F & \T & \F & ``非公式$q$'' \\
		$p\leftrightarrow{}q$ &&&& \T & \F & \F & \T & ``公式$p$等价于公式$q$'' \\
		$p\downarrow{}q$      &&&& \T & \F & \F & \F & ``或非'' \\
		$p\lor{}q$            &&&& \F & \T & \T & \T & ``析取'' \\
		$p\oplus{}q$          &&&& \F & \T & \T & \F & ``公式$p$异或公式$q$'' \\
		\multicolumn{4}{c}{$q$}                                                                            & \F & \T & \F & \T & ``公式$q$'' \\
		$p\nleftarrow{}q$     &&&& \F & \T & \F & \F & ``反非蕴含'' \\
		\multicolumn{4}{c}{$p$}                                                                            & \F & \F & \T & \T & ``公式$p$'' \\
		$p\nrightarrow{}q$    &&&& \F & \F & \T & \F & ``非蕴含'' \\
		$p\land{}q$           &&&& \F & \F & \F & \T & ``公式$p$与公式$q$'' \\
		\multicolumn{4}{c}{$\bot$}                                                                         & \F & \F & \F & \F & ``永假'' \\
	\end{tabular}
\end{table}

\begin{itemize}
	\item 否定式 $\hatn{\lnot{}p}{``非公式$p$''}$:当公式$p$为$\top$时否定式为$\bot$,当公式$p$为$\bot$时否定式为$\top$;
	\item 合取式 $\hatn{p\land{}q}{``公式$p$与公式$q$''}$:当公式$p$和$q$全部为$\top$时合取式为$\top$,否则合取式为$\bot$;
	\item 析取式 $\hatn{p\lor{}q}{``公式$p$或公式$q$''}$:当公式$p$和$q$存在为$\top$时析取式为$\top$,否则析取式为$\bot$;
	\item 蕴含式 $\hatn{p\rightarrow{}q}{``前件$p$蕴含后件$q$''}$:当前件$p$为$\top$而后件$q$为$\bot$时蕴含式为$\bot$,否则蕴含式为$\top$;
	\item 等价式 $\hatn{p\leftrightarrow{}q}{``公式$p$等价于公式$q$''}$:当公式$p$和$q$真值相同时等价式为$\top$,否则等价式为$\bot$.
\end{itemize}

\subsubsection{永真式}

\[ p \rightarrow p                                                                                 \text{,} \tag{同一律} \]
\[ \lnot{}p \lor p                                                                                 \text{,} \tag{排中律} \]
\[ \lnot(\lnot{}p \land p)                                                                         \text{,} \tag{矛盾律} \]
\[ \lnot\lnot{}p \leftrightarrow p                                                                 \text{,} \tag{双重否定律} \]
\[ (p \land q) \leftrightarrow (q \land p)                                                         \text{,} \tag{合取交换律} \]
\[ ((p \land q) \land r) \leftrightarrow (p \land (q \land r))                                     \text{,} \tag{合取结合律} \]
\[ (p \land (q \lor r)) \leftrightarrow ((p \land q) \lor (p \land r))                             \text{,} \tag{分配律} \]
\[ \lnot(p \land q) \leftrightarrow (\lnot{}p \lor \lnot{}q)                                       \text{,} \tag{德·摩根律} \]
\[ (p \lor q) \leftrightarrow (q \lor p)                                                           \text{,} \tag{析取交换律} \]
\[ ((p \lor q) \lor r) \leftrightarrow (p \lor (q \lor r))                                         \text{,} \tag{析取结合律} \]
\[ (p \lor (q \land r)) \leftrightarrow ((p \lor q) \land (p \lor r))                              \text{,} \tag{分配律} \]
\[ \lnot(p \lor q) \leftrightarrow (\lnot{}p \land \lnot{}q)                                       \text{,} \tag{德·摩根律} \]
\[ \lnot{}p \rightarrow (p \rightarrow q)                                                          \text{,} \tag{否定前件律} \]
\[ q \rightarrow (p \rightarrow q)                                                                 \text{,} \tag{肯定后件律} \]
\[ (p \rightarrow (q \rightarrow r)) \rightarrow ((p \rightarrow q) \rightarrow (p \rightarrow r)) \text{,} \tag{蕴含词分配律} \]
\[ (\lnot{}p \rightarrow \lnot{}q) \rightarrow (q \rightarrow p)                                   \text{,} \tag{换位律} \]
\[ (\lnot{}p \rightarrow p) \rightarrow p                                                          \text{,} \tag{否定肯定律} \]
\[ (p \rightarrow q) \rightarrow ((q \rightarrow r) \rightarrow (p \rightarrow r))                 \text{.} \tag{假设三段论} \]

\subsection{谓词演算简介}

