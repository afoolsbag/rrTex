\chapter{基础}

本章主要参照\citeauthor{WangFt2001}先生的\citetitle{WangFt2001}\cite{WangFt2001}.

\section{历史概述}

人类数学大体上经历了三个大的发展阶段:以几何数学为主体的初等数学阶段,以分析数学为主体的古典数学阶段,以集论数学为主体的现代数学阶段.

\section{逻辑准备}

\subsection{命题演算初步}

命题演算可看成一种形式语言 $\mathcal{L}=\mathcal{L}(\mathrm{A}, \Omega, \mathrm{Z}, \mathrm{I})$,其中:
\begin{itemize}
	\item 集 $\mathrm{A}$ 是由命题变元所组成的可数无限集,命题变元通常用 $p$、$q$、$r$、$\cdots$ 表示;
	\item 集 $\Omega$ 是由命题连接词所组成的集,
	\begin{itemize}
		\item $\Omega_0 = \{ \hatn{\top}{真}, \hatn{\bot}{假} \}$,
		\item $\Omega_1 = \{ \hatn{\lnot}{否定词} \}$,
		\item $\Omega_2 = \{ \hatn{\land}{合取词}, \hatn{\lor}{析取词}, \hatn{\implies}{蕴含词}, \hatn{\iff}{等价词} \}$;
	\end{itemize}
	\item 集 $\mathrm{Z}$ 是规则所组成的集,
	\begin{enumerate}
		\item 每个命题变元都是一条公式(叫做简单公式).
		\item 若 $p$ 和 $q$ 是公式,则 $\lnot{}p$、$p\land{}q$、$p\lor{}q$、$p\implies{}q$、$p\iff{}q$都是公式.
	\end{enumerate}
	\item 集 $\mathrm{I}$ 是公设所组成的集.
\end{itemize}

\newpage
\subsubsection{命题连接词}

\begin{itemize}
	\item $\hatn{\lnot{}p}{``非$p$''}$ 表示``公式 $p$ 的否定''
	\item $\hatn{p\land{}q}{``$p$与$q$''}$ 表示``公式 $p$ 和公式 $q$ 的同取''
	\item $\hatn{p\lor{}q}{``$p$或$q$''}$ 表示``公式 $p$ 和公式 $q$ 的析取'',连接词 $\lor$ 表示``兼或'',另有连接词 $\lxor$ 表示``异或''
	\item 蕴含式 $\hatn{p\implies{}q}{``前件$p$蕴含后件$q$''}$ 表示``如果前件 $p$ 为真,那么后件 $q$ 为真''
	\item $\hatn{p\iff{}q}{``$p$等价于$q$''}$ 表示``公式 $p$ 为真当且仅当公式 $q$ 也为真''
\end{itemize}


\begin{table}[h!]
	\centering
	\begin{minipage}[t]{0.30\textwidth}
		\centering
		\begin{tabular}{c | c}
			$p$    & $\lnot{}p$ \\
			\hline
			$\top$ & $\bot$     \\
			$\bot$ & $\top$     \\
		\end{tabular}
	\end{minipage}
	\quad
	\begin{minipage}[t]{0.62\textwidth}
		\begin{tabular}{c c | c c c c c}
			$p$    & $q$    & $p\land{}q$ & $p\lor{}q$ & $p\lxor{}q$ & $p\implies{}q$ & $p\iff{}q$ \\
			\hline
			$\top$ & $\top$ & $\top$      & $\top$     & $\bot$      & $\top$         & $\top$     \\
			$\top$ & $\bot$ & $\bot$      & $\top$     & $\top$      & $\bot$         & $\bot$     \\
			$\bot$ & $\top$ & $\bot$      & $\top$     & $\top$      & $\top$         & $\bot$     \\
			$\bot$ & $\bot$ & $\bot$      & $\bot$     & $\bot$      & $\top$         & $\top$     \\
		\end{tabular}
	\end{minipage}
\end{table}


\subsection{欧几里得(Euclid)几何}

\begin{description}
	\item[公理1] 等于同量的量相等.
	\item[公理2] 等量加等量,其和相等.
	\item[公理3] 等量减等量,其差相等.
	\item[公理4] 互相重合的一定相等.
	\item[公理5] 整体大于部分.
	\item[公设1] 从每一点到另一点可引直线.
	\item[公设2] 每条直线都可以无限延长.
	\item[公设3] 以任意点为中心可作半径等于任意长的圆.
	\item[公设4] 凡直角都相等.
	\item[公设5] (平行公设)同平面两直线与第三直线相交,若其中一侧的两个内角之和小于二直角,则该两直线必在这一侧相交.
\end{description}

\subsection{皮亚诺(Peano)自然数理论}

\subsubsection{皮亚诺公理(APeano xioms)}

\newtheorem{PA}{PA}
\begin{PA}\label{PA:1}
	零是个自然数.
\end{PA}
\begin{PA}\label{PA:2}
	每个自然数都有一个后继(也是自然数).
\end{PA}
\begin{PA}\label{PA:3}
	零不是任何自然数的后继.
\end{PA}
\begin{PA}\label{PA:4}
	不同的自然数有不同的后继.
\end{PA}
\begin{PA}[归纳公理]\label{PA:5}
	设由自然数组成的某个集合含有零,且每当该集含有某个自然数时便也同时含有这个数的后继,那么该集定含有全部自然数.
\end{PA}

\subsection{ZFC集论}

策梅洛(E.Zermelo)、弗兰克尔(A.A.Fraenkel)、选择公理(Axiom of Choice)
