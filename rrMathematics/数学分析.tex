\chapter{数学分析}

\section{函数}

\begin{Definition}\label{Definition:Function}
    设数集 $ D \subset \mathbb{R} $,则称映射 $ f: D \to \mathbb{R} $ 为定义在数集 $D$ 上的\rTermWref{函数}{Function}{function},通常简记为
    \[ y = f(x), x \in D \text{,} \]
    其中 $x$ 称为自变量,$y$ 称为因变量,$D$ 称为定义域,记作 $D_f$.
\end{Definition}

\begin{Definition}\label{Definition:ElementaryFunction}
    \rTermWref{初等函数}{ElementaryFunction}{elementary function}是由\rTerm{常数函数}{constant functions}、\rTerm{幂函数}{power functions}、\rTerm{指数函数}{exponential functions}、\rTerm{对数函数}{logarithm functions}、\rTerm{三角函数}{trigonometric functions}和\rTerm{反三角函数}{inverse trigonometric functions}经过有限次的有理运算和有限次的函数复合所产生、且在定义域上能用一个方程式表示的函数.
\end{Definition}

\section{极限}

\subsection{数列的极限}

\section{卷积}
