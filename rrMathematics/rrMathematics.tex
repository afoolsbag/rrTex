% !TeX encoding = UTF-8
% !TeX spellcheck = zh_CN
%% !TeX program = XeLaTeX

\documentclass{ctexbook}
% 导言区开始

% 布局
\usepackage[margin=1in,marginparwidth=60pt]{geometry}
\usepackage{layout}

% 文本支持
\usepackage{enumitem}
\usepackage{multicol}
\usepackage{multirow}
\usepackage{ulem}
\usepackage[dvipsnames,svgnames,x11names]{xcolor}

% 数学支持
\usepackage{amssymb}
\usepackage{amsthm}
\usepackage{mathtools}
\usepackage{unicode-math}

% 参考文献
\usepackage[backend=biber,style=gb7714-2015]{biblatex}
\addbibresource[location=local]{rrMathematics.bib}

% 超链接
\usepackage[colorlinks=true,linkcolor=blue,anchorcolor=violet,citecolor=magenta,unicode=true]{hyperref}

% 快速语法检查
\usepackage{syntonly}
%\syntaxonly

%-------------------------------------------------------------------------------
% 自定义文本命令

% Color Label
\newcommand{\rColorLbl}[1]{\fcolorbox{gray}{#1}{\color{#1}\rule{0.75em}{1ex}} {\slshape\small#1}}
% Command Without Arguments
\newcommand{\rCmdW}[1]{{\ttfamily\textbackslash#1}}
% Command With Optional Arguments
\newcommand{\rCmdO}[2]{{\ttfamily\textbackslash#1[\textcolor{gray}{#2}]}}
% Command With Mandatory Arguments
\newcommand{\rCmdM}[2]{{\ttfamily\textbackslash#1\{\textcolor{gray}{#2}\}}}
% Command With Both Arguments
\newcommand{\rCmdB}[3]{{\ttfamily\textbackslash#1[\textcolor{gray}{#2}]\{\textcolor{gray}{#3}\}}}

% 边注
% Margin Note
\newcommand{\rMarginNote}[1]{\marginpar[\footnotesize#1]{\footnotesize#1}}
% Margin Note & Hypertext Reference
\newcommand{\rMarginNoteHref}[2]{\marginpar[\footnotesize\href{#1}{#2}]{\footnotesize\href{#1}{#2}}}
% Margin Note & Wolfram Math World Reference
\newcommand{\rMarginNoteWref}[2]{\marginpar[\footnotesize\href{http://mathworld.wolfram.com/#1.html}{#2}]{\footnotesize\href{http://mathworld.wolfram.com/#1.html}{#2}}}

% 术语
% Term
\newcommand{\rTerm}[2]{\emph{#1}\rMarginNote{#2}}
% Term & Hypertext Reference
\newcommand{\rTermHref}[3]{\emph{#1}\rMarginNoteHref{#2}{#3}}
% Term & Wolfram Math World Reference
\newcommand{\rTermWref}[3]{\emph{#1}\rMarginNoteWref{#2}{#3}}

% Package Cite
\newcommand{\rPkgCite}[1]{\textsf{#1}\cite{#1}}
% Unicode Number
\newcommand{\rUniNum}[1]{\colorbox{Mulberry}{\href{https://unicode-table.com/#1/}{\color{White}\ttfamily\bfseries{}U+#1}}}

% 符号
% Unicode
\providecommand{\Unicode}{\ttfamily{}Unicode\textregistered{}}

%-------------------------------------------------------------------------------
% 自定义数学命令

% Hat Note
\newcommand{\rHatNote}[2]{\overset{\text{\footnotesize{}#2}}{#1}}

\begin{document}
% 正文区开始

% 标题页
\title{rrMathematics}
\author{zhengrr}
\date{\today}
\maketitle

% 目录页
\tableofcontents

% 章节
\chapter{\LaTeXe}

\section{排版文本}

\subsection{转义字符}

\begin{table}[h!]
    \centering
    \begin{tabular}{c l l}
        \hline
        \#             & \verb|\#|             & 井号     \\
        \$             & \verb|\$|             & 美刀符   \\
        \%             & \verb|\%|             & 百分号   \\
        \&             & \verb|\&|             & 和号     \\
        \textbackslash & \verb|\textbackslash| & 反斜线   \\
        \^{}           & \verb|\^{}|           & 脱字符   \\
        \_             & \verb|\_|             & 下划线   \\
        \{             & \verb|\{|             & 左花括号 \\
        \}             & \verb|\}|             & 右花括号 \\
        \~{}           & \verb|\~{}|           & 波浪号   \\
        \hline
    \end{tabular}
\end{table}

\subsection{标点符号}

\begin{table}[h!]
    \centering
    \begin{tabular}{c l l}
        \hline
        -          & \verb|-|          & 连字符 \\
        --         & \verb|--|         & 连接号 \\
        ---        & \verb|---|        & 破折号 \\
        `'         & \verb|`'|         & 单引号 \\
        ``''       & \verb|``''|       & 双引号 \\
        \P         & \verb|\P|         & 段落符 \\
        \S         & \verb|\S|         & 分节符 \\
        \copyright & \verb|\copyright| & 版权符 \\
        \dag       & \verb|\dag|       & 剑标   \\
        \ddag      & \verb|\ddag|      & 双剑标 \\
        \dots      & \verb|\dots|      & 省略号 \\
        \pounds    & \verb|\pounds|    & 英镑符 \\
        \hline
    \end{tabular}
\end{table}

\subsection{文本样式}

\begin{table}[h!]
    \centering
    \begin{tabular}{l l l l}
        \hline
        \textrm{roman}           & \rCmdM{textrm}{text}     & \rCmd{rmfamily}   & 罗马体(衬线字体) \\
        \textsf{sans serif}      & \rCmdM{textsf}{text}     & \rCmd{sffamily}   & 无衬线字体         \\
        \texttt{typewriter}      & \rCmdM{texttt}{text}     & \rCmd{ttfamily}   & 等宽字体           \\
        \hline
        \textbf{bold face}       & \rCmdM{textbf}{text}     & \rCmd{bfseries}   & 粗体               \\
        \textmd{medium}          & \rCmdM{textmd}{text}     & \rCmd{mdseries}   & 中等粗细           \\
        \hline
        \textit{italic}          & \rCmdM{textit}{text}     & \rCmd{itshape}    & 意大利斜体         \\
        \textsl{slanted}         & \rCmdM{textsl}{text}     & \rCmd{slshape}    & 倾斜体             \\
        \textsc{Small Caps}      & \rCmdM{textsc}{text}     & \rCmd{scshape}    & 小写字母大写       \\
        \textup{upright}         & \rCmdM{textup}{text}     & \rCmd{upshape}    & 直立体             \\
        \hline
        \emph{emphasized}        & \rCmdM{emph}{text}       & \rCmd{em}         & 强调(默认为斜体) \\
        \textnormal{normal font} & \rCmdM{textnormal}{text} & \rCmd{normalfont} & 默认字体           \\
        \uline{underlined}       & \rCmdM{uline}{text}      &                   & 下划线             \\
        \underline{underlined}   & \rCmdM{underline}{text}  &                   & 下划线             \\
        \hline
    \end{tabular}
\end{table}

\subsection{文本尺寸}

\begin{table}[h!]
    \centering
    \begin{tabular}{c c c c c c}
        \rCmd{tiny}              & \rCmd{scriptsize}        & \rCmd{footnotesize}          & \rCmd{small}   & \rCmd{normalsize}        &              \\
        \hline
        {\tiny tiny}             & {\scriptsize scriptsize} & {\footnotesize footnotesize} & {\small small} & {\normalsize normalsize} &              \\
        {\normalsize normalsize} & {\large large}           & {\Large Large}               & {\LARGE LARGE} & {\huge huge}             & {\Huge Huge} \\
        \hline
        \rCmd{normalsize}        & \rCmd{large}             & \rCmd{Large}                 & \rCmd{LARGE}   & \rCmd{huge}              & \rCmd{Huge}  \\
    \end{tabular}
\end{table}

\subsection{文本颜色}

依赖 \rPkgCite{xcolor} 宏包提供颜色支持:

\subsubsection{\rPkgCite{color} 宏包颜色}

\begin{table}[h!]
    \centering
    \begin{tabular}{l l l l}
        \rColLbl{black} & \rColLbl{red}  & \rColLbl{green}   & \rColLbl{blue}   \\
        \rColLbl{white} & \rColLbl{cyan} & \rColLbl{magenta} & \rColLbl{yellow} \\
    \end{tabular}
\end{table}

\subsubsection{\rPkgCite{xcolor} 宏包颜色}

\begin{table}[h!]
    \centering
    \begin{tabular}{l l l l}
        \rColLbl{darkgray} & \rColLbl{gray}  & \rColLbl{lightgray} &                \\
        \rColLbl{brown}    & \rColLbl{olive} & \rColLbl{orange}    & \rColLbl{lime} \\
        \rColLbl{purple}   & \rColLbl{teal}  & \rColLbl{violet}    & \rColLbl{pink} \\
    \end{tabular}
\end{table}

\subsubsection{\texttt{dvipsnames} 选项颜色}

\begin{multicols}{5}
    \noindent
    \rColLbl{Apricot}        \\
    \rColLbl{Aquamarine}     \\
    \rColLbl{Bittersweet}    \\
    \rColLbl{Black}          \\
    \rColLbl{Blue}           \\
    \rColLbl{BlueGreen}      \\
    \rColLbl{BlueViolet}     \\
    \rColLbl{BrickRed}       \\
    \rColLbl{Brown}          \\
    \rColLbl{BurntOrange}    \\
    \rColLbl{CadetBlue}      \\
    \rColLbl{CarnationPink}  \\
    \rColLbl{Cerulean}       \\
    \rColLbl{CornflowerBlue} \\
    \rColLbl{Cyan}           \\
    \rColLbl{Dandelion}      \\
    \rColLbl{DarkOrchid}     \\
    \rColLbl{Emerald}        \\
    \rColLbl{ForestGreen}    \\
    \rColLbl{Fuchsia}        \\
    \rColLbl{Goldenrod}      \\
    \rColLbl{Gray}           \\
    \rColLbl{Green}          \\
    \rColLbl{GreenYellow}    \\
    \rColLbl{JungleGreen}    \\
    \rColLbl{Lavender}       \\
    \rColLbl{LimeGreen}      \\
    \rColLbl{Magenta}        \\
    \rColLbl{Mahogany}       \\
    \rColLbl{Maroon}         \\
    \rColLbl{Melon}          \\
    \rColLbl{MidnightBlue}   \\
    \rColLbl{Mulberry}       \\
    \rColLbl{NavyBlue}       \\
    \rColLbl{OliveGreen}     \\
    \rColLbl{Orange}         \\
    \rColLbl{OrangeRed}      \\
    \rColLbl{Orchid}         \\
    \rColLbl{Peach}          \\
    \rColLbl{Periwinkle}     \\
    \rColLbl{PineGreen}      \\
    \rColLbl{Plum}           \\
    \rColLbl{ProcessBlue}    \\
    \rColLbl{Purple}         \\
    \rColLbl{RawSienna}      \\
    \rColLbl{Red}            \\
    \rColLbl{RedOrange}      \\
    \rColLbl{RedViolet}      \\
    \rColLbl{Rhodamine}      \\
    \rColLbl{RoyalBlue}      \\
    \rColLbl{RoyalPurple}    \\
    \rColLbl{RubineRed}      \\
    \rColLbl{Salmon}         \\
    \rColLbl{SeaGreen}       \\
    \rColLbl{Sepia}          \\
    \rColLbl{SkyBlue}        \\
    \rColLbl{SpringGreen}    \\
    \rColLbl{Tan}            \\
    \rColLbl{TealBlue}       \\
    \rColLbl{Thistle}        \\
    \rColLbl{Turquoise}      \\
    \rColLbl{Violet}         \\
    \rColLbl{VioletRed}      \\
    \rColLbl{White}          \\
    \rColLbl{WildStrawberry} \\
    \rColLbl{Yellow}         \\
    \rColLbl{YellowGreen}    \\
    \rColLbl{YellowOrange}   \\
\end{multicols}

\subsubsection{\texttt{svgnames} 选项颜色}

\begin{multicols}{4}
    \noindent
    \rColLbl{AliceBlue}            \\
    \rColLbl{AntiqueWhite}         \\
    \rColLbl{Aqua}                 \\
    \rColLbl{Aquamarine}           \\
    \rColLbl{Azure}                \\
    \rColLbl{Beige}                \\
    \rColLbl{Bisque}               \\
    \rColLbl{Black}                \\
    \rColLbl{BlanchedAlmond}       \\  
    \rColLbl{Blue}                 \\
    \rColLbl{BlueViolet}           \\
    \rColLbl{Brown}                \\
    \rColLbl{BurlyWood}            \\
    \rColLbl{CadetBlue}            \\
    \rColLbl{Chartreuse}           \\
    \rColLbl{Chocolate}            \\
    \rColLbl{Coral}                \\
    \rColLbl{CornflowerBlue}       \\
    \rColLbl{Cornsilk}             \\
    \rColLbl{Crimson}              \\
    \rColLbl{Cyan}                 \\
    \rColLbl{DarkBlue}             \\
    \rColLbl{DarkCyan}             \\
    \rColLbl{DarkGoldenrod}        \\
    \rColLbl{DarkGray}             \\
    \rColLbl{DarkGreen}            \\
    \rColLbl{DarkGrey}             \\
    \rColLbl{DarkKhaki}            \\
    \rColLbl{DarkMagenta}          \\
    \rColLbl{DarkOliveGreen}       \\
    \rColLbl{DarkOrange}           \\
    \rColLbl{DarkOrchid}           \\
    \rColLbl{DarkRed}              \\
    \rColLbl{DarkSalmon}           \\
    \rColLbl{DarkSeaGreen}         \\
    \rColLbl{DarkSlateBlue}        \\
    \rColLbl{DarkSlateGray}        \\
    \rColLbl{DarkSlateGrey}        \\
    \rColLbl{DarkTurquoise}        \\
    \rColLbl{DarkViolet}           \\
    \rColLbl{DeepPink}             \\
    \rColLbl{DeepSkyBlue}          \\
    \rColLbl{DimGray}              \\
    \rColLbl{DimGrey}              \\
    \rColLbl{DodgerBlue}           \\
    \rColLbl{FireBrick}            \\
    \rColLbl{FloralWhite}          \\
    \rColLbl{ForestGreen}          \\
    \rColLbl{Fuchsia}              \\
    \rColLbl{Gainsboro}            \\
    \rColLbl{GhostWhite}           \\
    \rColLbl{Gold}                 \\
    \rColLbl{Goldenrod}            \\
    \rColLbl{Gray}                 \\
    \rColLbl{Green}                \\
    \rColLbl{GreenYellow}          \\
    \rColLbl{Grey}                 \\
    \rColLbl{Honeydew}             \\
    \rColLbl{HotPink}              \\
    \rColLbl{IndianRed}            \\
    \rColLbl{Indigo}               \\
    \rColLbl{Ivory}                \\
    \rColLbl{Khaki}                \\
    \rColLbl{Lavender}             \\
    \rColLbl{LavenderBlush}        \\
    \rColLbl{LawnGreen}            \\
    \rColLbl{LemonChiffon}         \\
    \rColLbl{LightBlue}            \\
    \rColLbl{LightCoral}           \\
    \rColLbl{LightCyan}            \\
    \rColLbl{LightGoldenrod}       \\
    \rColLbl{LightGoldenrodYellow} \\
    \rColLbl{LightGray}            \\
    \rColLbl{LightGreen}           \\
    \rColLbl{LightGrey}            \\
    \rColLbl{LightPink}            \\
    \rColLbl{LightSalmon}          \\
    \rColLbl{LightSeaGreen}        \\
    \rColLbl{LightSkyBlue}         \\
    \rColLbl{LightSlateBlue}       \\
    \rColLbl{LightSlateGray}       \\
    \rColLbl{LightSlateGrey}       \\
    \rColLbl{LightSteelBlue}       \\
    \rColLbl{LightYellow}          \\
    \rColLbl{Lime}                 \\
    \rColLbl{LimeGreen}            \\
    \rColLbl{Linen}                \\
    \rColLbl{Magenta}              \\
    \rColLbl{Maroon}               \\
    \rColLbl{MediumAquamarine}     \\
    \rColLbl{MediumBlue}           \\
    \rColLbl{MediumOrchid}         \\
    \rColLbl{MediumPurple}         \\
    \rColLbl{MediumSeaGreen}       \\
    \rColLbl{MediumSlateBlue}      \\
    \rColLbl{MediumSpringGreen}    \\
    \rColLbl{MediumTurquoise}      \\
    \rColLbl{MediumVioletRed}      \\
    \rColLbl{MidnightBlue}         \\
    \rColLbl{MintCream}            \\
    \rColLbl{MistyRose}            \\
    \rColLbl{Moccasin}             \\
    \rColLbl{NavajoWhite}          \\
    \rColLbl{Navy}                 \\
    \rColLbl{NavyBlue}             \\
    \rColLbl{OldLace}              \\
    \rColLbl{Olive}                \\
    \rColLbl{OliveDrab}            \\
    \rColLbl{Orange}               \\
    \rColLbl{OrangeRed}            \\
    \rColLbl{Orchid}               \\
    \rColLbl{PaleGoldenrod}        \\
    \rColLbl{PaleGreen}            \\
    \rColLbl{PaleTurquoise}        \\
    \rColLbl{PaleVioletRed}        \\
    \rColLbl{PapayaWhip}           \\
    \rColLbl{PeachPuff}            \\
    \rColLbl{Peru}                 \\
    \rColLbl{Pink}                 \\
    \rColLbl{Plum}                 \\
    \rColLbl{PowderBlue}           \\
    \rColLbl{Purple}               \\
    \rColLbl{Red}                  \\
    \rColLbl{RosyBrown}            \\
    \rColLbl{RoyalBlue}            \\
    \rColLbl{SaddleBrown}          \\
    \rColLbl{Salmon}               \\
    \rColLbl{SandyBrown}           \\
    \rColLbl{SeaGreen}             \\
    \rColLbl{Seashell}             \\
    \rColLbl{Sienna}               \\
    \rColLbl{Silver}               \\
    \rColLbl{SkyBlue}              \\
    \rColLbl{SlateBlue}            \\
    \rColLbl{SlateGray}            \\
    \rColLbl{SlateGrey}            \\
    \rColLbl{Snow}                 \\
    \rColLbl{SpringGreen}          \\
    \rColLbl{SteelBlue}            \\
    \rColLbl{Tan}                  \\
    \rColLbl{Teal}                 \\
    \rColLbl{Thistle}              \\
    \rColLbl{Tomato}               \\
    \rColLbl{Turquoise}            \\
    \rColLbl{Violet}               \\
    \rColLbl{VioletRed}            \\
    \rColLbl{Wheat}                \\
    \rColLbl{White}                \\
    \rColLbl{WhiteSmoke}           \\
    \rColLbl{Yellow}               \\
    \rColLbl{YellowGreen}          \\
\end{multicols}

\subsubsection{\texttt{x11names} 选项颜色}

\begin{multicols}{4}
    \noindent
    \rColLbl{AntiqueWhite1}   \\
    \rColLbl{AntiqueWhite2}   \\
    \rColLbl{AntiqueWhite3}   \\
    \rColLbl{AntiqueWhite4}   \\
    \rColLbl{Aquamarine1}     \\
    \rColLbl{Aquamarine2}     \\
    \rColLbl{Aquamarine3}     \\
    \rColLbl{Aquamarine4}     \\
    \rColLbl{Azure1}          \\
    \rColLbl{Azure2}          \\
    \rColLbl{Azure3}          \\
    \rColLbl{Azure4}          \\
    \rColLbl{Bisque1}         \\
    \rColLbl{Bisque2}         \\
    \rColLbl{Bisque3}         \\
    \rColLbl{Bisque4}         \\
    \rColLbl{Blue1}           \\
    \rColLbl{Blue2}           \\
    \rColLbl{Blue3}           \\
    \rColLbl{Blue4}           \\
    \rColLbl{Brown1}          \\
    \rColLbl{Brown2}          \\
    \rColLbl{Brown3}          \\
    \rColLbl{Brown4}          \\
    \rColLbl{Burlywood1}      \\
    \rColLbl{Burlywood2}      \\
    \rColLbl{Burlywood3}      \\
    \rColLbl{Burlywood4}      \\
    \rColLbl{CadetBlue1}      \\
    \rColLbl{CadetBlue2}      \\
    \rColLbl{CadetBlue3}      \\
    \rColLbl{CadetBlue4}      \\
    \rColLbl{Chartreuse1}     \\
    \rColLbl{Chartreuse2}     \\
    \rColLbl{Chartreuse3}     \\
    \rColLbl{Chartreuse4}     \\
    \rColLbl{Chocolate1}      \\
    \rColLbl{Chocolate2}      \\
    \rColLbl{Chocolate3}      \\
    \rColLbl{Chocolate4}      \\
    \rColLbl{Coral1}          \\
    \rColLbl{Coral2}          \\
    \rColLbl{Coral3}          \\
    \rColLbl{Coral4}          \\
    \rColLbl{Cornsilk1}       \\
    \rColLbl{Cornsilk2}       \\
    \rColLbl{Cornsilk3}       \\
    \rColLbl{Cornsilk4}       \\
    \rColLbl{Cyan1}           \\
    \rColLbl{Cyan2}           \\
    \rColLbl{Cyan3}           \\
    \rColLbl{Cyan4}           \\
    \rColLbl{DarkGoldenrod1}  \\
    \rColLbl{DarkGoldenrod2}  \\
    \rColLbl{DarkGoldenrod3}  \\
    \rColLbl{DarkGoldenrod4}  \\
    \rColLbl{DarkOliveGreen1} \\
    \rColLbl{DarkOliveGreen2} \\
    \rColLbl{DarkOliveGreen3} \\
    \rColLbl{DarkOliveGreen4} \\
    \rColLbl{DarkOrange1}     \\
    \rColLbl{DarkOrange2}     \\
    \rColLbl{DarkOrange3}     \\
    \rColLbl{DarkOrange4}     \\
    \rColLbl{DarkOrchid1}     \\
    \rColLbl{DarkOrchid2}     \\
    \rColLbl{DarkOrchid3}     \\
    \rColLbl{DarkOrchid4}     \\
    \rColLbl{DarkSeaGreen1}   \\
    \rColLbl{DarkSeaGreen2}   \\
    \rColLbl{DarkSeaGreen3}   \\
    \rColLbl{DarkSeaGreen4}   \\
    \rColLbl{DarkSlateGray1}  \\
    \rColLbl{DarkSlateGray2}  \\
    \rColLbl{DarkSlateGray3}  \\
    \rColLbl{DarkSlateGray4}  \\
    \rColLbl{DeepPink1}       \\
    \rColLbl{DeepPink2}       \\
    \rColLbl{DeepPink3}       \\
    \rColLbl{DeepPink4}       \\
    \rColLbl{DeepSkyBlue1}    \\
    \rColLbl{DeepSkyBlue2}    \\
    \rColLbl{DeepSkyBlue3}    \\
    \rColLbl{DeepSkyBlue4}    \\
    \rColLbl{DodgerBlue1}     \\
    \rColLbl{DodgerBlue2}     \\
    \rColLbl{DodgerBlue3}     \\
    \rColLbl{DodgerBlue4}     \\
    \rColLbl{Firebrick1}      \\
    \rColLbl{Firebrick2}      \\
    \rColLbl{Firebrick3}      \\
    \rColLbl{Firebrick4}      \\
    \rColLbl{Gold1}           \\
    \rColLbl{Gold2}           \\
    \rColLbl{Gold3}           \\
    \rColLbl{Gold4}           \\
    \rColLbl{Goldenrod1}      \\
    \rColLbl{Goldenrod2}      \\
    \rColLbl{Goldenrod3}      \\
    \rColLbl{Goldenrod4}      \\
    \rColLbl{Green1}          \\
    \rColLbl{Green2}          \\
    \rColLbl{Green3}          \\
    \rColLbl{Green4}          \\
    \rColLbl{Honeydew1}       \\
    \rColLbl{Honeydew2}       \\
    \rColLbl{Honeydew3}       \\
    \rColLbl{Honeydew4}       \\
    \rColLbl{HotPink1}        \\
    \rColLbl{HotPink2}        \\
    \rColLbl{HotPink3}        \\
    \rColLbl{HotPink4}        \\
    \rColLbl{IndianRed1}      \\
    \rColLbl{IndianRed2}      \\
    \rColLbl{IndianRed3}      \\
    \rColLbl{IndianRed4}      \\
    \rColLbl{Ivory1}          \\
    \rColLbl{Ivory2}          \\
    \rColLbl{Ivory3}          \\
    \rColLbl{Ivory4}          \\
    \rColLbl{Khaki1}          \\
    \rColLbl{Khaki2}          \\
    \rColLbl{Khaki3}          \\
    \rColLbl{Khaki4}          \\
    \rColLbl{LavenderBlush1}  \\
    \rColLbl{LavenderBlush2}  \\
    \rColLbl{LavenderBlush3}  \\
    \rColLbl{LavenderBlush4}  \\
    \rColLbl{LemonChiffon1}   \\
    \rColLbl{LemonChiffon2}   \\
    \rColLbl{LemonChiffon3}   \\
    \rColLbl{LemonChiffon4}   \\
    \rColLbl{LightBlue1}      \\
    \rColLbl{LightBlue2}      \\
    \rColLbl{LightBlue3}      \\
    \rColLbl{LightBlue4}      \\
    \rColLbl{LightCyan1}      \\
    \rColLbl{LightCyan2}      \\
    \rColLbl{LightCyan3}      \\
    \rColLbl{LightCyan4}      \\
    \rColLbl{LightGoldenrod1} \\
    \rColLbl{LightGoldenrod2} \\
    \rColLbl{LightGoldenrod3} \\
    \rColLbl{LightGoldenrod4} \\
    \rColLbl{LightPink1}      \\
    \rColLbl{LightPink2}      \\
    \rColLbl{LightPink3}      \\
    \rColLbl{LightPink4}      \\
    \rColLbl{LightSalmon1}    \\
    \rColLbl{LightSalmon2}    \\
    \rColLbl{LightSalmon3}    \\
    \rColLbl{LightSalmon4}    \\
    \rColLbl{LightSkyBlue1}   \\
    \rColLbl{LightSkyBlue2}   \\
    \rColLbl{LightSkyBlue3}   \\
    \rColLbl{LightSkyBlue4}   \\
    \rColLbl{LightSteelBlue1} \\
    \rColLbl{LightSteelBlue2} \\
    \rColLbl{LightSteelBlue3} \\
    \rColLbl{LightSteelBlue4} \\
    \rColLbl{LightYellow1}    \\
    \rColLbl{LightYellow2}    \\
    \rColLbl{LightYellow3}    \\
    \rColLbl{LightYellow4}    \\
    \rColLbl{Magenta1}        \\
    \rColLbl{Magenta2}        \\
    \rColLbl{Magenta3}        \\
    \rColLbl{Magenta4}        \\
    \rColLbl{Maroon1}         \\
    \rColLbl{Maroon2}         \\
    \rColLbl{Maroon3}         \\
    \rColLbl{Maroon4}         \\
    \rColLbl{MediumOrchid1}   \\
    \rColLbl{MediumOrchid2}   \\
    \rColLbl{MediumOrchid3}   \\
    \rColLbl{MediumOrchid4}   \\
    \rColLbl{MediumPurple1}   \\
    \rColLbl{MediumPurple2}   \\
    \rColLbl{MediumPurple3}   \\
    \rColLbl{MediumPurple4}   \\
    \rColLbl{MistyRose1}      \\
    \rColLbl{MistyRose2}      \\
    \rColLbl{MistyRose3}      \\
    \rColLbl{MistyRose4}      \\
    \rColLbl{NavajoWhite1}    \\
    \rColLbl{NavajoWhite2}    \\
    \rColLbl{NavajoWhite3}    \\
    \rColLbl{NavajoWhite4}    \\
    \rColLbl{OliveDrab1}      \\
    \rColLbl{OliveDrab2}      \\
    \rColLbl{OliveDrab3}      \\
    \rColLbl{OliveDrab4}      \\
    \rColLbl{Orange1}         \\
    \rColLbl{Orange2}         \\
    \rColLbl{Orange3}         \\
    \rColLbl{Orange4}         \\
    \rColLbl{OrangeRed1}      \\
    \rColLbl{OrangeRed2}      \\
    \rColLbl{OrangeRed3}      \\
    \rColLbl{OrangeRed4}      \\
    \rColLbl{Orchid1}         \\
    \rColLbl{Orchid2}         \\
    \rColLbl{Orchid3}         \\
    \rColLbl{Orchid4}         \\
    \rColLbl{PaleGreen1}      \\
    \rColLbl{PaleGreen2}      \\
    \rColLbl{PaleGreen3}      \\
    \rColLbl{PaleGreen4}      \\
    \rColLbl{PaleTurquoise1}  \\
    \rColLbl{PaleTurquoise2}  \\
    \rColLbl{PaleTurquoise3}  \\
    \rColLbl{PaleTurquoise4}  \\
    \rColLbl{PaleVioletRed1}  \\
    \rColLbl{PaleVioletRed2}  \\
    \rColLbl{PaleVioletRed3}  \\
    \rColLbl{PaleVioletRed4}  \\
    \rColLbl{PeachPuff1}      \\
    \rColLbl{PeachPuff2}      \\
    \rColLbl{PeachPuff3}      \\
    \rColLbl{PeachPuff4}      \\
    \rColLbl{Pink1}           \\
    \rColLbl{Pink2}           \\
    \rColLbl{Pink3}           \\
    \rColLbl{Pink4}           \\
    \rColLbl{Plum1}           \\
    \rColLbl{Plum2}           \\
    \rColLbl{Plum3}           \\
    \rColLbl{Plum4}           \\
    \rColLbl{Purple1}         \\
    \rColLbl{Purple2}         \\
    \rColLbl{Purple3}         \\
    \rColLbl{Purple4}         \\
    \rColLbl{Red1}            \\
    \rColLbl{Red2}            \\
    \rColLbl{Red3}            \\
    \rColLbl{Red4}            \\
    \rColLbl{RosyBrown1}      \\
    \rColLbl{RosyBrown2}      \\
    \rColLbl{RosyBrown3}      \\
    \rColLbl{RosyBrown4}      \\
    \rColLbl{RoyalBlue1}      \\
    \rColLbl{RoyalBlue2}      \\
    \rColLbl{RoyalBlue3}      \\
    \rColLbl{RoyalBlue4}      \\
    \rColLbl{Salmon1}         \\
    \rColLbl{Salmon2}         \\
    \rColLbl{Salmon3}         \\
    \rColLbl{Salmon4}         \\
    \rColLbl{SeaGreen1}       \\
    \rColLbl{SeaGreen2}       \\
    \rColLbl{SeaGreen3}       \\
    \rColLbl{SeaGreen4}       \\
    \rColLbl{Seashell1}       \\
    \rColLbl{Seashell2}       \\
    \rColLbl{Seashell3}       \\
    \rColLbl{Seashell4}       \\
    \rColLbl{Sienna1}         \\
    \rColLbl{Sienna2}         \\
    \rColLbl{Sienna3}         \\
    \rColLbl{Sienna4}         \\
    \rColLbl{SkyBlue1}        \\
    \rColLbl{SkyBlue2}        \\
    \rColLbl{SkyBlue3}        \\
    \rColLbl{SkyBlue4}        \\
    \rColLbl{SlateBlue1}      \\
    \rColLbl{SlateBlue2}      \\
    \rColLbl{SlateBlue3}      \\
    \rColLbl{SlateBlue4}      \\
    \rColLbl{SlateGray1}      \\
    \rColLbl{SlateGray2}      \\
    \rColLbl{SlateGray3}      \\
    \rColLbl{SlateGray4}      \\
    \rColLbl{Snow1}           \\
    \rColLbl{Snow2}           \\
    \rColLbl{Snow3}           \\
    \rColLbl{Snow4}           \\
    \rColLbl{SpringGreen1}    \\
    \rColLbl{SpringGreen2}    \\
    \rColLbl{SpringGreen3}    \\
    \rColLbl{SpringGreen4}    \\
    \rColLbl{SteelBlue1}      \\
    \rColLbl{SteelBlue2}      \\
    \rColLbl{SteelBlue3}      \\
    \rColLbl{SteelBlue4}      \\
    \rColLbl{Tan1}            \\
    \rColLbl{Tan2}            \\
    \rColLbl{Tan3}            \\
    \rColLbl{Tan4}            \\
    \rColLbl{Thistle1}        \\
    \rColLbl{Thistle2}        \\
    \rColLbl{Thistle3}        \\
    \rColLbl{Thistle4}        \\
    \rColLbl{Tomato1}         \\
    \rColLbl{Tomato2}         \\
    \rColLbl{Tomato3}         \\
    \rColLbl{Tomato4}         \\
    \rColLbl{Turquoise1}      \\
    \rColLbl{Turquoise2}      \\
    \rColLbl{Turquoise3}      \\
    \rColLbl{Turquoise4}      \\
    \rColLbl{VioletRed1}      \\
    \rColLbl{VioletRed2}      \\
    \rColLbl{VioletRed3}      \\
    \rColLbl{VioletRed4}      \\
    \rColLbl{Wheat1}          \\
    \rColLbl{Wheat2}          \\
    \rColLbl{Wheat3}          \\
    \rColLbl{Wheat4}          \\
    \rColLbl{Yellow1}         \\
    \rColLbl{Yellow2}         \\
    \rColLbl{Yellow3}         \\
    \rColLbl{Yellow4}         \\
    \rColLbl{Gray0}           \\
    \rColLbl{Green0}          \\
    \rColLbl{Grey0}           \\
    \rColLbl{Maroon0}         \\
    \rColLbl{Purple0}         \\
\end{multicols}

\subsection{超链接}

依赖 \rPkgCite{hyperref} 宏包提供超链接支持。

\newpage
\section{排版文档}

\begin{table}[h!]
    \centering
    \begin{minipage}[t]{0.46\textwidth}
        \centering
        \begin{tabular}{l l}
            \hline
            \rCmdM{chapter}{title}       & 章     \\
            \rCmdM{section}{title}       & 节     \\
            \rCmdM{subsection}{title}    & 小节   \\
            \rCmdM{subsubsection}{title} & 小小节 \\
            \rCmdM{paragraph}{title}     & 段     \\
            \rCmdM{subparagraph}{title}  & 小段   \\
            \rCmd{tableofcontents}       & 目录   \\
            \hline
        \end{tabular}
    \end{minipage}
    \qquad
    \begin{minipage}[t]{0.46\textwidth}
        \centering
        \begin{tabular}{l l}
            \hline
            \hline
        \end{tabular}
    \end{minipage}
\end{table}

\newpage
\layout

\newpage
\section{排版公式}

美国数学协会(American Mathematical Society)提供 \AmS 宏集以扩展 \LaTeX 公式排版,其核心是 \rPkgCite{amsmath} 宏包,对多行公式的排版提供了有力的支持.
此外,\rPkgCite{amsfonts} 宏包以及基于它的 \textsf{amssymb} 宏包提供了丰富的数学符号;\rPkgCite{amsthm} 宏包扩展了 \LaTeX 定理证明格式.

本节所述命令多用于 \LaTeX 数学模式,使用数学模式的方法有多种:

\begin{enumerate}
    \item \TeX 提供的行内公式:
    \begin{verbatim}
        $ e^{\i\pi} + 1 = 0 $
    \end{verbatim}
    
    \item \TeX 提供的行间公式:
    \begin{verbatim}
        $$
            e^{\i\pi} + 1 = 0
        $$
    \end{verbatim}
    
    \item \LaTeX 提供的行内公式:
    \begin{verbatim}
        \( e^{\i\pi} + 1 = 0 \)
    \end{verbatim}
    
    \item \LaTeX 提供的行间公式:
    \begin{verbatim}
        \[  # 或 \begin{displaymath}
            e^{\i\pi} + 1 = 0
        \]  # 或 \end{displaymath}
    \end{verbatim}
    
    \item \LaTeX 提供的有序号行间公式:
    \begin{verbatim}
        \begin{equation}
            e^{\i\pi} + 1 = 0
        \end{equation}
    \end{verbatim}
    
    \item \AmS 提供的无序号行间公式:
    \begin{verbatim}
        \begin{equation*}
            e^{\i\pi} + 1 = 0
        \end{equation*}
    \end{verbatim}
\end{enumerate}

另外,对于 \LaTeX 符号的检索,此外链文档\cite{symbols}提供了符号列表,此外链工具\cite{Detexify}提供了符号搜索.

\newpage
\subsection{希腊字母}

\begin{table}[h!]
    \centering
    \begin{tabular}{c@{ }l c@{ }l c@{ }l c@{ }l c@{ }l c@{ }l}
        \hline
        \multicolumn{2}{c}{\Unicode} & \multicolumn{4}{c}{\LaTeX{}} & \multicolumn{2}{c}{\Unicode} & \multicolumn{4}{c}{\LaTeX{}} \\
        \hline
        Α & \rUniNum{0391} &        $A$ & \verb|A|        &               &                    & α & \rUniNum{03B1} &   $\alpha$ & \verb|\alpha|   &               &                    \\
        Β & \rUniNum{0392} &        $B$ & \verb|B|        &               &                    & β & \rUniNum{03B2} &    $\beta$ & \verb|\beta|    &               &                    \\
        Γ & \rUniNum{0393} &   $\Gamma$ & \verb|\Gamma|   &   $\varGamma$ & \verb|\varGamma|   & γ & \rUniNum{03B3} &   $\gamma$ & \verb|\gamma|   &               &                    \\
        Δ & \rUniNum{0394} &   $\Delta$ & \verb|\Delta|   &   $\varDelta$ & \verb|\varDelta|   & δ & \rUniNum{03B4} &   $\delta$ & \verb|\delta|   &               &                    \\
        Ε & \rUniNum{0395} &        $E$ & \verb|E|        &               &                    & ε & \rUniNum{03B5} & $\epsilon$ & \verb|\epsilon| & $\varepsilon$ & \verb|\varepsilon| \\
        Ζ & \rUniNum{0396} &        $Z$ & \verb|Z|        &               &                    & ζ & \rUniNum{03B6} &    $\zeta$ & \verb|\zeta|    &               &                    \\
        Η & \rUniNum{0397} &        $H$ & \verb|H|        &               &                    & η & \rUniNum{03B7} &     $\eta$ & \verb|\eta|     &               &                    \\
        Θ & \rUniNum{0398} &   $\Theta$ & \verb|\Theta|   &   $\varTheta$ & \verb|\varTheta|   & θ & \rUniNum{03B8} &   $\theta$ & \verb|\theta|   &   $\vartheta$ & \verb|\vartheta|   \\
        Ι & \rUniNum{0399} &        $I$ & \verb|I|        &               &                    & ι & \rUniNum{03B9} &    $\iota$ & \verb|\iota|    &               &                    \\
        Κ & \rUniNum{039A} &        $K$ & \verb|K|        &               &                    & κ & \rUniNum{03BA} &   $\kappa$ & \verb|\kappa|   &   $\varkappa$ & \verb|\varkappa|   \\
        Λ & \rUniNum{039B} &  $\Lambda$ & \verb|\Lambda|  &  $\varLambda$ & \verb|\varLambda|  & λ & \rUniNum{03BB} &  $\lambda$ & \verb|\lambda|  &               &                    \\
        Μ & \rUniNum{039C} &        $M$ & \verb|M|        &               &                    & μ & \rUniNum{03BC} &      $\mu$ & \verb|\mu|      &               &                    \\
        Ν & \rUniNum{039D} &        $N$ & \verb|N|        &               &                    & ν & \rUniNum{03BD} &      $\nu$ & \verb|\nu|      &               &                    \\
        Ξ & \rUniNum{039E} &      $\Xi$ & \verb|\Xi|      &      $\varXi$ & \verb|\varXi|      & ξ & \rUniNum{03BE} &      $\xi$ & \verb|\xi|      &               &                    \\
        Ο & \rUniNum{039F} &        $O$ & \verb|O|        &               &                    & ο & \rUniNum{03BF} &        $o$ & \verb|o|        &               &                    \\
        Π & \rUniNum{03A0} &      $\Pi$ & \verb|\Pi|      &      $\varPi$ & \verb|\varPi|      & π & \rUniNum{03C0} &      $\pi$ & \verb|\pi|      &      $\varpi$ & \verb|\varpi|      \\
        Ρ & \rUniNum{03A1} &        $P$ & \verb|P|        &               &                    & ρ & \rUniNum{03C1} &     $\rho$ & \verb|\rho|     &     $\varrho$ & \verb|\varrho|     \\
        Σ & \rUniNum{03A3} &   $\Sigma$ & \verb|\Sigma|   &   $\varSigma$ & \verb|\varSigma|   & σ & \rUniNum{03C2} &   $\sigma$ & \verb|\sigma|   &   $\varsigma$ & \verb|\varsigma|   \\
        Τ & \rUniNum{03A4} &        $T$ & \verb|T|        &               &                    & τ & \rUniNum{03C3} &     $\tau$ & \verb|\tau|     &               &                    \\
        Υ & \rUniNum{03A5} & $\Upsilon$ & \verb|\Upsilon| & $\varUpsilon$ & \verb|\varUpsilon| & υ & \rUniNum{03C4} & $\upsilon$ & \verb|\upsilon| &               &                    \\
        Φ & \rUniNum{03A6} &     $\Phi$ & \verb|\Phi|     &     $\varPhi$ & \verb|\varPhi|     & φ & \rUniNum{03C5} &     $\phi$ & \verb|\phi|     &     $\varphi$ & \verb|\varphi|     \\
        Χ & \rUniNum{03A7} &        $X$ & \verb|X|        &               &                    & χ & \rUniNum{03C6} &     $\chi$ & \verb|\chi|     &               &                    \\
        Ψ & \rUniNum{03A8} &     $\Psi$ & \verb|\Psi|     &     $\varPsi$ & \verb|\varPsi|     & ψ & \rUniNum{03C7} &     $\psi$ & \verb|\psi|     &               &                    \\
        Ω & \rUniNum{03A9} &   $\Omega$ & \verb|\Omega|   &   $\varOmega$ & \verb|\varOmega|   & ω & \rUniNum{03C8} &   $\omega$ & \verb|\omega|   &               &                    \\
        Ͱ & \rUniNum{0370} &            &                 &               &                    & ͱ & \rUniNum{0371} &            &                 &               &                    \\
        Ͳ & \rUniNum{0372} &            &                 &               &                    & ͳ & \rUniNum{0373} &            &                 &               &                    \\
        Ͷ & \rUniNum{0376} &            &                 &               &                    & ͷ & \rUniNum{0377} &            &                 &               &                    \\
        Ϙ & \rUniNum{03D8} &            &                 &               &                    & ϙ & \rUniNum{03D9} &            &                 &               &                    \\
        Ϛ & \rUniNum{03DA} &            &                 &               &                    & ϛ & \rUniNum{03DB} &            &                 &               &                    \\
        Ϝ & \rUniNum{03DC} &            &                 &               &                    & ϝ & \rUniNum{03DD} & $\digamma$ & \verb|\digamma| &               &                    \\
        Ϟ & \rUniNum{03DE} &            &                 &               &                    & ϟ & \rUniNum{03DF} &            &                 &               &                    \\
        Ϡ & \rUniNum{03E0} &            &                 &               &                    & ϡ & \rUniNum{03E1} &            &                 &               &                    \\
        \hline
    \end{tabular}
\end{table}

\newpage
\subsection{拉丁字母}

\begin{table}[h!]
    \centering
    \begin{tabular}{c@{ }l c@{ }l c@{ }l c@{ }l}
        \hline
        \multicolumn{2}{c}{\Unicode} & \multicolumn{2}{c}{\LaTeX{}} & \multicolumn{2}{c}{\Unicode} & \multicolumn{2}{c}{\LaTeX{}} \\
        \hline
        A & \rUniNum{0041} & $A$ & \verb|A| & a & \rUniNum{0061} & $a$ & \verb|a| \\
        B & \rUniNum{0042} & $B$ & \verb|B| & b & \rUniNum{0062} & $b$ & \verb|b| \\
        C & \rUniNum{0043} & $C$ & \verb|C| & c & \rUniNum{0063} & $c$ & \verb|c| \\
        D & \rUniNum{0044} & $D$ & \verb|D| & e & \rUniNum{0064} & $e$ & \verb|e| \\
        E & \rUniNum{0045} & $E$ & \verb|E| & r & \rUniNum{0065} & $r$ & \verb|r| \\
        F & \rUniNum{0046} & $F$ & \verb|F| & f & \rUniNum{0066} & $f$ & \verb|f| \\
        G & \rUniNum{0047} & $G$ & \verb|G| & g & \rUniNum{0067} & $g$ & \verb|g| \\
        H & \rUniNum{0048} & $H$ & \verb|H| & h & \rUniNum{0068} & $h$ & \verb|h| \\
        I & \rUniNum{0049} & $I$ & \verb|I| & i & \rUniNum{0069} & $i$ & \verb|i| \\
        J & \rUniNum{004A} & $J$ & \verb|J| & j & \rUniNum{006A} & $j$ & \verb|j| \\
        K & \rUniNum{004B} & $K$ & \verb|K| & k & \rUniNum{006B} & $k$ & \verb|k| \\
        L & \rUniNum{004C} & $L$ & \verb|L| & l & \rUniNum{006C} & $l$ & \verb|l| \\
        M & \rUniNum{004D} & $M$ & \verb|M| & m & \rUniNum{006D} & $m$ & \verb|m| \\
        N & \rUniNum{004E} & $N$ & \verb|N| & n & \rUniNum{006E} & $n$ & \verb|n| \\
        O & \rUniNum{004F} & $O$ & \verb|O| & o & \rUniNum{006F} & $o$ & \verb|o| \\
        P & \rUniNum{0050} & $P$ & \verb|P| & p & \rUniNum{0070} & $p$ & \verb|p| \\
        Q & \rUniNum{0051} & $Q$ & \verb|Q| & q & \rUniNum{0071} & $q$ & \verb|q| \\
        R & \rUniNum{0052} & $R$ & \verb|R| & r & \rUniNum{0072} & $r$ & \verb|r| \\
        S & \rUniNum{0053} & $S$ & \verb|S| & s & \rUniNum{0073} & $s$ & \verb|s| \\
        T & \rUniNum{0054} & $T$ & \verb|T| & t & \rUniNum{0074} & $t$ & \verb|t| \\
        U & \rUniNum{0055} & $U$ & \verb|U| & u & \rUniNum{0075} & $u$ & \verb|u| \\
        V & \rUniNum{0056} & $V$ & \verb|V| & v & \rUniNum{0076} & $v$ & \verb|v| \\
        W & \rUniNum{0057} & $W$ & \verb|W| & w & \rUniNum{0077} & $w$ & \verb|w| \\
        X & \rUniNum{0058} & $X$ & \verb|X| & x & \rUniNum{0078} & $x$ & \verb|x| \\
        Y & \rUniNum{0059} & $Y$ & \verb|Y| & y & \rUniNum{0079} & $y$ & \verb|y| \\
        Z & \rUniNum{005A} & $Z$ & \verb|Z| & z & \rUniNum{007A} & $z$ & \verb|z| \\
        \hline
    \end{tabular}
\end{table}

\begin{table}[h!]
    \centering
    \begin{tabular}{l l l}
        \hline
        $\mathbb{ABC}$           & \rCmdM{mathbb}{math}     & 黑板粗体(仅大写),常用于表示特殊集合 \\
        $\mathbf{ABCdef123}$     & \rCmdM{mathbf}{math}     & 粗体,常用于向量                       \\
        $\mathcal{ABC}$          & \rCmdM{mathcal}{math}    & 书法体(仅大写),常用于层、概型和范畴 \\
        $\mathfrak{ABCabc123}$   & \rCmdM{mathfrak}{math}   & 一种德国风格粗体,常用于群和环         \\
        $\mathit{ABCdef123}$     & \rCmdM{mathit}{math}     & 意大利斜体                             \\
        $\mathnormal{ABCdef123}$ & \rCmdM{mathnormal}{math} & 默认字体                               \\
        $\mathrm{ABCdef123}$     & \rCmdM{mathrm}{math}     & 罗马体(衬线字体),常用于单位和函数   \\
        $\mathsf{ABCdef123}$     & \rCmdM{mathsf}{math}     & 无衬线字体                             \\
        $\mathtt{ABCdef123}$     & \rCmdM{mathtt}{math}     & 等宽字体                               \\
        \hline
    \end{tabular}
\end{table}

\chapter{数学基础}

本章主要参照\citeauthor{WangFt2001}先生的\citetitle{WangFt2001}\cite{WangFt2001}.

\section{历史概述}

人类数学大体上经历了三个大的发展阶段:以几何数学为主体的初等数学阶段,以分析数学为主体的古典数学阶段,以集论数学为主体的现代数学阶段.

\section{逻辑准备}

\subsection{形式语言初步}

\rTermWref{形式语言}{FormalLanguage}{formal language}可以由\rTermWref{形式文法}{Grammar}{formal grammar}描述,形式文法是一种四元组$G=(N,\Sigma,P,S)$,其中:
\begin{itemize}
	\item \rTerm{非终结符}{nonterminal symbols}有限集$N$;
	\item \rTerm{终结符}{terminal symbols}有限集$\Sigma$,且$N\cap\Sigma=\emptyset$;集$V=N\cup\Sigma$称为\rTerm{词汇表}{vocabulary};
	\item \rTerm{产生式规则}{production rules}有限集$P$;
	\item \rTerm{开始符号}{start symbol}$S$,且$S\in{}N$.
\end{itemize}

譬如,一种可以表示二进制整数的形式文法和一例推导示例:
\begin{multicols}{2}
	\[
	G_\text{binary} = (N, \Sigma, P, S), \begin{cases}
	    N      = \{ S_\text{(start)}, D_\text{(digital)} \} \\
	    \Sigma = \{ -, 0, 1 \}                              \\
	    P      = \begin{cases}
	                 \text{1.} S \to D  \\
	                 \text{2.} S \to -D \\
	                 \text{3.} D \to 0D \\
	                 \text{4.} D \to 1D \\
	                 \text{5.} D \to 0  \\
	                 \text{6.} D \to 1  \\
	             \end{cases}                                \\
	    S                                                   \\
	\end{cases}
	\] \\
	\[
	\begin{aligned}
	    S &\underset{2}{\Rightarrow} -D    \\
	      &\underset{4}{\Rightarrow} -1D   \\
	      &\underset{3}{\Rightarrow} -10D  \\
	      &\underset{4}{\Rightarrow} -101D \\
	      &\underset{5}{\Rightarrow} -1010 \\
	\end{aligned}
	\]
\end{multicols}

\subsection{命题演算初步}

\rTermWref{命题演算}{PropositionalCalculus}{propositional calculus}是一种\rTerm{形式系统}{formal system},可描述为$\mathcal{L}(\mathrm{A},\Omega,\mathrm{Z},\mathrm{I})$,其中:
\begin{itemize}
	\item \rTerm{命题变量}{propositional variables}集$\rHatNote{\mathrm{A}}{``alpha''}=\{p,q,r,\cdots\}$;
	\item \rTerm{逻辑联结词}{logical connectives}集
	      $\rHatNote{\Omega}{``omega''} = \Omega_0 \cup \Omega_1 \cup \Omega_2, \begin{cases}
	            \Omega_0 = \{ \rHatNote{\top}{真}, \rHatNote{\bot}{假} \}                                                                              \text{,} \\
	            \Omega_1 = \{ \rHatNote{\lnot}{否定词} \}                                                                                              \text{,} \\
	            \Omega_2 = \{ \rHatNote{\land}{合取词}, \rHatNote{\lor}{析取词}, \rHatNote{\to}{蕴含词}, \rHatNote{\leftrightarrow}{等价词}, \cdots \} \text{.} \\
	       \end{cases}$
	\item \rTerm{推理规则}{inference rules}集
	      $\rHatNote{\mathrm{Z}}{``zeta''} = \begin{cases}
	           \text{每个命题变量都是公式;}                                                                                          \\
	           \text{若$p$和$q$是公式,则$\lnot{}p$、$p\land{}q$、$p\lor{}q$、$p\to{}q$、$p\leftrightarrow{}q$、$\cdots$也都是公式;} \\
	           \text{除外都不是公式.}                                                                                                \\
	       \end{cases}$
	\item \rTerm{公理}{axioms}集$\rHatNote{\mathrm{I}}{``iota''}$.
\end{itemize}

\subsubsection{逻辑联结词}

\begin{table}[h!]
	\centering
	\newcommand{\T}{{\color{blue}$\top$}}
	\newcommand{\F}{{\color{red}$\bot$}}
	\newcommand{\FoF}{\F$\circ$\F}
	\newcommand{\FoT}{\F$\circ$\T}
	\newcommand{\ToF}{\T$\circ$\F}
	\newcommand{\ToT}{\T$\circ$\T}
	\begin{tabular}{c c c c c c c c l}
		\multicolumn{4}{c}{等价公式} & \multicolumn{4}{c}{真值表} & 助记 \\
		\hline
		\multicolumn{4}{c}{$p\circ{}q$}                                                                                                                            & \FoF & \FoT & \ToF & \ToT &          \\
		\hline
		\multicolumn{4}{c}{$\rHatNote{\top}{``真''}$}                                                                                                              & \T   & \T   & \T   & \T   & 恒真     \\
		$\rHatNote{p\uparrow{}q}{``公式$p$与非公式$q$''}$          & $p\rightarrow\lnot{}q$     & $\lnot{}p\leftarrow{}q$      & $\lnot{}p\lor\lnot{}q$            & \T   & \T   & \T   & \F   & 存假为真 \\
		$\rHatNote{p\to{}q}{``公式$p$蕴含公式$q$''}$               & $p\uparrow\lnot{}q$        & $\lnot{}p\lor{}q$            & $\lnot{}p\gets\lnot{}q$           & \T   & \T   & \F   & \T   &          \\
		\multicolumn{4}{c}{$\rHatNote{\lnot{}p}{``非公式$p$''}$}                                                                                                   & \T   & \T   & \F   & \F   &          \\
		$\rHatNote{p\gets{}q}{``公式$p$蕴含于公式$q$''}$           & $p\lor\lnot{}q$            & $\lnot{}p\uparrow{}q$        & $\lnot{}p\to\lnot{}q$             & \T   & \F   & \T   & \T   &          \\
		\multicolumn{4}{c}{$\rHatNote{\lnot{}q}{``非公式$q$''}$}                                                                                                   & \T   & \F   & \T   & \F   &          \\
		$\rHatNote{p\leftrightarrow{}q}{``公式$p$等价于公式$q$''}$ & $p\oplus\lnot{}q$          & $\lnot{}p\oplus{}q$          & $\lnot{}p\leftrightarrow\lnot{}q$ & \T   & \F   & \F   & \T   & 相同为真 \\
		$\rHatNote{p\downarrow{}q}{``公式$p$或非公式$q$''}$        & $p\not\gets\lnot{}q$       & $\lnot{}p\not\to{}q$         & $\lnot{}p\land\lnot{}q$           & \T   & \F   & \F   & \F   & 全假为真 \\
		$\rHatNote{p\lor{}q}{``公式$p$或公式$q$''}$                & $p\gets\lnot{}q$           & $\lnot{}p\to{}q$             & $\lnot{}p\uparrow\lnot{}q$        & \F   & \T   & \T   & \T   & 存真为真 \\
		$\rHatNote{p\oplus{}q}{``公式$p$异或公式$q$''}$            & $p\leftrightarrow\lnot{}q$ & $\lnot{}p\leftrightarrow{}q$ & $\lnot{}p\oplus\lnot{}q$          & \F   & \T   & \T   & \F   & 相异为真 \\
		\multicolumn{4}{c}{$\rHatNote{q}{``公式$q$''}$}                                                                                                            & \F   & \T   & \F   & \T   &          \\
		$\rHatNote{p\not\gets{}q}{``公式$p$非蕴含于公式$q$''}$     & $p\downarrow\lnot{}q$      & $\lnot{}p\land{}q$           & $\lnot{}p\not\to\lnot{}q$         & \F   & \T   & \F   & \F   &          \\
		\multicolumn{4}{c}{$\rHatNote{p}{``公式$p$''}$}                                                                                                            & \F   & \F   & \T   & \T   &          \\
		$\rHatNote{p\not\to{}q}{``公式$p$非蕴含公式$q$''}$         & $p\land\lnot{}q$           & $\lnot{}p\downarrow{}q$      & $\lnot{}p\not\gets\lnot{}q$       & \F   & \F   & \T   & \F   &          \\
		$\rHatNote{p\land{}q}{``公式$p$与公式$q$''}$               & $p\not\to\lnot{}q$         & $\lnot{}p\not\gets{}q$       & $\lnot{}p\downarrow\lnot{}q$      & \F   & \F   & \F   & \T   & 全真为真 \\
		\multicolumn{4}{c}{$\rHatNote{\bot}{``假''}$}                                                                                                              & \F   & \F   & \F   & \F   & 恒假     \\
	\end{tabular}
\end{table}

\subsubsection{永真式}

\[ p \to p                                                             \text{,} \tag{同一律} \]
\[ \lnot{}p \lor p                                                     \text{,} \tag{排中律} \]
\[ \lnot(\lnot{}p \land p)                                             \text{,} \tag{矛盾律} \]
\[ \lnot\lnot{}p \leftrightarrow p                                     \text{,} \tag{双重否定律} \]
\[ (p \land q) \leftrightarrow (q \land p)                             \text{,} \tag{合取交换律} \]
\[ ((p \land q) \land r) \leftrightarrow (p \land (q \land r))         \text{,} \tag{合取结合律} \]
\[ (p \land (q \lor r)) \leftrightarrow ((p \land q) \lor (p \land r)) \text{,} \tag{分配律} \]
\[ \lnot(p \land q) \leftrightarrow (\lnot{}p \lor \lnot{}q)           \text{,} \tag{De Morgan 律} \]
\[ (p \lor q) \leftrightarrow (q \lor p)                               \text{,} \tag{析取交换律} \]
\[ ((p \lor q) \lor r) \leftrightarrow (p \lor (q \lor r))             \text{,} \tag{析取结合律} \]
\[ (p \lor (q \land r)) \leftrightarrow ((p \lor q) \land (p \lor r))  \text{,} \tag{分配律} \]
\[ \lnot(p \lor q) \leftrightarrow (\lnot{}p \land \lnot{}q)           \text{,} \tag{De Morgan 律} \]
\[ \lnot{}p \to (p \to q)                                              \text{,} \tag{否定前件律} \]
\[ q \to (p \to q)                                                     \text{,} \tag{肯定后件律} \]
\[ (p \to (q \to r)) \to ((p \to q) \to (p \to r))                     \text{,} \tag{蕴含词分配律} \]
\[ (\lnot{}p \to \lnot{}q) \to (q \to p)                               \text{,} \tag{换位律} \]
\[ (\lnot{}p \to p) \to p                                              \text{,} \tag{否定肯定律} \]
\[ (p \to q) \to ((q \to r) \to (p \to r))                             \text{.} \tag{假设三段论} \]

\subsection{谓词演算初步}

\rTermWref{谓词演算}{PredicateCalculus}{predicate calculus}是一种形式系统,可描述为$\mathcal{L}(\mathrm{A},\Omega,\mathrm{Z},\mathrm{I})$,其中:
\begin{itemize}
	\item 代词集$\rHatNote{\mathrm{A}}{``alpha''}=\{x, y, z, \cdots\}$;
	\item 终结符集
	      $\rHatNote{\Omega}{``omega''} = \Omega_\text{量词} \cup \Omega_\text{名词} \cup \Omega_\text{映射} \cup \Omega_\text{谓词} \cup \Omega_\text{逻辑联结词} = \begin{cases}
	          \Omega_\text{量词} = \{ \rHatNote{\forall}{``全称量词''}, \rHatNote{\exists}{``存在量词''} \} \\
	          \Omega_\text{名词} = \{ a, b, c, \cdots \}                                                    \\
	          \Omega_\text{映射} = \{ f, g, h, \cdots \}                                                    \\
	          \Omega_\text{谓词} = \{ R, S, T, \cdots \}                                                    \\
	          \Omega_\text{逻辑联结词} = \{ \lnot, \land, \lor, \to, \leftrightarrow, \cdots \}             \\
	       \end{cases}$
	\item 推理规则集
	      $\rHatNote{\mathrm{Z}}{``zeta''} = \begin{cases}
	          \text{每个代词、名词都是项;}                                                                                          \\
	          \text{若$f$是$n$元映射,且$t_1$、$\cdots$、$t_n$是项,则$f(t_1,\cdots,t_n)$也是项;}                                   \\
	          \text{除外都不是项;}                                                                                                  \\
	          \text{若$R$是$n$元谓词,且$t_1$、$\cdots$、$t_n$是项,则$R(t_1,\cdots,t_n)$是公式;}                                   \\
	          \text{若$p$和$q$是公式,则$\lnot{}p$、$p\land{}q$、$p\lor{}q$、$p\to{}q$、$p\leftrightarrow{}q$、$\cdots$也都是公式;} \\
	          \text{若$x$是代词,且$p$是公式,则$\forall{}xp$、$\exists{}xp$也是公式;}                                              \\
	          \text{除外都不是公式.}                                                                                                \\
	       \end{cases}$
	\item 公理集$\rHatNote{\mathrm{I}}{``iota''}$.
\end{itemize}

\section{集论概念}

\rTermWref{ZF集论}{Zermelo-FraenkelSetTheory}{Zermelo-Fraenkel Set Theory}是一种谓词演算,其代词叫做``集'',其谓词有二:$\rHatNote{=}{``等于''}$和$\rHatNote{\in}{``属于''}$.

\newtheorem{ZF}{ZF}

\begin{ZF}[\emph{外延公理}]\rMarginNote{axiom of extensionality}\label{ZF:1}
	$\forall x \forall y ( \forall z (z \in x \leftrightarrow z \in y) \leftrightarrow x = y )$
\end{ZF}
\begin{ZF}[内涵公理]\label{ZF:2}
	$\forall s \exists y \forall x ( x \in y \leftrightarrow x \in s \land p(x) )$
\end{ZF}

\chapter{集合、映射和二元运算}

\section{集合}

\subsection{集合概念}

一般的,\emph{集合}(\href{http://mathworld.wolfram.com/Set.html}{Set},简称\emph{集})是指具有某种特定性质的事物的总体,组成这个集合的事物称为该集合的\emph{元素}(\href{http://mathworld.wolfram.com/Element.html}{Element},简称\emph{元}).
通常用大写拉丁字母 $A$,$B$,$C$,$\cdots$ 表示集合,用小写拉丁字母 $a$,$b$,$c$,$\cdots$ 表示集合的元素.

如果元素 $e$ 是集合 $S$ 的元素,就说``元素 $e$ \emph{属于}集合 $S$''(记作 $e\in{}S$)或``集合 $S$ \emph{拥有}元素 $e$''(记作 $S\ni{}e$).
如果元素 $e$ 不是集合 $S$ 的元素,就说``元素 $e$ \emph{不属于}集合 $S$''(记作 $e\not\in{}S$)或``集合 $S$ \emph{不拥有}元素 $e$''(记作 $S\not\ni{}e$).
一个集合,若它只拥有限个元素,则称为\emph{有限集};不是有限集的集合称为\emph{无限集}.

通常使用\emph{列举法}、\emph{描述法}或\emph{文氏图}(\href{http://mathworld.wolfram.com/VennDiagram.html}{Venn Diagram})表示集合:
例如,由元素 $a_1$,$a_2$,$\cdots$,$a_n$ 组成的集合 $A$ 可表示为
\[
A = \{ a_1, a_2, \cdots, a_n \} \text{;}
\]
由具有某种性质 $P$ 的元素 $b$ 的全体组成的集合 $B$ 可表示为
\[
B = \{ b | b \text{具有性质} P \} \text{.}
\]

\subsection{集合间的关系}

如果集合 $S$ 的元素都是集合 $L$ 的元素,则称集合 $S$ 是集合 $L$ 的\emph{子集}(\href{http://mathworld.wolfram.com/Subset.html}{Subset}),记作 $S\subset{}L$(读作``集合 $S$ \emph{包含于}集合 $L$'');
或称集合 $L$ 是集合 $S$ 的\emph{超集}(\href{http://mathworld.wolfram.com/Superset.html}{Superset}),记作 $L\supset{}S$(读作``集合 $L$ \emph{包含}集合 $S$'').

如果集合 $A$ 的元素都是集合 $B$ 的元素,且集合 $B$ 的元素也都是集合 $A$ 的元素,即集合 $A$ 与集合 $B$ 互相包含,则称集合 $A$ 与集合 $B$ \emph{相等},记作 $A=B$(读作``集合 $A$ \emph{等于}集合 $B$'').

如果集合 $S$ 包含于集合 $L$,且它们不相等,则称集合 $S$ 是集合 $L$ 的\emph{真子集}(\href{http://mathworld.wolfram.com/ProperSubset.html}{Proper Subset}),记作 $S\subsetneqq{}L$(读作``集合 $S$ \emph{真包含于}集合 $L$'');
或称集合 $L$ 是集合 $S$ 的\emph{真超集}(\href{http://mathworld.wolfram.com/ProperSuperset.html}{Proper Superset}),记作 $L\supsetneqq{}S$(读作``集合 $L$ \emph{真包含}集合 $S$'').

不拥有任何元素的集合称为\emph{空集}(\href{http://mathworld.wolfram.com/EmptySet.html}{Empty Set}),记作 $\varnothing$,规定空集是任何集合的子集.
有时,在指定上下文中可以将所有研究对象组成一个集合 $U$,所研究的其它集合都是集合 $U$ 的子集,此时我们称集合 $U$ 为\emph{全集}(\href{http://mathworld.wolfram.com/UniversalSet.html}{Universe}).

\subsection{并集、交集、差集、对称差集、补集和余集}

由所有属于集合 $A$ 或者属于集合 $B$ 的元素组成的集合,称为集合 $A$ 与集合 $B$ 的\emph{并集}(\href{http://mathworld.wolfram.com/Union.html}{Union},简称\emph{并}),记作 $A\cup{}B$,该运算也称为\emph{并}(\href{http://mathworld.wolfram.com/Cup.html}{Cup}),即
\[
A \cup B = \{ e | e \in A \text{或} e \in B \} \text{;}
\]
由所有既属于集合 $A$ 又属于集合 $B$ 的元素组成的集合,称为集合 $A$ 与集合 $B$ 的\emph{交集}(\href{http://mathworld.wolfram.com/Intersection.html}{Intersection},简称\emph{交}),记作 $A\cap{}B$,该运算也称为\emph{交}(\href{http://mathworld.wolfram.com/Cap.html}{Cap}),即
\[
A \cap B = \{ e | e \in A \text{且} e \in B \} \text{;}
\]
由所有属于集合 $A$ 但不属于集合 $B$ 的元素组成的集合,称为集合 $A$ 与集合 $B$ 的\emph{差集}(\href{http://mathworld.wolfram.com/SetDifference.html}{Set Difference},简称\emph{差}),记作 $A\setminus{}B$,该运算称为\emph{减}(\href{http://mathworld.wolfram.com/SetMinus.html}{Set Minus}),即
\[
A \setminus B = \{ e | e \in A \text{且} e \not\in B \} \text{;}
\]
由所有属于集合 $A$ 或者属于集合 $B$ 但不同时属于两集合的元素组成的集合,称为集合 $A$ 与集合 $B$ 的\emph{对称差集}(\href{http://mathworld.wolfram.com/SymmetricDifference.html}{Symmetric Difference}),记作 $A\ominus{}B$,即
\[
A \ominus B = (A \cup B) \setminus (A \cap B) \text{;}
\]
设有超集 $L$ 包含子集 $S$,则称超集 $L$ 与子集 $S$ 的差集,为超集 $L$ 中子集 $S$ 的\emph{补集}(\href{http://mathworld.wolfram.com/ComplementSet.html}{Complement Set}),记作 $\complement_LS$,即
\[
\begin{aligned}
\text{存在超集$L$和其子集$S$,} \complement_LS & = L \setminus S \\
                                     & = \{ e | e \in L \text{且} e \not\in S \} \text{;}
\end{aligned}
\]
特别的,若存在全集 $U$,则称全集 $U$ 中子集 $S$ 的补集,为集合 $S$ 的\emph{余集},记作 $S^c$,即
\[
\begin{aligned}
\text{存在全集$U$,} S^c &= \complement_US \\
                         &= U \setminus S \\
                         &= \{ e | e \in U \text{且} e \not\in S \} \text{.}
\end{aligned}
\]

\subsection{笛卡尔乘积}

由所有``属于集合 $A$ 的任一元素 $a$ 和属于集合 $B$ 的任一元素 $b$ 组成的有序对 $(a, b)$'' 组成的集合,称为集合 $A$ 与集合 $B$ 的\emph{笛卡尔乘积}(\href{http://mathworld.wolfram.com/CartesianProduct.html}{Cartesian Product}),记作 $A\times{}B$,即
\[
A \times B = \{ (a, b) | a \in A \text{且} b \in B \} \text{.}
\]

\newpage
\section{映射}

\subsection{映射概念}

设集合 $X$ 和集合 $Y$ 是两个非空集合,对于集合 $X$ 中任一元素 $x$,依照某种对应规律 $f$,恒有集合 $Y$ 中唯一确定元素 $y$ 与之对应,则称对应规律 $f$ 为一个从集合 $X$ 到集合 $Y$ 的映射(\href{http://mathworld.wolfram.com/Map.html}{Map}),记作
\[
\begin{aligned}
                f&: X \to Y \\
\text{或} \quad f&: x \mapsto y \text{,}
\end{aligned}
\]
其中:
\begin{itemize}
	\item 集合 $X$ 称为映射 $f$ 的\emph{定义域}(\href{http://mathworld.wolfram.com/Domain.html}{Domain}),记作 $D_f$;
	\item 集合 $Y$ 称为映射 $f$ 的\emph{陪域}(\href{http://mathworld.wolfram.com/Codomain.html}{Codomain}),记作 $C_f$;
	\item 元素 $x$ 称为在映射 $f$ 下元素 $y$ 的一个\emph{原像}(\href{http://mathworld.wolfram.com/Preimage.html}{Preimage});
	\item 元素 $y=f(x)$ 称为在映射 $f$ 下元素 $x$ 的\emph{像}(\href{http://mathworld.wolfram.com/Image.html}{Image});
	\item 集合 $f(X)=\{f(x)|x\in{}X\}$ 称为映射 $f$ 的\emph{值域}(\href{http://mathworld.wolfram.com/Range.html}{Range}),记作 $R_f$.
\end{itemize}

若对于定义域 $D_f$ 中任意原像 $d_1$、$d_2$ 不等,对应的像 $f(d_1)$、$f(d_2)$ 恒不等,则称映射 $f$ 是\emph{单射}(\href{http://mathworld.wolfram.com/Injection.html}{Injection});
若对于陪域 $C_f$ 中任一元素 $c$,都存在原像 $d$ 使得 $f(d)=c$,则称映射 $f$ 是\emph{满射}(\href{http://mathworld.wolfram.com/Surjection.html}{Surjection});
若映射 $f$ 既是单射又是满射,则称其为\emph{双射}(\href{http://mathworld.wolfram.com/Bijection.html}{Bijection}).

\subsection{逆映射}

设映射 $f$ 是一个从定义域 $D_f$ 到陪域 $C_f$ 的单射,则对于值域 $R_f$ 中任一像 $r$ 都存在唯一原像 $d$ 与之对应,于是可以定义一个从值域 $R_f$ 到定义域 $D_f$ 的新映射
\[
f^{-1}: r \mapsto d \text{,}
\]
这个映射称为映射 $f$ 的\emph{逆映射}(\href{http://mathworld.wolfram.com/InverseFunction.html}{Inverse Function}),其定义域 $D_{f^{-1}}=R_f$,值域 $R_{f^{-1}}=D_f$.

\subsection{复合映射}

设有两映射 $g=X\to{}Y_S$ 和 $f=Y_L\to{}Z$ 且 $Y_S\subset{}Y_L$,则对于集合 $X$ 中任一元素 $x$ 恒有集合 $Z$ 中唯一确定元素 $z=f(g(x))$ 与之对应,于是可以定义一个从集合 $X$ 到集合 $Z$ 的新映射
\[
f \circ g: x \mapsto f(g(x)) \text{,}
\]
这个映射称为映射 $g$ 和映射 $f$ 构成的\emph{复合映射}(\href{http://mathworld.wolfram.com/Composition.html}{Composition}).

\newpage
\section{二元运算}

从集合 $S$ 的笛卡尔平方 $S\times{}S$ 到集合 $S$ 的映射 $f:S\times{}S\to{}S$,称作集合 $S$ 上的\emph{二元运算}(\href{http://mathworld.wolfram.com/BinaryOperation.html}{Binary Operation});
若元素 $a$、$b$ 是集合 $S$ 的元素,通常将二元运算的像 $f(a, b)$ 记作 $a f b$.

%设集合 $S$ 上的二元运算 $*:S\times{}S\to{}S$,元素 $e_0$、$e_1$ 是集合 $S$ 的元素:
%\begin{itemize}
%	\item 若 $\forall a \in S \implies e_0 * a = e_0$,则称元素 $e_0$ 为二元运算 $*$ 的左零元;
%\end{itemize}


% 附录
\appendix

% 参考文献
\nocite{*}
\printbibliography[heading=bibliography,title=参考文献]

\end{document}
