% !TeX encoding = UTF-8
% !TeX spellcheck = zh_CN
%% !TeX program = XeLaTeX

\documentclass{ctexbook}
% 导言区开始

% 布局
\usepackage[margin=1in,marginparwidth=60pt]{geometry}
\usepackage{layout}

% 文本支持
\usepackage{enumitem}
\usepackage{multicol}
\usepackage{multirow}
\usepackage{ulem}
\usepackage[dvipsnames,svgnames,x11names]{xcolor}

% 数学支持
\usepackage{amssymb}
\usepackage{amsthm}
\usepackage{mathtools}
\usepackage{unicode-math}

% 参考文献
\usepackage[backend=biber,style=gb7714-2015]{biblatex}
\addbibresource[location=local]{rrMathematics.bib}

% 超链接
\usepackage[colorlinks=true,linkcolor=blue,anchorcolor=violet,citecolor=magenta,unicode=true]{hyperref}

% 快速语法检查
\usepackage{syntonly}
%\syntaxonly

%-------------------------------------------------------------------------------
% 自定义文本命令

% Color Label
\newcommand{\rColorLbl}[1]{\fcolorbox{gray}{#1}{\color{#1}\rule{0.75em}{1ex}} {\slshape\small#1}}
% Command Without Arguments
\newcommand{\rCmdW}[1]{{\ttfamily\textbackslash#1}}
% Command With Optional Arguments
\newcommand{\rCmdO}[2]{{\ttfamily\textbackslash#1[\textcolor{gray}{#2}]}}
% Command With Mandatory Arguments
\newcommand{\rCmdM}[2]{{\ttfamily\textbackslash#1\{\textcolor{gray}{#2}\}}}
% Command With Both Arguments
\newcommand{\rCmdB}[3]{{\ttfamily\textbackslash#1[\textcolor{gray}{#2}]\{\textcolor{gray}{#3}\}}}

% 边注
% Margin Note
\newcommand{\rMarginNote}[1]{\marginpar[\footnotesize#1]{\footnotesize#1}}
% Margin Note & Hypertext Reference
\newcommand{\rMarginNoteHref}[2]{\marginpar[\footnotesize\href{#1}{#2}]{\footnotesize\href{#1}{#2}}}
% Margin Note & Wolfram Math World Reference
\newcommand{\rMarginNoteWref}[2]{\marginpar[\footnotesize\href{http://mathworld.wolfram.com/#1.html}{#2}]{\footnotesize\href{http://mathworld.wolfram.com/#1.html}{#2}}}

% 术语
% Term
\newcommand{\rTerm}[2]{\emph{#1}\rMarginNote{#2}}
% Term & Hypertext Reference
\newcommand{\rTermHref}[3]{\emph{#1}\rMarginNoteHref{#2}{#3}}
% Term & Wolfram Math World Reference
\newcommand{\rTermWref}[3]{\emph{#1}\rMarginNoteWref{#2}{#3}}

% Package Cite
\newcommand{\rPkgCite}[1]{\textsf{#1}\cite{#1}}
% Unicode Number
\newcommand{\rUniNum}[1]{\colorbox{Mulberry}{\href{https://unicode-table.com/#1/}{\color{White}\ttfamily\bfseries{}U+#1}}}

% 符号
% Unicode
\providecommand{\Unicode}{\ttfamily{}Unicode\textregistered{}}

%-------------------------------------------------------------------------------
% 自定义数学命令

% Hat Note
\newcommand{\rHatNote}[2]{\overset{\text{\footnotesize{}#2}}{#1}}

\begin{document}
% 正文区开始

% 标题页
\title{rrMathematics}
\author{zhengrr}
\date{\today}
\maketitle

% 目录页
\tableofcontents

% 章节
\chapter{\LaTeXe}

\section{排版文本}

\subsection{转义字符}

\begin{table}[h!]
	\centering
	\begin{tabular}{c l l}
		\hline
		\#             & \verb|\#|             & 井号     \\
		\$             & \verb|\$|             & 美刀符   \\
		\%             & \verb|\%|             & 百分号   \\
		\&             & \verb|\&|             & 和号     \\
		\textbackslash & \verb|\textbackslash| & 反斜线   \\
		\^{}           & \verb|\^{}|           & 脱字符   \\
		\_             & \verb|\_|             & 下划线   \\
		\{             & \verb|\{|             & 左花括号 \\
		\}             & \verb|\}|             & 右花括号 \\
		\~{}           & \verb|\~{}|           & 波浪号   \\
		\hline
	\end{tabular}
\end{table}

\subsection{标点符号}

\begin{table}[h!]
	\centering
	\begin{tabular}{c l l}
		\hline
		-          & \verb|-|          & 连字符 \\
		--         & \verb|--|         & 连接号 \\
		---        & \verb|---|        & 破折号 \\
		`'         & \verb|`'|         & 单引号 \\
		``''       & \verb|``''|       & 双引号 \\
		\P         & \verb|\P|         & 段落符 \\
		\S         & \verb|\S|         & 分节符 \\
		\copyright & \verb|\copyright| & 版权符 \\
		\dag       & \verb|\dag|       & 剑标   \\
		\ddag      & \verb|\ddag|      & 双剑标 \\
		\dots      & \verb|\dots|      & 省略号 \\
		\pounds    & \verb|\pounds|    & 英镑符 \\
		\hline
	\end{tabular}
\end{table}

\subsection{文本样式}

\begin{table}[h!]
	\centering
	\begin{tabular}{l l l l}
		\hline
		\textrm{roman}           & \rCmdM{textrm}{text}     & \rCmdW{rmfamily}   & 罗马体(衬线字体) \\
		\textsf{sans serif}      & \rCmdM{textsf}{text}     & \rCmdW{sffamily}   & 无衬线字体         \\
		\texttt{typewriter}      & \rCmdM{texttt}{text}     & \rCmdW{ttfamily}   & 等宽字体           \\
		\hline
		\textbf{bold face}       & \rCmdM{textbf}{text}     & \rCmdW{bfseries}   & 粗体               \\
		\textmd{medium}          & \rCmdM{textmd}{text}     & \rCmdW{mdseries}   & 中等粗细           \\
		\hline
		\textit{italic}          & \rCmdM{textit}{text}     & \rCmdW{itshape}    & 意大利斜体         \\
		\textsl{slanted}         & \rCmdM{textsl}{text}     & \rCmdW{slshape}    & 倾斜体             \\
		\textsc{Small Caps}      & \rCmdM{textsc}{text}     & \rCmdW{scshape}    & 小写字母大写       \\
		\textup{upright}         & \rCmdM{textup}{text}     & \rCmdW{upshape}    & 直立体             \\
		\hline
		\emph{emphasized}        & \rCmdM{emph}{text}       & \rCmdW{em}         & 强调(默认为斜体) \\
		\textnormal{normal font} & \rCmdM{textnormal}{text} & \rCmdW{normalfont} & 默认字体           \\
		\uline{underlined}       & \rCmdM{uline}{text}      &                    & 下划线             \\
		\underline{underlined}   & \rCmdM{underline}{text}  &                    & 下划线             \\
		\hline
	\end{tabular}
\end{table}

\subsection{文本尺寸}

\begin{table}[h!]
	\centering
	\begin{tabular}{c c c c c c}
		\rCmdW{tiny}             & \rCmdW{scriptsize}       & \rCmdW{footnotesize}         & \rCmdW{small}  & \rCmdW{normalsize}       &              \\
		\hline
		{\tiny tiny}             & {\scriptsize scriptsize} & {\footnotesize footnotesize} & {\small small} & {\normalsize normalsize} &              \\
		{\normalsize normalsize} & {\large large}           & {\Large Large}               & {\LARGE LARGE} & {\huge huge}             & {\Huge Huge} \\
		\hline
		\rCmdW{normalsize}       & \rCmdW{large}            & \rCmdW{Large}                & \rCmdW{LARGE}  & \rCmdW{huge}             & \rCmdW{Huge} \\
	\end{tabular}
\end{table}

\subsection{文本颜色}

依赖 \rPkgCite{xcolor} 宏包提供颜色支持:

\subsubsection{\rPkgCite{color} 宏包颜色}

\begin{table}[h!]
	\centering
	\begin{tabular}{l l l l}
		\rColorLbl{black} & \rColorLbl{red}  & \rColorLbl{green}   & \rColorLbl{blue}   \\
		\rColorLbl{white} & \rColorLbl{cyan} & \rColorLbl{magenta} & \rColorLbl{yellow} \\
	\end{tabular}
\end{table}

\subsubsection{\rPkgCite{xcolor} 宏包颜色}

\begin{table}[h!]
	\centering
	\begin{tabular}{l l l l}
		\rColorLbl{darkgray} & \rColorLbl{gray}  & \rColorLbl{lightgray} &                  \\
		\rColorLbl{brown}    & \rColorLbl{olive} & \rColorLbl{orange}    & \rColorLbl{lime} \\
		\rColorLbl{purple}   & \rColorLbl{teal}  & \rColorLbl{violet}    & \rColorLbl{pink} \\
	\end{tabular}
\end{table}

\subsubsection{\texttt{dvipsnames} 选项颜色}

\begin{multicols}{5}
	\noindent
	\rColorLbl{Apricot}        \\
	\rColorLbl{Aquamarine}     \\
	\rColorLbl{Bittersweet}    \\
	\rColorLbl{Black}          \\
	\rColorLbl{Blue}           \\
	\rColorLbl{BlueGreen}      \\
	\rColorLbl{BlueViolet}     \\
	\rColorLbl{BrickRed}       \\
	\rColorLbl{Brown}          \\
	\rColorLbl{BurntOrange}    \\
	\rColorLbl{CadetBlue}      \\
	\rColorLbl{CarnationPink}  \\
	\rColorLbl{Cerulean}       \\
	\rColorLbl{CornflowerBlue} \\
	\rColorLbl{Cyan}           \\
	\rColorLbl{Dandelion}      \\
	\rColorLbl{DarkOrchid}     \\
	\rColorLbl{Emerald}        \\
	\rColorLbl{ForestGreen}    \\
	\rColorLbl{Fuchsia}        \\
	\rColorLbl{Goldenrod}      \\
	\rColorLbl{Gray}           \\
	\rColorLbl{Green}          \\
	\rColorLbl{GreenYellow}    \\
	\rColorLbl{JungleGreen}    \\
	\rColorLbl{Lavender}       \\
	\rColorLbl{LimeGreen}      \\
	\rColorLbl{Magenta}        \\
	\rColorLbl{Mahogany}       \\
	\rColorLbl{Maroon}         \\
	\rColorLbl{Melon}          \\
	\rColorLbl{MidnightBlue}   \\
	\rColorLbl{Mulberry}       \\
	\rColorLbl{NavyBlue}       \\
	\rColorLbl{OliveGreen}     \\
	\rColorLbl{Orange}         \\
	\rColorLbl{OrangeRed}      \\
	\rColorLbl{Orchid}         \\
	\rColorLbl{Peach}          \\
	\rColorLbl{Periwinkle}     \\
	\rColorLbl{PineGreen}      \\
	\rColorLbl{Plum}           \\
	\rColorLbl{ProcessBlue}    \\
	\rColorLbl{Purple}         \\
	\rColorLbl{RawSienna}      \\
	\rColorLbl{Red}            \\
	\rColorLbl{RedOrange}      \\
	\rColorLbl{RedViolet}      \\
	\rColorLbl{Rhodamine}      \\
	\rColorLbl{RoyalBlue}      \\
	\rColorLbl{RoyalPurple}    \\
	\rColorLbl{RubineRed}      \\
	\rColorLbl{Salmon}         \\
	\rColorLbl{SeaGreen}       \\
	\rColorLbl{Sepia}          \\
	\rColorLbl{SkyBlue}        \\
	\rColorLbl{SpringGreen}    \\
	\rColorLbl{Tan}            \\
	\rColorLbl{TealBlue}       \\
	\rColorLbl{Thistle}        \\
	\rColorLbl{Turquoise}      \\
	\rColorLbl{Violet}         \\
	\rColorLbl{VioletRed}      \\
	\rColorLbl{White}          \\
	\rColorLbl{WildStrawberry} \\
	\rColorLbl{Yellow}         \\
	\rColorLbl{YellowGreen}    \\
	\rColorLbl{YellowOrange}   \\
\end{multicols}

\subsubsection{\texttt{svgnames} 选项颜色}

\begin{multicols}{4}
	\noindent
	\rColorLbl{AliceBlue}            \\
	\rColorLbl{AntiqueWhite}         \\
	\rColorLbl{Aqua}                 \\
	\rColorLbl{Aquamarine}           \\
	\rColorLbl{Azure}                \\
	\rColorLbl{Beige}                \\
	\rColorLbl{Bisque}               \\
	\rColorLbl{Black}                \\
	\rColorLbl{BlanchedAlmond}       \\  
	\rColorLbl{Blue}                 \\
	\rColorLbl{BlueViolet}           \\
	\rColorLbl{Brown}                \\
	\rColorLbl{BurlyWood}            \\
	\rColorLbl{CadetBlue}            \\
	\rColorLbl{Chartreuse}           \\
	\rColorLbl{Chocolate}            \\
	\rColorLbl{Coral}                \\
	\rColorLbl{CornflowerBlue}       \\
	\rColorLbl{Cornsilk}             \\
	\rColorLbl{Crimson}              \\
	\rColorLbl{Cyan}                 \\
	\rColorLbl{DarkBlue}             \\
	\rColorLbl{DarkCyan}             \\
	\rColorLbl{DarkGoldenrod}        \\
	\rColorLbl{DarkGray}             \\
	\rColorLbl{DarkGreen}            \\
	\rColorLbl{DarkGrey}             \\
	\rColorLbl{DarkKhaki}            \\
	\rColorLbl{DarkMagenta}          \\
	\rColorLbl{DarkOliveGreen}       \\
	\rColorLbl{DarkOrange}           \\
	\rColorLbl{DarkOrchid}           \\
	\rColorLbl{DarkRed}              \\
	\rColorLbl{DarkSalmon}           \\
	\rColorLbl{DarkSeaGreen}         \\
	\rColorLbl{DarkSlateBlue}        \\
	\rColorLbl{DarkSlateGray}        \\
	\rColorLbl{DarkSlateGrey}        \\
	\rColorLbl{DarkTurquoise}        \\
	\rColorLbl{DarkViolet}           \\
	\rColorLbl{DeepPink}             \\
	\rColorLbl{DeepSkyBlue}          \\
	\rColorLbl{DimGray}              \\
	\rColorLbl{DimGrey}              \\
	\rColorLbl{DodgerBlue}           \\
	\rColorLbl{FireBrick}            \\
	\rColorLbl{FloralWhite}          \\
	\rColorLbl{ForestGreen}          \\
	\rColorLbl{Fuchsia}              \\
	\rColorLbl{Gainsboro}            \\
	\rColorLbl{GhostWhite}           \\
	\rColorLbl{Gold}                 \\
	\rColorLbl{Goldenrod}            \\
	\rColorLbl{Gray}                 \\
	\rColorLbl{Green}                \\
	\rColorLbl{GreenYellow}          \\
	\rColorLbl{Grey}                 \\
	\rColorLbl{Honeydew}             \\
	\rColorLbl{HotPink}              \\
	\rColorLbl{IndianRed}            \\
	\rColorLbl{Indigo}               \\
	\rColorLbl{Ivory}                \\
	\rColorLbl{Khaki}                \\
	\rColorLbl{Lavender}             \\
	\rColorLbl{LavenderBlush}        \\
	\rColorLbl{LawnGreen}            \\
	\rColorLbl{LemonChiffon}         \\
	\rColorLbl{LightBlue}            \\
	\rColorLbl{LightCoral}           \\
	\rColorLbl{LightCyan}            \\
	\rColorLbl{LightGoldenrod}       \\
	\rColorLbl{LightGoldenrodYellow} \\
	\rColorLbl{LightGray}            \\
	\rColorLbl{LightGreen}           \\
	\rColorLbl{LightGrey}            \\
	\rColorLbl{LightPink}            \\
	\rColorLbl{LightSalmon}          \\
	\rColorLbl{LightSeaGreen}        \\
	\rColorLbl{LightSkyBlue}         \\
	\rColorLbl{LightSlateBlue}       \\
	\rColorLbl{LightSlateGray}       \\
	\rColorLbl{LightSlateGrey}       \\
	\rColorLbl{LightSteelBlue}       \\
	\rColorLbl{LightYellow}          \\
	\rColorLbl{Lime}                 \\
	\rColorLbl{LimeGreen}            \\
	\rColorLbl{Linen}                \\
	\rColorLbl{Magenta}              \\
	\rColorLbl{Maroon}               \\
	\rColorLbl{MediumAquamarine}     \\
	\rColorLbl{MediumBlue}           \\
	\rColorLbl{MediumOrchid}         \\
	\rColorLbl{MediumPurple}         \\
	\rColorLbl{MediumSeaGreen}       \\
	\rColorLbl{MediumSlateBlue}      \\
	\rColorLbl{MediumSpringGreen}    \\
	\rColorLbl{MediumTurquoise}      \\
	\rColorLbl{MediumVioletRed}      \\
	\rColorLbl{MidnightBlue}         \\
	\rColorLbl{MintCream}            \\
	\rColorLbl{MistyRose}            \\
	\rColorLbl{Moccasin}             \\
	\rColorLbl{NavajoWhite}          \\
	\rColorLbl{Navy}                 \\
	\rColorLbl{NavyBlue}             \\
	\rColorLbl{OldLace}              \\
	\rColorLbl{Olive}                \\
	\rColorLbl{OliveDrab}            \\
	\rColorLbl{Orange}               \\
	\rColorLbl{OrangeRed}            \\
	\rColorLbl{Orchid}               \\
	\rColorLbl{PaleGoldenrod}        \\
	\rColorLbl{PaleGreen}            \\
	\rColorLbl{PaleTurquoise}        \\
	\rColorLbl{PaleVioletRed}        \\
	\rColorLbl{PapayaWhip}           \\
	\rColorLbl{PeachPuff}            \\
	\rColorLbl{Peru}                 \\
	\rColorLbl{Pink}                 \\
	\rColorLbl{Plum}                 \\
	\rColorLbl{PowderBlue}           \\
	\rColorLbl{Purple}               \\
	\rColorLbl{Red}                  \\
	\rColorLbl{RosyBrown}            \\
	\rColorLbl{RoyalBlue}            \\
	\rColorLbl{SaddleBrown}          \\
	\rColorLbl{Salmon}               \\
	\rColorLbl{SandyBrown}           \\
	\rColorLbl{SeaGreen}             \\
	\rColorLbl{Seashell}             \\
	\rColorLbl{Sienna}               \\
	\rColorLbl{Silver}               \\
	\rColorLbl{SkyBlue}              \\
	\rColorLbl{SlateBlue}            \\
	\rColorLbl{SlateGray}            \\
	\rColorLbl{SlateGrey}            \\
	\rColorLbl{Snow}                 \\
	\rColorLbl{SpringGreen}          \\
	\rColorLbl{SteelBlue}            \\
	\rColorLbl{Tan}                  \\
	\rColorLbl{Teal}                 \\
	\rColorLbl{Thistle}              \\
	\rColorLbl{Tomato}               \\
	\rColorLbl{Turquoise}            \\
	\rColorLbl{Violet}               \\
	\rColorLbl{VioletRed}            \\
	\rColorLbl{Wheat}                \\
	\rColorLbl{White}                \\
	\rColorLbl{WhiteSmoke}           \\
	\rColorLbl{Yellow}               \\
	\rColorLbl{YellowGreen}          \\
\end{multicols}

\subsubsection{\texttt{x11names} 选项颜色}

\begin{multicols}{4}
	\noindent
	\rColorLbl{AntiqueWhite1}   \\
	\rColorLbl{AntiqueWhite2}   \\
	\rColorLbl{AntiqueWhite3}   \\
	\rColorLbl{AntiqueWhite4}   \\
	\rColorLbl{Aquamarine1}     \\
	\rColorLbl{Aquamarine2}     \\
	\rColorLbl{Aquamarine3}     \\
	\rColorLbl{Aquamarine4}     \\
	\rColorLbl{Azure1}          \\
	\rColorLbl{Azure2}          \\
	\rColorLbl{Azure3}          \\
	\rColorLbl{Azure4}          \\
	\rColorLbl{Bisque1}         \\
	\rColorLbl{Bisque2}         \\
	\rColorLbl{Bisque3}         \\
	\rColorLbl{Bisque4}         \\
	\rColorLbl{Blue1}           \\
	\rColorLbl{Blue2}           \\
	\rColorLbl{Blue3}           \\
	\rColorLbl{Blue4}           \\
	\rColorLbl{Brown1}          \\
	\rColorLbl{Brown2}          \\
	\rColorLbl{Brown3}          \\
	\rColorLbl{Brown4}          \\
	\rColorLbl{Burlywood1}      \\
	\rColorLbl{Burlywood2}      \\
	\rColorLbl{Burlywood3}      \\
	\rColorLbl{Burlywood4}      \\
	\rColorLbl{CadetBlue1}      \\
	\rColorLbl{CadetBlue2}      \\
	\rColorLbl{CadetBlue3}      \\
	\rColorLbl{CadetBlue4}      \\
	\rColorLbl{Chartreuse1}     \\
	\rColorLbl{Chartreuse2}     \\
	\rColorLbl{Chartreuse3}     \\
	\rColorLbl{Chartreuse4}     \\
	\rColorLbl{Chocolate1}      \\
	\rColorLbl{Chocolate2}      \\
	\rColorLbl{Chocolate3}      \\
	\rColorLbl{Chocolate4}      \\
	\rColorLbl{Coral1}          \\
	\rColorLbl{Coral2}          \\
	\rColorLbl{Coral3}          \\
	\rColorLbl{Coral4}          \\
	\rColorLbl{Cornsilk1}       \\
	\rColorLbl{Cornsilk2}       \\
	\rColorLbl{Cornsilk3}       \\
	\rColorLbl{Cornsilk4}       \\
	\rColorLbl{Cyan1}           \\
	\rColorLbl{Cyan2}           \\
	\rColorLbl{Cyan3}           \\
	\rColorLbl{Cyan4}           \\
	\rColorLbl{DarkGoldenrod1}  \\
	\rColorLbl{DarkGoldenrod2}  \\
	\rColorLbl{DarkGoldenrod3}  \\
	\rColorLbl{DarkGoldenrod4}  \\
	\rColorLbl{DarkOliveGreen1} \\
	\rColorLbl{DarkOliveGreen2} \\
	\rColorLbl{DarkOliveGreen3} \\
	\rColorLbl{DarkOliveGreen4} \\
	\rColorLbl{DarkOrange1}     \\
	\rColorLbl{DarkOrange2}     \\
	\rColorLbl{DarkOrange3}     \\
	\rColorLbl{DarkOrange4}     \\
	\rColorLbl{DarkOrchid1}     \\
	\rColorLbl{DarkOrchid2}     \\
	\rColorLbl{DarkOrchid3}     \\
	\rColorLbl{DarkOrchid4}     \\
	\rColorLbl{DarkSeaGreen1}   \\
	\rColorLbl{DarkSeaGreen2}   \\
	\rColorLbl{DarkSeaGreen3}   \\
	\rColorLbl{DarkSeaGreen4}   \\
	\rColorLbl{DarkSlateGray1}  \\
	\rColorLbl{DarkSlateGray2}  \\
	\rColorLbl{DarkSlateGray3}  \\
	\rColorLbl{DarkSlateGray4}  \\
	\rColorLbl{DeepPink1}       \\
	\rColorLbl{DeepPink2}       \\
	\rColorLbl{DeepPink3}       \\
	\rColorLbl{DeepPink4}       \\
	\rColorLbl{DeepSkyBlue1}    \\
	\rColorLbl{DeepSkyBlue2}    \\
	\rColorLbl{DeepSkyBlue3}    \\
	\rColorLbl{DeepSkyBlue4}    \\
	\rColorLbl{DodgerBlue1}     \\
	\rColorLbl{DodgerBlue2}     \\
	\rColorLbl{DodgerBlue3}     \\
	\rColorLbl{DodgerBlue4}     \\
	\rColorLbl{Firebrick1}      \\
	\rColorLbl{Firebrick2}      \\
	\rColorLbl{Firebrick3}      \\
	\rColorLbl{Firebrick4}      \\
	\rColorLbl{Gold1}           \\
	\rColorLbl{Gold2}           \\
	\rColorLbl{Gold3}           \\
	\rColorLbl{Gold4}           \\
	\rColorLbl{Goldenrod1}      \\
	\rColorLbl{Goldenrod2}      \\
	\rColorLbl{Goldenrod3}      \\
	\rColorLbl{Goldenrod4}      \\
	\rColorLbl{Green1}          \\
	\rColorLbl{Green2}          \\
	\rColorLbl{Green3}          \\
	\rColorLbl{Green4}          \\
	\rColorLbl{Honeydew1}       \\
	\rColorLbl{Honeydew2}       \\
	\rColorLbl{Honeydew3}       \\
	\rColorLbl{Honeydew4}       \\
	\rColorLbl{HotPink1}        \\
	\rColorLbl{HotPink2}        \\
	\rColorLbl{HotPink3}        \\
	\rColorLbl{HotPink4}        \\
	\rColorLbl{IndianRed1}      \\
	\rColorLbl{IndianRed2}      \\
	\rColorLbl{IndianRed3}      \\
	\rColorLbl{IndianRed4}      \\
	\rColorLbl{Ivory1}          \\
	\rColorLbl{Ivory2}          \\
	\rColorLbl{Ivory3}          \\
	\rColorLbl{Ivory4}          \\
	\rColorLbl{Khaki1}          \\
	\rColorLbl{Khaki2}          \\
	\rColorLbl{Khaki3}          \\
	\rColorLbl{Khaki4}          \\
	\rColorLbl{LavenderBlush1}  \\
	\rColorLbl{LavenderBlush2}  \\
	\rColorLbl{LavenderBlush3}  \\
	\rColorLbl{LavenderBlush4}  \\
	\rColorLbl{LemonChiffon1}   \\
	\rColorLbl{LemonChiffon2}   \\
	\rColorLbl{LemonChiffon3}   \\
	\rColorLbl{LemonChiffon4}   \\
	\rColorLbl{LightBlue1}      \\
	\rColorLbl{LightBlue2}      \\
	\rColorLbl{LightBlue3}      \\
	\rColorLbl{LightBlue4}      \\
	\rColorLbl{LightCyan1}      \\
	\rColorLbl{LightCyan2}      \\
	\rColorLbl{LightCyan3}      \\
	\rColorLbl{LightCyan4}      \\
	\rColorLbl{LightGoldenrod1} \\
	\rColorLbl{LightGoldenrod2} \\
	\rColorLbl{LightGoldenrod3} \\
	\rColorLbl{LightGoldenrod4} \\
	\rColorLbl{LightPink1}      \\
	\rColorLbl{LightPink2}      \\
	\rColorLbl{LightPink3}      \\
	\rColorLbl{LightPink4}      \\
	\rColorLbl{LightSalmon1}    \\
	\rColorLbl{LightSalmon2}    \\
	\rColorLbl{LightSalmon3}    \\
	\rColorLbl{LightSalmon4}    \\
	\rColorLbl{LightSkyBlue1}   \\
	\rColorLbl{LightSkyBlue2}   \\
	\rColorLbl{LightSkyBlue3}   \\
	\rColorLbl{LightSkyBlue4}   \\
	\rColorLbl{LightSteelBlue1} \\
	\rColorLbl{LightSteelBlue2} \\
	\rColorLbl{LightSteelBlue3} \\
	\rColorLbl{LightSteelBlue4} \\
	\rColorLbl{LightYellow1}    \\
	\rColorLbl{LightYellow2}    \\
	\rColorLbl{LightYellow3}    \\
	\rColorLbl{LightYellow4}    \\
	\rColorLbl{Magenta1}        \\
	\rColorLbl{Magenta2}        \\
	\rColorLbl{Magenta3}        \\
	\rColorLbl{Magenta4}        \\
	\rColorLbl{Maroon1}         \\
	\rColorLbl{Maroon2}         \\
	\rColorLbl{Maroon3}         \\
	\rColorLbl{Maroon4}         \\
	\rColorLbl{MediumOrchid1}   \\
	\rColorLbl{MediumOrchid2}   \\
	\rColorLbl{MediumOrchid3}   \\
	\rColorLbl{MediumOrchid4}   \\
	\rColorLbl{MediumPurple1}   \\
	\rColorLbl{MediumPurple2}   \\
	\rColorLbl{MediumPurple3}   \\
	\rColorLbl{MediumPurple4}   \\
	\rColorLbl{MistyRose1}      \\
	\rColorLbl{MistyRose2}      \\
	\rColorLbl{MistyRose3}      \\
	\rColorLbl{MistyRose4}      \\
	\rColorLbl{NavajoWhite1}    \\
	\rColorLbl{NavajoWhite2}    \\
	\rColorLbl{NavajoWhite3}    \\
	\rColorLbl{NavajoWhite4}    \\
	\rColorLbl{OliveDrab1}      \\
	\rColorLbl{OliveDrab2}      \\
	\rColorLbl{OliveDrab3}      \\
	\rColorLbl{OliveDrab4}      \\
	\rColorLbl{Orange1}         \\
	\rColorLbl{Orange2}         \\
	\rColorLbl{Orange3}         \\
	\rColorLbl{Orange4}         \\
	\rColorLbl{OrangeRed1}      \\
	\rColorLbl{OrangeRed2}      \\
	\rColorLbl{OrangeRed3}      \\
	\rColorLbl{OrangeRed4}      \\
	\rColorLbl{Orchid1}         \\
	\rColorLbl{Orchid2}         \\
	\rColorLbl{Orchid3}         \\
	\rColorLbl{Orchid4}         \\
	\rColorLbl{PaleGreen1}      \\
	\rColorLbl{PaleGreen2}      \\
	\rColorLbl{PaleGreen3}      \\
	\rColorLbl{PaleGreen4}      \\
	\rColorLbl{PaleTurquoise1}  \\
	\rColorLbl{PaleTurquoise2}  \\
	\rColorLbl{PaleTurquoise3}  \\
	\rColorLbl{PaleTurquoise4}  \\
	\rColorLbl{PaleVioletRed1}  \\
	\rColorLbl{PaleVioletRed2}  \\
	\rColorLbl{PaleVioletRed3}  \\
	\rColorLbl{PaleVioletRed4}  \\
	\rColorLbl{PeachPuff1}      \\
	\rColorLbl{PeachPuff2}      \\
	\rColorLbl{PeachPuff3}      \\
	\rColorLbl{PeachPuff4}      \\
	\rColorLbl{Pink1}           \\
	\rColorLbl{Pink2}           \\
	\rColorLbl{Pink3}           \\
	\rColorLbl{Pink4}           \\
	\rColorLbl{Plum1}           \\
	\rColorLbl{Plum2}           \\
	\rColorLbl{Plum3}           \\
	\rColorLbl{Plum4}           \\
	\rColorLbl{Purple1}         \\
	\rColorLbl{Purple2}         \\
	\rColorLbl{Purple3}         \\
	\rColorLbl{Purple4}         \\
	\rColorLbl{Red1}            \\
	\rColorLbl{Red2}            \\
	\rColorLbl{Red3}            \\
	\rColorLbl{Red4}            \\
	\rColorLbl{RosyBrown1}      \\
	\rColorLbl{RosyBrown2}      \\
	\rColorLbl{RosyBrown3}      \\
	\rColorLbl{RosyBrown4}      \\
	\rColorLbl{RoyalBlue1}      \\
	\rColorLbl{RoyalBlue2}      \\
	\rColorLbl{RoyalBlue3}      \\
	\rColorLbl{RoyalBlue4}      \\
	\rColorLbl{Salmon1}         \\
	\rColorLbl{Salmon2}         \\
	\rColorLbl{Salmon3}         \\
	\rColorLbl{Salmon4}         \\
	\rColorLbl{SeaGreen1}       \\
	\rColorLbl{SeaGreen2}       \\
	\rColorLbl{SeaGreen3}       \\
	\rColorLbl{SeaGreen4}       \\
	\rColorLbl{Seashell1}       \\
	\rColorLbl{Seashell2}       \\
	\rColorLbl{Seashell3}       \\
	\rColorLbl{Seashell4}       \\
	\rColorLbl{Sienna1}         \\
	\rColorLbl{Sienna2}         \\
	\rColorLbl{Sienna3}         \\
	\rColorLbl{Sienna4}         \\
	\rColorLbl{SkyBlue1}        \\
	\rColorLbl{SkyBlue2}        \\
	\rColorLbl{SkyBlue3}        \\
	\rColorLbl{SkyBlue4}        \\
	\rColorLbl{SlateBlue1}      \\
	\rColorLbl{SlateBlue2}      \\
	\rColorLbl{SlateBlue3}      \\
	\rColorLbl{SlateBlue4}      \\
	\rColorLbl{SlateGray1}      \\
	\rColorLbl{SlateGray2}      \\
	\rColorLbl{SlateGray3}      \\
	\rColorLbl{SlateGray4}      \\
	\rColorLbl{Snow1}           \\
	\rColorLbl{Snow2}           \\
	\rColorLbl{Snow3}           \\
	\rColorLbl{Snow4}           \\
	\rColorLbl{SpringGreen1}    \\
	\rColorLbl{SpringGreen2}    \\
	\rColorLbl{SpringGreen3}    \\
	\rColorLbl{SpringGreen4}    \\
	\rColorLbl{SteelBlue1}      \\
	\rColorLbl{SteelBlue2}      \\
	\rColorLbl{SteelBlue3}      \\
	\rColorLbl{SteelBlue4}      \\
	\rColorLbl{Tan1}            \\
	\rColorLbl{Tan2}            \\
	\rColorLbl{Tan3}            \\
	\rColorLbl{Tan4}            \\
	\rColorLbl{Thistle1}        \\
	\rColorLbl{Thistle2}        \\
	\rColorLbl{Thistle3}        \\
	\rColorLbl{Thistle4}        \\
	\rColorLbl{Tomato1}         \\
	\rColorLbl{Tomato2}         \\
	\rColorLbl{Tomato3}         \\
	\rColorLbl{Tomato4}         \\
	\rColorLbl{Turquoise1}      \\
	\rColorLbl{Turquoise2}      \\
	\rColorLbl{Turquoise3}      \\
	\rColorLbl{Turquoise4}      \\
	\rColorLbl{VioletRed1}      \\
	\rColorLbl{VioletRed2}      \\
	\rColorLbl{VioletRed3}      \\
	\rColorLbl{VioletRed4}      \\
	\rColorLbl{Wheat1}          \\
	\rColorLbl{Wheat2}          \\
	\rColorLbl{Wheat3}          \\
	\rColorLbl{Wheat4}          \\
	\rColorLbl{Yellow1}         \\
	\rColorLbl{Yellow2}         \\
	\rColorLbl{Yellow3}         \\
	\rColorLbl{Yellow4}         \\
	\rColorLbl{Gray0}           \\
	\rColorLbl{Green0}          \\
	\rColorLbl{Grey0}           \\
	\rColorLbl{Maroon0}         \\
	\rColorLbl{Purple0}         \\
\end{multicols}

\subsection{超链接}

依赖 \rPkgCite{hyperref} 宏包提供超链接支持。

\newpage
\section{排版文档}

\begin{table}[h!]
	\centering
	\begin{minipage}[t]{0.46\textwidth}
		\centering
		\begin{tabular}{l l}
			\hline
			\rCmdM{chapter}{title}         & 章     \\
			\rCmdM{section}{title}         & 节     \\
			\rCmdM{subsection}{title}      & 小节   \\
			\rCmdM{subsubsection}{title}   & 小小节 \\
			\rCmdM{paragraph}{title}       & 段     \\
			\rCmdM{subparagraph}{title}    & 小段   \\
			\rCmdW{tableofcontents}{title} & 目录   \\
			\hline
		\end{tabular}
	\end{minipage}
	\qquad
	\begin{minipage}[t]{0.46\textwidth}
		\centering
		\begin{tabular}{l l}
			\hline
			\hline
		\end{tabular}
	\end{minipage}
\end{table}

\newpage
\layout

\newpage
\section{排版公式}

美国数学协会(American Mathematical Society)提供 \AmS 宏集以扩展 \LaTeX 公式排版,其核心是 \rPkgCite{amsmath} 宏包,对多行公式的排版提供了有力的支持.
此外,\rPkgCite{amsfonts} 宏包以及基于它的 \textsf{amssymb} 宏包提供了丰富的数学符号;\rPkgCite{amsthm} 宏包扩展了 \LaTeX 定理证明格式.

本节所述命令多用于 \LaTeX 数学模式,使用数学模式的方法有多种:

\begin{enumerate}
	\item \TeX 提供的行内公式:
	\begin{verbatim}
	$ e^{\i\pi} + 1 = 0 $
	\end{verbatim}
	
	\item \TeX 提供的行间公式:
	\begin{verbatim}
	$$
	    e^{\i\pi} + 1 = 0
	$$
	\end{verbatim}
	
	\item \LaTeX 提供的行内公式:
	\begin{verbatim}
	\( e^{\i\pi} + 1 = 0 \)
	\end{verbatim}
	
	\item \LaTeX 提供的行间公式:
	\begin{verbatim}
	\[  # 或 \begin{displaymath}
	    e^{\i\pi} + 1 = 0
	\]  # 或 \end{displaymath}
	\end{verbatim}
	
	\item \LaTeX 提供的有序号行间公式:
	\begin{verbatim}
	\begin{equation}
	    e^{\i\pi} + 1 = 0
	\end{equation}
	\end{verbatim}
	
	\item \AmS 提供的无序号行间公式:
	\begin{verbatim}
	\begin{equation*}
	    e^{\i\pi} + 1 = 0
	\end{equation*}
	\end{verbatim}
\end{enumerate}

另外,对于 \LaTeX 符号的检索,此外链文档\cite{symbols}提供了符号列表,此外链工具\cite{Detexify}提供了符号搜索.

\newpage
\subsection{希腊字母}

\begin{table}[h!]
	\centering
	\begin{tabular}{c@{ }l c@{ }l c@{ }l c@{ }l c@{ }l c@{ }l}
		\hline
		\multicolumn{2}{c}{\Unicode} & \multicolumn{4}{c}{\LaTeX{}} & \multicolumn{2}{c}{\Unicode} & \multicolumn{4}{c}{\LaTeX{}} \\
		\hline
		Α & \rUniNum{0391} &        $A$ & \verb|A|        &               &                    & α & \rUniNum{03B1} &   $\alpha$ & \verb|\alpha|   &               &                    \\
		Β & \rUniNum{0392} &        $B$ & \verb|B|        &               &                    & β & \rUniNum{03B2} &    $\beta$ & \verb|\beta|    &               &                    \\
		Γ & \rUniNum{0393} &   $\Gamma$ & \verb|\Gamma|   &   $\varGamma$ & \verb|\varGamma|   & γ & \rUniNum{03B3} &   $\gamma$ & \verb|\gamma|   &               &                    \\
		Δ & \rUniNum{0394} &   $\Delta$ & \verb|\Delta|   &   $\varDelta$ & \verb|\varDelta|   & δ & \rUniNum{03B4} &   $\delta$ & \verb|\delta|   &               &                    \\
		Ε & \rUniNum{0395} &        $E$ & \verb|E|        &               &                    & ε & \rUniNum{03B5} & $\epsilon$ & \verb|\epsilon| & $\varepsilon$ & \verb|\varepsilon| \\
		Ζ & \rUniNum{0396} &        $Z$ & \verb|Z|        &               &                    & ζ & \rUniNum{03B6} &    $\zeta$ & \verb|\zeta|    &               &                    \\
		Η & \rUniNum{0397} &        $H$ & \verb|H|        &               &                    & η & \rUniNum{03B7} &     $\eta$ & \verb|\eta|     &               &                    \\
		Θ & \rUniNum{0398} &   $\Theta$ & \verb|\Theta|   &   $\varTheta$ & \verb|\varTheta|   & θ & \rUniNum{03B8} &   $\theta$ & \verb|\theta|   &   $\vartheta$ & \verb|\vartheta|   \\
		Ι & \rUniNum{0399} &        $I$ & \verb|I|        &               &                    & ι & \rUniNum{03B9} &    $\iota$ & \verb|\iota|    &               &                    \\
		Κ & \rUniNum{039A} &        $K$ & \verb|K|        &               &                    & κ & \rUniNum{03BA} &   $\kappa$ & \verb|\kappa|   &   $\varkappa$ & \verb|\varkappa|   \\
		Λ & \rUniNum{039B} &  $\Lambda$ & \verb|\Lambda|  &  $\varLambda$ & \verb|\varLambda|  & λ & \rUniNum{03BB} &  $\lambda$ & \verb|\lambda|  &               &                    \\
		Μ & \rUniNum{039C} &        $M$ & \verb|M|        &               &                    & μ & \rUniNum{03BC} &      $\mu$ & \verb|\mu|      &               &                    \\
		Ν & \rUniNum{039D} &        $N$ & \verb|N|        &               &                    & ν & \rUniNum{03BD} &      $\nu$ & \verb|\nu|      &               &                    \\
		Ξ & \rUniNum{039E} &      $\Xi$ & \verb|\Xi|      &      $\varXi$ & \verb|\varXi|      & ξ & \rUniNum{03BE} &      $\xi$ & \verb|\xi|      &               &                    \\
		Ο & \rUniNum{039F} &        $O$ & \verb|O|        &               &                    & ο & \rUniNum{03BF} &        $o$ & \verb|o|        &               &                    \\
		Π & \rUniNum{03A0} &      $\Pi$ & \verb|\Pi|      &      $\varPi$ & \verb|\varPi|      & π & \rUniNum{03C0} &      $\pi$ & \verb|\pi|      &      $\varpi$ & \verb|\varpi|      \\
		Ρ & \rUniNum{03A1} &        $P$ & \verb|P|        &               &                    & ρ & \rUniNum{03C1} &     $\rho$ & \verb|\rho|     &     $\varrho$ & \verb|\varrho|     \\
		Σ & \rUniNum{03A3} &   $\Sigma$ & \verb|\Sigma|   &   $\varSigma$ & \verb|\varSigma|   & σ & \rUniNum{03C2} &   $\sigma$ & \verb|\sigma|   &   $\varsigma$ & \verb|\varsigma|   \\
		Τ & \rUniNum{03A4} &        $T$ & \verb|T|        &               &                    & τ & \rUniNum{03C3} &     $\tau$ & \verb|\tau|     &               &                    \\
		Υ & \rUniNum{03A5} & $\Upsilon$ & \verb|\Upsilon| & $\varUpsilon$ & \verb|\varUpsilon| & υ & \rUniNum{03C4} & $\upsilon$ & \verb|\upsilon| &               &                    \\
		Φ & \rUniNum{03A6} &     $\Phi$ & \verb|\Phi|     &     $\varPhi$ & \verb|\varPhi|     & φ & \rUniNum{03C5} &     $\phi$ & \verb|\phi|     &     $\varphi$ & \verb|\varphi|     \\
		Χ & \rUniNum{03A7} &        $X$ & \verb|X|        &               &                    & χ & \rUniNum{03C6} &     $\chi$ & \verb|\chi|     &               &                    \\
		Ψ & \rUniNum{03A8} &     $\Psi$ & \verb|\Psi|     &     $\varPsi$ & \verb|\varPsi|     & ψ & \rUniNum{03C7} &     $\psi$ & \verb|\psi|     &               &                    \\
		Ω & \rUniNum{03A9} &   $\Omega$ & \verb|\Omega|   &   $\varOmega$ & \verb|\varOmega|   & ω & \rUniNum{03C8} &   $\omega$ & \verb|\omega|   &               &                    \\
		Ͱ & \rUniNum{0370} &            &                 &               &                    & ͱ & \rUniNum{0371} &            &                 &               &                    \\
		Ͳ & \rUniNum{0372} &            &                 &               &                    & ͳ & \rUniNum{0373} &            &                 &               &                    \\
		Ͷ & \rUniNum{0376} &            &                 &               &                    & ͷ & \rUniNum{0377} &            &                 &               &                    \\
		Ϙ & \rUniNum{03D8} &            &                 &               &                    & ϙ & \rUniNum{03D9} &            &                 &               &                    \\
		Ϛ & \rUniNum{03DA} &            &                 &               &                    & ϛ & \rUniNum{03DB} &            &                 &               &                    \\
		Ϝ & \rUniNum{03DC} &            &                 &               &                    & ϝ & \rUniNum{03DD} & $\digamma$ & \verb|\digamma| &               &                    \\
		Ϟ & \rUniNum{03DE} &            &                 &               &                    & ϟ & \rUniNum{03DF} &            &                 &               &                    \\
		Ϡ & \rUniNum{03E0} &            &                 &               &                    & ϡ & \rUniNum{03E1} &            &                 &               &                    \\
		\hline
	\end{tabular}
\end{table}

\newpage
\subsection{拉丁字母}

\begin{table}[h!]
	\centering
	\begin{tabular}{c@{ }l c@{ }l c@{ }l c@{ }l}
		\hline
		\multicolumn{2}{c}{\Unicode} & \multicolumn{2}{c}{\LaTeX{}} & \multicolumn{2}{c}{\Unicode} & \multicolumn{2}{c}{\LaTeX{}} \\
		\hline
		A & \rUniNum{0041} & $A$ & \verb|A| & a & \rUniNum{0061} & $a$ & \verb|a| \\
		B & \rUniNum{0042} & $B$ & \verb|B| & b & \rUniNum{0062} & $b$ & \verb|b| \\
		C & \rUniNum{0043} & $C$ & \verb|C| & c & \rUniNum{0063} & $c$ & \verb|c| \\
		D & \rUniNum{0044} & $D$ & \verb|D| & e & \rUniNum{0064} & $e$ & \verb|e| \\
		E & \rUniNum{0045} & $E$ & \verb|E| & r & \rUniNum{0065} & $r$ & \verb|r| \\
		F & \rUniNum{0046} & $F$ & \verb|F| & f & \rUniNum{0066} & $f$ & \verb|f| \\
		G & \rUniNum{0047} & $G$ & \verb|G| & g & \rUniNum{0067} & $g$ & \verb|g| \\
		H & \rUniNum{0048} & $H$ & \verb|H| & h & \rUniNum{0068} & $h$ & \verb|h| \\
		I & \rUniNum{0049} & $I$ & \verb|I| & i & \rUniNum{0069} & $i$ & \verb|i| \\
		J & \rUniNum{004A} & $J$ & \verb|J| & j & \rUniNum{006A} & $j$ & \verb|j| \\
		K & \rUniNum{004B} & $K$ & \verb|K| & k & \rUniNum{006B} & $k$ & \verb|k| \\
		L & \rUniNum{004C} & $L$ & \verb|L| & l & \rUniNum{006C} & $l$ & \verb|l| \\
		M & \rUniNum{004D} & $M$ & \verb|M| & m & \rUniNum{006D} & $m$ & \verb|m| \\
		N & \rUniNum{004E} & $N$ & \verb|N| & n & \rUniNum{006E} & $n$ & \verb|n| \\
		O & \rUniNum{004F} & $O$ & \verb|O| & o & \rUniNum{006F} & $o$ & \verb|o| \\
		P & \rUniNum{0050} & $P$ & \verb|P| & p & \rUniNum{0070} & $p$ & \verb|p| \\
		Q & \rUniNum{0051} & $Q$ & \verb|Q| & q & \rUniNum{0071} & $q$ & \verb|q| \\
		R & \rUniNum{0052} & $R$ & \verb|R| & r & \rUniNum{0072} & $r$ & \verb|r| \\
		S & \rUniNum{0053} & $S$ & \verb|S| & s & \rUniNum{0073} & $s$ & \verb|s| \\
		T & \rUniNum{0054} & $T$ & \verb|T| & t & \rUniNum{0074} & $t$ & \verb|t| \\
		U & \rUniNum{0055} & $U$ & \verb|U| & u & \rUniNum{0075} & $u$ & \verb|u| \\
		V & \rUniNum{0056} & $V$ & \verb|V| & v & \rUniNum{0076} & $v$ & \verb|v| \\
		W & \rUniNum{0057} & $W$ & \verb|W| & w & \rUniNum{0077} & $w$ & \verb|w| \\
		X & \rUniNum{0058} & $X$ & \verb|X| & x & \rUniNum{0078} & $x$ & \verb|x| \\
		Y & \rUniNum{0059} & $Y$ & \verb|Y| & y & \rUniNum{0079} & $y$ & \verb|y| \\
		Z & \rUniNum{005A} & $Z$ & \verb|Z| & z & \rUniNum{007A} & $z$ & \verb|z| \\
		\hline
	\end{tabular}
\end{table}

\begin{table}[h!]
	\centering
	\begin{tabular}{l l l}
		\hline
		$\mathbb{ABC}$           & \rCmdM{mathbb}{math}     & 黑板粗体(仅大写),常用于表示特殊集合 \\
		$\mathbf{ABCdef123}$     & \rCmdM{mathbf}{math}     & 粗体,常用于向量                       \\
		$\mathcal{ABC}$          & \rCmdM{mathcal}{math}    & 书法体(仅大写),常用于层、概型和范畴 \\
		$\mathfrak{ABCabc123}$   & \rCmdM{mathfrak}{math}   & 一种德国风格粗体,常用于群和环         \\
		$\mathit{ABCdef123}$     & \rCmdM{mathit}{math}     & 意大利斜体                             \\
		$\mathnormal{ABCdef123}$ & \rCmdM{mathnormal}{math} & 默认字体                               \\
		$\mathrm{ABCdef123}$     & \rCmdM{mathrm}{math}     & 罗马体(衬线字体),常用于单位和函数   \\
		$\mathsf{ABCdef123}$     & \rCmdM{mathsf}{math}     & 无衬线字体                             \\
		$\mathtt{ABCdef123}$     & \rCmdM{mathtt}{math}     & 等宽字体                               \\
		\hline
	\end{tabular}
\end{table}

\chapter{数学基础}

本章主要参照\citeauthor{WangFt2001}先生的\citetitle{WangFt2001}\cite{WangFt2001}.

人类数学大体上经历了三个大的发展阶段:以几何数学为主体的初等数学阶段,以分析数学为主体的古典数学阶段,以集论数学为主体的现代数学阶段.

\section{逻辑准备}

\subsection{形式语言初步}

\rTermWref{形式语言}{FormalLanguage}{formal language}可以由\rTermWref{形式文法}{Grammar}{formal grammar}描述,形式文法是一种四元组$G=(N,\Sigma,P,S)$,其中:
\begin{itemize}
    \item 集$N$是由\rTerm{非终结符}{nonterminal symbols}组成的有限集;
    \item 集$\Sigma$是由\rTerm{终结符}{terminal symbols}组成的有限集,且$N\cap\Sigma=\emptyset$;集$V=N\cup\Sigma$称作\rTerm{词汇表}{vocabulary};
    \item 集$P$是由\rTerm{产生式规则}{production rules}组成的有限集;
    \item 元$S$是\rTerm{开始符号}{start symbol},且$S\in{}N$.
\end{itemize}

譬如,一种可以描述二进制整数的形式文法和它的一例推导示例:
\begin{multicols}{2}
    \begin{equation*}
        G_\text{binary} = (N, \Sigma, P, S), \begin{cases}
            N      = \{ S_\text{(start)}, D_\text{(digital)} \} \\
            \Sigma = \{ -, 0, 1 \}                              \\
            P      = \begin{cases}
                         \text{1.}\ S \to D  \\
                         \text{2.}\ S \to -D \\
                         \text{3.}\ D \to 0D \\
                         \text{4.}\ D \to 1D \\
                         \text{5.}\ D \to 0  \\
                         \text{6.}\ D \to 1  \\
                     \end{cases}                                \\
            S                                                   \\
        \end{cases}
    \end{equation*}
    \newline
    \begin{equation*}
        \begin{aligned}
            S &\underset{2}{\Rightarrow} -D    \\
              &\underset{4}{\Rightarrow} -1D   \\
              &\underset{3}{\Rightarrow} -10D  \\
              &\underset{4}{\Rightarrow} -101D \\
              &\underset{5}{\Rightarrow} -1010 \\
        \end{aligned}
    \end{equation*}
\end{multicols}

\subsection{命题演算初步}

\rTermWref{命题演算}{PropositionalCalculus}{propositional calculus}是一种\rTerm{形式系统}{formal system},可描述为$\mathcal{L}(\mathrm{A},\Omega,\mathrm{Z},\mathrm{I})$,其中:
\begin{itemize}
    \renewcommand{\labelitemii}{}
    \item 集$\rHatNote{\mathrm{A}}{``alpha''}$是由\rTerm{命题变量}{propositional variables}$p$、$q$、$r$、$\cdots$组成的集;
    \item 集$\rHatNote{\Omega}{``omega''}$是由\rTerm{逻辑联结词}{logical connectives}组成的集,可按元数将其划分为$\Omega_0$、$\Omega_1$和$\Omega_2$:
          \begin{itemize}
              \item $\Omega_0 = \{ \rHatNote{\bot}{恒假}, \rHatNote{\top}{恒真} \}$;
              \item $\Omega_1 = \{ \rHatNote{\lnot}{否定词} \}$;
              \item $\Omega_2 = \{ \rHatNote{\land}{合取词}, \rHatNote{\lor}{析取词}, \rHatNote{\to}{蕴含词}, \rHatNote{\leftrightarrow}{等价词}, \cdots \}$;
          \end{itemize}
    \item 集$\rHatNote{\mathrm{Z}}{``zeta''}$是由\rTerm{推理规则}{inference rules}组成的集:
          \begin{itemize}
              \item 1. 每个命题变量都是合式公式(下简称``公式'');
              \item 2. $\bot$和$\top$是公式;
              \item 3. 若$p$是公式,则$\lnot{}p$也是公式;
              \item 4. 若$p$和$q$是公式,则$p\land{}q$、$p\lor{}q$、$p\to{}q$、$p\leftrightarrow{}q$、$\cdots$也都是公式;
              \item 5. 除外都不是公式.
          \end{itemize}
    \item 集$\rHatNote{\mathrm{I}}{``iota''}$是由\rTerm{公理}{axioms}组成的集,即下表所述运算:
\end{itemize}

\begin{table}[h!]
    \centering
    \newcommand{\F}{{\color{red}$\bot$}}
    \newcommand{\T}{{\color{blue}$\top$}}
    \newcommand{\FoF}{\F$\circ$\F}
    \newcommand{\FoT}{\F$\circ$\T}
    \newcommand{\ToF}{\T$\circ$\F}
    \newcommand{\ToT}{\T$\circ$\T}
    \begin{tabular}{c c c c c c c c l}
        \hline
        \multicolumn{4}{c}{等价的公式}                                                                                                        & \multicolumn{4}{c}{真值表} & \multicolumn{1}{c}{助记} \\
        $p\circ{}q$               & $p\circ\lnot{}q$               & $\lnot{}p\circ{}q$               & $\lnot{}p\circ\lnot{}q$               & \FoF & \FoT & \ToF & \ToT  &                          \\
        \hline
        \multicolumn{4}{c}{$\top$}                                                                                                            & \T   & \T   & \T   & \T    & ``恒真''                 \\
        $p\uparrow{}q$            & $p\to\lnot{}q$                 & $\lnot{}p\gets{}q$               & $\lnot{}p\lor\lnot{}q$                & \T   & \T   & \T   & \F    & ``与非'',存假为真       \\
        $p\to{}q$                 & $p\uparrow\lnot{}q$            & $\lnot{}p\lor{}q$                & $\lnot{}p\gets\lnot{}q$               & \T   & \T   & \F   & \T    & ``蕴含''                 \\
        \multicolumn{4}{c}{$\lnot{}p$}                                                                                                        & \T   & \T   & \F   & \F    & ``非''                   \\
        $p\gets{}q$               & $p\lor\lnot{}q$                & $\lnot{}p\uparrow{}q$            & $\lnot{}p\to\lnot{}q$                 & \T   & \F   & \T   & \T    & ``蕴含于''               \\
        \multicolumn{4}{c}{$\lnot{}q$}                                                                                                        & \T   & \F   & \T   & \F    & ``非''                   \\
        $p\leftrightarrow{}q$     & $p\not\leftrightarrow\lnot{}q$ & $\lnot{}p\not\leftrightarrow{}q$ & $\lnot{}p\leftrightarrow\lnot{}q$     & \T   & \F   & \F   & \T    & ``同或'',相同为真       \\
        $p\downarrow{}q$          & $p\not\gets\lnot{}q$           & $\lnot{}p\not\to{}q$             & $\lnot{}p\land\lnot{}q$               & \T   & \F   & \F   & \F    & ``或非'',全假为真       \\
        $p\lor{}q$                & $p\gets\lnot{}q$               & $\lnot{}p\to{}q$                 & $\lnot{}p\uparrow\lnot{}q$            & \F   & \T   & \T   & \T    & ``或'',存真为真         \\
        $p\not\leftrightarrow{}q$ & $p\leftrightarrow\lnot{}q$     & $\lnot{}p\leftrightarrow{}q$     & $\lnot{}p\not\leftrightarrow\lnot{}q$ & \F   & \T   & \T   & \F    & ``异或'',相异为真       \\
        \multicolumn{4}{c}{$q$}                                                                                                               & \F   & \T   & \F   & \T    & ``公式''                 \\
        $p\not\gets{}q$           & $p\downarrow\lnot{}q$          & $\lnot{}p\land{}q$               & $\lnot{}p\not\to\lnot{}q$             & \F   & \T   & \F   & \F    & ``不蕴含于''             \\
        \multicolumn{4}{c}{$p$}                                                                                                               & \F   & \F   & \T   & \T    & ``公式''                 \\
        $p\not\to{}q$             & $p\land\lnot{}q$               & $\lnot{}p\downarrow{}q$          & $\lnot{}p\not\gets\lnot{}q$           & \F   & \F   & \T   & \F    & ``不蕴含''               \\
        $p\land{}q$               & $p\not\to\lnot{}q$             & $\lnot{}p\not\gets{}q$           & $\lnot{}p\downarrow\lnot{}q$          & \F   & \F   & \F   & \T    & ``与'',全真为真         \\
        \multicolumn{4}{c}{$\bot$}                                                                                                            & \F   & \F   & \F   & \F    & ``恒假''                 \\
        \hline
    \end{tabular}
\end{table}

\subsubsection{重言式和矛盾式}

若$p$、$q$、$r$是公式,可以证明下述等式恒成立:

\[ (p \to p)                                                               \equiv \top \tag{同一律} \]
\[ (\lnot{}p \land p)                                                      \equiv \bot \tag{矛盾律} \]
\[ (\lnot{}p \lor p)                                                       \equiv \top \tag{排中律} \]
\[ (\lnot\lnot{}p \leftrightarrow p)                                       \equiv \top \tag{双重否定律} \]

也可以证明下述合取运算规律恒成立:

\[ [(p \land q) \leftrightarrow (q \land p)]                               \equiv \top \tag{合取交换律} \]
\[ \{[(p \land q) \land r] \leftrightarrow [p \land (q \land r)]\}         \equiv \top \tag{合取结合律} \]
\[ \{[p \land (q \lor r)] \leftrightarrow [(p \land q) \lor (p \land r)]\} \equiv \top \tag{合取分配律} \]
\[ [\lnot(p \land q) \leftrightarrow (\lnot{}p \lor \lnot{}q)]             \equiv \top \tag{合取德·摩根律} \]

也可以证明下述析取运算规律恒成立:

\[ [(p \lor q) \leftrightarrow (q \lor p)]                                 \equiv \top \tag{析取交换律} \]
\[ \{[(p \lor q) \lor r] \leftrightarrow [p \lor (q \lor r)]\}             \equiv \top \tag{析取结合律} \]
\[ \{[p \lor (q \land r)] \leftrightarrow [(p \lor q) \land (p \lor r)]\}  \equiv \top \tag{析取分配律} \]
\[ [\lnot(p \lor q) \leftrightarrow (\lnot{}p \land \lnot{}q)]             \equiv \top \tag{析取德·摩根律} \]

也可以证明下述蕴含运算规律恒成立:

\[ [\lnot{}p \to (p \to q)]                                                \equiv \top \tag{否定前件律} \]
\[ [q \to (p \to q)]                                                       \equiv \top \tag{肯定后件律} \]
\[ \{[p \to (q \to r)] \to [(p \to q) \to (p \to r)]\}                     \equiv \top \tag{蕴含分配律} \]
\[ [(\lnot{}p \to \lnot{}q) \to (q \to p)]                                 \equiv \top \tag{蕴含换位律} \]
\[ [(\lnot{}p \to p) \to p]                                                \equiv \top \tag{否定肯定律} \]
\[ \{(p \to q) \to [(q \to r) \to (p \to r)]\}                             \equiv \top \tag{假设三段论} \]

\subsection{谓词演算初步}

\rTermWref{谓词演算}{PredicateCalculus}{predicate calculus}是一种形式系统,可描述为$\mathcal{L}(\mathrm{A},\Omega,\mathrm{Z},\mathrm{I})$,其中:
\begin{itemize}
    \renewcommand{\labelitemii}{}
    \item 集$\rHatNote{\mathrm{A}}{``alpha''}$是由代词$x$、$y$、$z$、$\cdots$组成的集;
    \item 集$\rHatNote{\Omega}{``omega''}$是由终结符组成的集,可将其划分为$\Omega_\text{量词}$、$\Omega_\text{名词}$、$\Omega_\text{映射}$、$\Omega_\text{谓词}$和$\Omega_\text{逻辑联结词}$:
          \begin{itemize}
              \item $\Omega_\text{量词} = \{ \rHatNote{\forall}{全称量词}, \rHatNote{\exists}{存在量词} \}$;
              \item $\Omega_\text{名词} = \{ a, b, c, \cdots \}$;
              \item $\Omega_\text{映射} = \{ f, g, h, \cdots \}$;
              \item $\Omega_\text{谓词} = \{ R, S, T, \cdots \}$;
              \item $\Omega_\text{逻辑联结词} = \{ \bot, \top, \lnot, \land, \lor, \to, \leftrightarrow, \cdots \}$;
          \end{itemize}
    \item 集$\rHatNote{\mathrm{Z}}{``zeta''}$是由推理规则组成的集:
          \begin{itemize}
              \item 1. 每个代词、名词都是项;
              \item 2. 若$f$是$n$元映射,且$t_1$、$\cdots$、$t_n$是项,则$f(t_1,\cdots,t_n)$也是项;
              \item 3. 除外都不是项;
              \item 4. 若$R$是$n$元谓词,且$t_1$、$\cdots$、$t_n$是项,则$R(t_1,\cdots,t_n)$是合式公式(下简称``公式'');
              \item 5. $\bot$和$\top$是公式;
              \item 6. 若$p$是公式,则$\lnot{}p$也是公式;
              \item 7. 若$p$和$q$是公式,则$p\land{}q$、$p\lor{}q$、$p\to{}q$、$p\leftrightarrow{}q$、$\cdots$也都是公式;
              \item 8. 若$x$是代词且$p$是公式,则$\forall{}x:p$、$\exists{}x:p$也都是公式;
              \item 9. 除外都不是公式.
           \end{itemize}
    \item 集$\rHatNote{\mathrm{I}}{``iota''}$是由\rTerm{逻辑公理}{logical axioms}组成的集,除继承命题演算公理外:
\end{itemize}

\newtheorem{LogicalAxioms}{LA}

\begin{LogicalAxioms}[\emph{等词公理}]\rMgnNote{axiom of equality}\label{LA:E}
    对于任一项$t$,公式
    \begin{equation*}
        t = t
    \end{equation*}
    \hfill 恒成立.
\end{LogicalAxioms}

\begin{LogicalAxioms}[\emph{全称例化公理模式}]\rMgnNote{axiom scheme for universal instantiation}\label{LA:UI}
    对于任一公式$\phi$、代词$x$和项$t$,公式
    \begin{equation*}
        (\forall x: \phi) \to \phi_{x\gets{}t}
    \end{equation*}
    \hfill 恒成立.
\end{LogicalAxioms}

\begin{LogicalAxioms}[\emph{存在泛化公理模式}]\rMgnNote{axiom scheme for existential generalization}\label{LA:EG}
    对于任一公式$\phi$、代词$x$和项$t$,公式
    \begin{equation*}
        \phi_{x\gets{}t} \to (\exists x: \phi)
    \end{equation*}
    \hfill 恒成立.
\end{LogicalAxioms}

\section{集论概念}

\rTermWref{策梅洛--弗兰克尔集论}{Zermelo-FraenkelSetTheory}{Zermelo-Fraenkel set theory}是一种谓词演算,
其项也被称作``\rTermWref{集合}{Set}{set}''(简称``集'')或``\rTermWref{元素}{Element}{element}''(简称``元''),
其谓词集$\Omega_\text{谓词}=\{\rHatNote{=}{等词},\rHatNote{\in}{``属于''}\}$,
其公理集$\mathrm{I}$有:

\newtheorem{ZermeloFraenkelAxioms}{ZF}      % 策梅洛–弗兰克尔公理
\setcounter{ZermeloFraenkelAxioms}{-1}
\newtheorem{ZermeloFraenkelTheorems}{定理}  % 策梅洛–弗兰克尔定理

\begin{ZermeloFraenkelAxioms}[\emph{集存在引理}]\rMgnNote{lemma of set exists}\label{ZF:SE}
    $ \exists S: S = S $.
\end{ZermeloFraenkelAxioms}

\hrule
\begin{proof}
    令假定集$ \Gamma = \{ \text{LA\ref{LA:E}}, \text{LA\ref{LA:EG}} \} $,有:
    \begin{enumerate}
        \item $ \phi_{x\gets{}t} \to (\exists x: \phi) $     \hfill LA\ref{LA:EG}
        \item $ (x = x)_{x\gets{}t} \to (\exists x: x = x) $ \hfill 令公式$\phi$为$ x = x $
        \item $ (t = t) \to (\exists x: x = x) $
        \item $ \top \to (\exists x: x = x) $                \hfill LA\ref{LA:E}
        \item $ \exists x: x = x $
        \item $ \exists S: S = S $                           \qedhere
    \end{enumerate}
\end{proof}
\hrule
\vspace{2ex}

集存在引理说明集合存在.

\begin{ZermeloFraenkelAxioms}[\emph{外延公理}]\rMgnNote{axiom of extensionality}\label{ZF:E}
    $ \forall A, \forall B: A = B \iff (\forall e: e \in A \leftrightarrow e \in B) $.
\end{ZermeloFraenkelAxioms}

外延公理说明一个集合完全由它所包含的元素确定:两个集合若拥有相同的元素,则这两个集合相等;反之亦然.

\begin{ZermeloFraenkelAxioms}[\emph{规范公理模式}]\rMgnNote{axiom schema of specification}\label{ZF:S}
    $ \forall U, \exists S, \forall e: e \in S \leftrightarrow e \in U \land P(e) $.
\end{ZermeloFraenkelAxioms}

规范公理模式说明给定一个集合,从该集合中选定符合条件的若干元素可以组成集合,即``\rTermWref{子集}{Subset}{subset}''.

由 ZF\ref{ZF:SE} 和 ZF\ref{ZF:S} 可以构造\rTermWref{空集}{EmptySet}{empty set}$ \emptyset = \{ e | e \in S \land (e \neq e) \} $——我们得到的第一个确定的集合.

\begin{ZermeloFraenkelAxioms}[\emph{配对公理}]\rMgnNote{axiom of pairing}\label{ZF:P}
    $ \forall a, \forall b, \exists S, \forall e: e \in S \leftrightarrow e = a \lor e = b $.
\end{ZermeloFraenkelAxioms}

配对公理说明给定两个元素$a$和$b$,则$ \{ a, b \} $是集合,称作``元素$a$和$b$的\rTermWref{对集}{Pair}{pair}''.
当给定的两个元素相等都为$e$时,$ \{ e, e \} $可以记作$ \{ e \} $,称作``元素$e$的\rTermWref{独集}{SingletonSet}{singleton}''.
形如$ \{ \{ a \}, \{ a, b \} \} $的对集可以记作$ (a, b) $称作``元素$a$和$b$的\rTermWref{有序对集}{OrderedPair}{ordered pair}''.

\begin{ZermeloFraenkelAxioms}[并集公理]\label{ZFA:4}
\end{ZermeloFraenkelAxioms}

\begin{ZermeloFraenkelAxioms}[幂集公理]\label{ZFA:5}
\end{ZermeloFraenkelAxioms}

\begin{ZermeloFraenkelAxioms}[无穷公理]\label{ZFA:6}
\end{ZermeloFraenkelAxioms}

\begin{ZermeloFraenkelAxioms}[替换公理]\label{ZFA:7}
\end{ZermeloFraenkelAxioms}

\subsection{正则公理、基础公理和选择公理}

\begin{ZermeloFraenkelAxioms}[正则公理]\label{ZFA:8}
    $ \forall S, \exists e: (\exists z: z \in S) \implies [e \in S \land \lnot(\exists y: y \in S \land y \in e)] $
\end{ZermeloFraenkelAxioms}
ZFA\ref{ZFA:8}说明所有非空集合中都存在这样的元素:该元素与该集合的交集为空集.


\section{集合}

\subsection{集合概念}

一般的,\emph{集合}(\href{http://mathworld.wolfram.com/Set.html}{Set},简称\emph{集})是指具有某种特定性质的事物的总体,组成这个集合的事物称为该集合的\emph{元素}(\href{http://mathworld.wolfram.com/Element.html}{Element},简称\emph{元}).
通常用大写拉丁字母 $A$,$B$,$C$,$\cdots$ 表示集合,用小写拉丁字母 $a$,$b$,$c$,$\cdots$ 表示集合的元素.

如果元素 $e$ 是集合 $S$ 的元素,就说``元素 $e$ \emph{属于}集合 $S$''(记作 $e\in{}S$)或``集合 $S$ \emph{拥有}元素 $e$''(记作 $S\ni{}e$).
如果元素 $e$ 不是集合 $S$ 的元素,就说``元素 $e$ \emph{不属于}集合 $S$''(记作 $e\not\in{}S$)或``集合 $S$ \emph{不拥有}元素 $e$''(记作 $S\not\ni{}e$).
一个集合,若它只拥有限个元素,则称为\emph{有限集};不是有限集的集合称为\emph{无限集}.

通常使用\emph{列举法}、\emph{描述法}或\emph{文氏图}(\href{http://mathworld.wolfram.com/VennDiagram.html}{Venn Diagram})表示集合:
例如,由元素 $a_1$,$a_2$,$\cdots$,$a_n$ 组成的集合 $A$ 可表示为
\[
A = \{ a_1, a_2, \cdots, a_n \} \text{;}
\]
由具有某种性质 $P$ 的元素 $b$ 的全体组成的集合 $B$ 可表示为
\[
B = \{ b | b \text{具有性质} P \} \text{.}
\]

\subsection{集合间的关系}

如果集合 $S$ 的元素都是集合 $L$ 的元素,则称集合 $S$ 是集合 $L$ 的\emph{子集}(\href{http://mathworld.wolfram.com/Subset.html}{Subset}),记作 $S\subset{}L$(读作``集合 $S$ \emph{包含于}集合 $L$'');
或称集合 $L$ 是集合 $S$ 的\emph{超集}(\href{http://mathworld.wolfram.com/Superset.html}{Superset}),记作 $L\supset{}S$(读作``集合 $L$ \emph{包含}集合 $S$'').

如果集合 $A$ 的元素都是集合 $B$ 的元素,且集合 $B$ 的元素也都是集合 $A$ 的元素,即集合 $A$ 与集合 $B$ 互相包含,则称集合 $A$ 与集合 $B$ \emph{相等},记作 $A=B$(读作``集合 $A$ \emph{等于}集合 $B$'').

如果集合 $S$ 包含于集合 $L$,且它们不相等,则称集合 $S$ 是集合 $L$ 的\emph{真子集}(\href{http://mathworld.wolfram.com/ProperSubset.html}{Proper Subset}),记作 $S\subsetneqq{}L$(读作``集合 $S$ \emph{真包含于}集合 $L$'');
或称集合 $L$ 是集合 $S$ 的\emph{真超集}(\href{http://mathworld.wolfram.com/ProperSuperset.html}{Proper Superset}),记作 $L\supsetneqq{}S$(读作``集合 $L$ \emph{真包含}集合 $S$'').

不拥有任何元素的集合称为\emph{空集}(\href{http://mathworld.wolfram.com/EmptySet.html}{Empty Set}),记作 $\varnothing$,规定空集是任何集合的子集.
有时,在指定上下文中可以将所有研究对象组成一个集合 $U$,所研究的其它集合都是集合 $U$ 的子集,此时我们称集合 $U$ 为\emph{全集}(\href{http://mathworld.wolfram.com/UniversalSet.html}{Universe}).

\subsection{并集、交集、差集、对称差集、补集和余集}

由所有属于集合 $A$ 或者属于集合 $B$ 的元素组成的集合,称为集合 $A$ 与集合 $B$ 的\emph{并集}(\href{http://mathworld.wolfram.com/Union.html}{Union},简称\emph{并}),记作 $A\cup{}B$,该运算也称为\emph{并}(\href{http://mathworld.wolfram.com/Cup.html}{Cup}),即
\[
A \cup B = \{ e | e \in A \text{或} e \in B \} \text{;}
\]
由所有既属于集合 $A$ 又属于集合 $B$ 的元素组成的集合,称为集合 $A$ 与集合 $B$ 的\emph{交集}(\href{http://mathworld.wolfram.com/Intersection.html}{Intersection},简称\emph{交}),记作 $A\cap{}B$,该运算也称为\emph{交}(\href{http://mathworld.wolfram.com/Cap.html}{Cap}),即
\[
A \cap B = \{ e | e \in A \text{且} e \in B \} \text{;}
\]
由所有属于集合 $A$ 但不属于集合 $B$ 的元素组成的集合,称为集合 $A$ 与集合 $B$ 的\emph{差集}(\href{http://mathworld.wolfram.com/SetDifference.html}{Set Difference},简称\emph{差}),记作 $A\setminus{}B$,该运算称为\emph{减}(\href{http://mathworld.wolfram.com/SetMinus.html}{Set Minus}),即
\[
A \setminus B = \{ e | e \in A \text{且} e \not\in B \} \text{;}
\]
由所有属于集合 $A$ 或者属于集合 $B$ 但不同时属于两集合的元素组成的集合,称为集合 $A$ 与集合 $B$ 的\emph{对称差集}(\href{http://mathworld.wolfram.com/SymmetricDifference.html}{Symmetric Difference}),记作 $A\ominus{}B$,即
\[
A \ominus B = (A \cup B) \setminus (A \cap B) \text{;}
\]
设有超集 $L$ 包含子集 $S$,则称超集 $L$ 与子集 $S$ 的差集,为超集 $L$ 中子集 $S$ 的\emph{补集}(\href{http://mathworld.wolfram.com/ComplementSet.html}{Complement Set}),记作 $\complement_LS$,即
\[
\begin{aligned}
\text{存在超集$L$和其子集$S$,} \complement_LS & = L \setminus S \\
& = \{ e | e \in L \text{且} e \not\in S \} \text{;}
\end{aligned}
\]
特别的,若存在全集 $U$,则称全集 $U$ 中子集 $S$ 的补集,为集合 $S$ 的\emph{余集},记作 $S^c$,即
\[
\begin{aligned}
\text{存在全集$U$,} S^c &= \complement_US \\
&= U \setminus S \\
&= \{ e | e \in U \text{且} e \not\in S \} \text{.}
\end{aligned}
\]

\subsection{笛卡尔乘积}

由所有``属于集合 $A$ 的任一元素 $a$ 和属于集合 $B$ 的任一元素 $b$ 组成的有序对 $(a, b)$'' 组成的集合,称为集合 $A$ 与集合 $B$ 的\emph{笛卡尔乘积}(\href{http://mathworld.wolfram.com/CartesianProduct.html}{Cartesian Product}),记作 $A\times{}B$,即
\[
A \times B = \{ (a, b) | a \in A \text{且} b \in B \} \text{.}
\]

\newpage
\section{映射}

\subsection{映射概念}

设集合 $X$ 和集合 $Y$ 是两个非空集合,对于集合 $X$ 中任一元素 $x$,依照某种对应规律 $f$,恒有集合 $Y$ 中唯一确定元素 $y$ 与之对应,则称对应规律 $f$ 为一个从集合 $X$ 到集合 $Y$ 的映射(\href{http://mathworld.wolfram.com/Map.html}{Map}),记作
\[
\begin{aligned}
f&: X \to Y \\
\text{或} \quad f&: x \mapsto y \text{,}
\end{aligned}
\]
其中:
\begin{itemize}
    \item 集合 $X$ 称为映射 $f$ 的\emph{定义域}(\href{http://mathworld.wolfram.com/Domain.html}{Domain}),记作 $D_f$;
    \item 集合 $Y$ 称为映射 $f$ 的\emph{陪域}(\href{http://mathworld.wolfram.com/Codomain.html}{Codomain}),记作 $C_f$;
    \item 元素 $x$ 称为在映射 $f$ 下元素 $y$ 的一个\emph{原像}(\href{http://mathworld.wolfram.com/Preimage.html}{Preimage});
    \item 元素 $y=f(x)$ 称为在映射 $f$ 下元素 $x$ 的\emph{像}(\href{http://mathworld.wolfram.com/Image.html}{Image});
    \item 集合 $f(X)=\{f(x)|x\in{}X\}$ 称为映射 $f$ 的\emph{值域}(\href{http://mathworld.wolfram.com/Range.html}{Range}),记作 $R_f$.
\end{itemize}

若对于定义域 $D_f$ 中任意原像 $d_1$、$d_2$ 不等,对应的像 $f(d_1)$、$f(d_2)$ 恒不等,则称映射 $f$ 是\emph{单射}(\href{http://mathworld.wolfram.com/Injection.html}{Injection});
若对于陪域 $C_f$ 中任一元素 $c$,都存在原像 $d$ 使得 $f(d)=c$,则称映射 $f$ 是\emph{满射}(\href{http://mathworld.wolfram.com/Surjection.html}{Surjection});
若映射 $f$ 既是单射又是满射,则称其为\emph{双射}(\href{http://mathworld.wolfram.com/Bijection.html}{Bijection}).

\subsection{逆映射}

设映射 $f$ 是一个从定义域 $D_f$ 到陪域 $C_f$ 的单射,则对于值域 $R_f$ 中任一像 $r$ 都存在唯一原像 $d$ 与之对应,于是可以定义一个从值域 $R_f$ 到定义域 $D_f$ 的新映射
\[
f^{-1}: r \mapsto d \text{,}
\]
这个映射称为映射 $f$ 的\emph{逆映射}(\href{http://mathworld.wolfram.com/InverseFunction.html}{Inverse Function}),其定义域 $D_{f^{-1}}=R_f$,值域 $R_{f^{-1}}=D_f$.

\subsection{复合映射}

设有两映射 $g=X\to{}Y_S$ 和 $f=Y_L\to{}Z$ 且 $Y_S\subset{}Y_L$,则对于集合 $X$ 中任一元素 $x$ 恒有集合 $Z$ 中唯一确定元素 $z=f(g(x))$ 与之对应,于是可以定义一个从集合 $X$ 到集合 $Z$ 的新映射
\[
f \circ g: x \mapsto f(g(x)) \text{,}
\]
这个映射称为映射 $g$ 和映射 $f$ 构成的\emph{复合映射}(\href{http://mathworld.wolfram.com/Composition.html}{Composition}).

\newpage
\section{二元运算}

从集合 $S$ 的笛卡尔平方 $S\times{}S$ 到集合 $S$ 的映射 $f:S\times{}S\to{}S$,称作集合 $S$ 上的\emph{二元运算}(\href{http://mathworld.wolfram.com/BinaryOperation.html}{Binary Operation});
若元素 $a$、$b$ 是集合 $S$ 的元素,通常将二元运算的像 $f(a, b)$ 记作 $a f b$.

%设集合 $S$ 上的二元运算 $*:S\times{}S\to{}S$,元素 $e_0$、$e_1$ 是集合 $S$ 的元素:
%\begin{itemize}
%	\item 若 $\forall a \in S \implies e_0 * a = e_0$,则称元素 $e_0$ 为二元运算 $*$ 的左零元;
%\end{itemize}


\chapter{集合、映射和二元运算}

\section{集合}

\subsection{集合与元素间的关系}

一般的,\emph{集合}(\href{http://mathworld.wolfram.com/Set.html}{Set},简称\emph{集})是指具有某种特定性质的事物的总体,组成这个集合的事物称为该集合的\emph{元素}(\href{http://mathworld.wolfram.com/Element.html}{Element},简称\emph{元}).
通常用大写拉丁字母 $A$,$B$,$C$,$\cdots$ 表示集合,用小写拉丁字母 $a$,$b$,$c$,$\cdots$ 表示集合的元素.

如果元素 $e$ 是集合 $S$ 的元素,就说元素 $e$ \emph{属于}集合 $S$(记作 $e\in{}S$)或集合 $S$ \emph{拥有}元素 $e$(记作 $S\ni{}e$).
如果元素 $e$ 不是集合 $S$ 的元素,就说元素 $e$ \emph{不属于}集合 $S$(记作 $e\not\in{}S$) 或集合 $S$ \emph{不拥有}元素 $e$(记作 $S\not\ni{}e$).
一个集合,若它只拥有限个元素,则称为\emph{有限集};不是有限集的集合称为\emph{无限集}.

\begin{table}[h]
	\centering
	\begin{tabular}{l l l}
		\hline
		$e\in{}S$     & 元素 $e$ 属于集合 $S$   & \multirow{2}{*}{元素 $e$ 是集合 $S$ 的元素}   \\
		$S\ni{}e$     & 集合 $S$ 拥有元素 $e$   &                                               \\
		$e\not\in{}S$ & 元素 $e$ 不属于集合 $S$ & \multirow{2}{*}{元素 $e$ 不是集合 $S$ 的元素} \\
		$S\not\ni{}e$ & 集合 $S$ 不拥有元素 $e$ &                                               \\
		\hline
	\end{tabular}
\end{table}

通常使用\emph{列举法}、\emph{描述法}或\emph{文氏图}(\href{http://mathworld.wolfram.com/VennDiagram.html}{Venn Diagram})表示集合:
例如,由元素 $e_1$,$e_2$,$\cdots$,$e_n$ 组成的集合 $S_1$ 可表示为
\[ S_1 = \{ e_1, e_2, \cdots, e_n \} \text{;} \]
由具有某种性质 $P$ 的元素 $e$ 的全体组成的集合 $S_2$ 可表示为
\[ S_2 = \{ e | e \text{具有性质} P \} \text{.} \]

\subsection{集合与集合间的关系}

如何集合 $V$ 的元素都是集合 $W$ 的元素,则称集合 $V$ 是集合 $W$ 的\emph{子集}(\href{http://mathworld.wolfram.com/Subset.html}{Subset}),记作 $V\subset{}W$(读作集合 $V$ \emph{包含于}集合 $W$);
或称集合 $W$ 是集合 $V$ 的\emph{超集}(\href{http://mathworld.wolfram.com/Superset.html}{Superset}),记作 $W\supset{}V$(读作集合 $W$ \emph{包含}集合 $V$).

如果集合 $S_1$ 的元素都是集合 $S_2$ 的元素,且集合 $S_2$ 的元素也都是集合 $S_1$ 的元素,即集合 $S_1$ 与集合 $S_2$ 互相包含,则称集合 $S_1$ 与集合 $S_2$ \emph{相等},记作 $S_1=S_2$(读作集合 $S_1$ \emph{等于} $S_2$).

如果集合 $V$ 包含于集合 $W$,且它们不相等,则称集合 $V$ 是集合 $W$ 的\emph{真子集}(\href{http://mathworld.wolfram.com/ProperSubset.html}{Proper Subset}),记作 $V\subsetneqq{}W$(读作集合 $V$ \emph{真包含于}集合 $W$);
或称集合 $W$ 是集合 $V$ 的\emph{真超集}(\href{http://mathworld.wolfram.com/ProperSuperset.html}{Proper Superset}),记作 $W\supsetneqq{}V$(读作集合 $W$ \emph{真包含}集合 $V$).

\begin{table}[h]
	\centering
	\begin{tabular}{l l l}
		\hline
		$V\subset{}W$     & 集合 $V$ 包含于集合 $W$   & 集合 $V$ 是集合 $W$ 的子集   \\
		$W\supset{}V$     & 集合 $W$ 包含集合 $V$     & 集合 $W$ 是集合 $V$ 的超集   \\
		$S_1=S_2$         & 集合 $S_1$ 等于集合 $S_2$ & 集合 $S_1$ 与集合 $S_2$ 相等 \\
		$V\subsetneqq{}W$ & 集合 $V$ 真包含于集合 $W$ & 集合 $V$ 是集合 $W$ 的子集   \\
		$W\supsetneqq{}V$ & 集合 $W$ 真包含集合 $V$   & 集合 $W$ 是集合 $V$ 的超集   \\
		\hline
	\end{tabular}
\end{table}

不拥有任何元素的集合称为\emph{空集}(\href{http://mathworld.wolfram.com/EmptySet.html}{Empty Set}),记作 $\emptyset$,规定空集是任何集合的子集;
在指定上下文中,拥有所有元素的集合称作\emph{全集}(\href{http://mathworld.wolfram.com/UniversalSet.html}{Universe}),全集通常用 $U$ 表示.

\subsection{集合与集合间的运算}


% 附录
\appendix

% 参考文献
\nocite{*}
\printbibliography[heading=bibliography,title=参考文献]

\end{document}
