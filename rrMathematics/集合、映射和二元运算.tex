\chapter{集合、映射和二元运算}

\section{集合}

\subsection{集合概念}

一般的,\emph{集合}(\href{http://mathworld.wolfram.com/Set.html}{Set},简称\emph{集})是指具有某种特定性质的事物的总体,组成这个集合的事物称为该集合的\emph{元素}(\href{http://mathworld.wolfram.com/Element.html}{Element},简称\emph{元}).
通常用大写拉丁字母 $A$,$B$,$C$,$\cdots$ 表示集合,用小写拉丁字母 $a$,$b$,$c$,$\cdots$ 表示集合的元素.

如果元素 $e$ 是集合 $S$ 的元素,就说``元素 $e$ \emph{属于}集合 $S$''(记作 $e\in{}S$)或``集合 $S$ \emph{拥有}元素 $e$''(记作 $S\ni{}e$).
如果元素 $e$ 不是集合 $S$ 的元素,就说``元素 $e$ \emph{不属于}集合 $S$''(记作 $e\not\in{}S$)或``集合 $S$ \emph{不拥有}元素 $e$''(记作 $S\not\ni{}e$).
一个集合,若它只拥有限个元素,则称为\emph{有限集};不是有限集的集合称为\emph{无限集}.

通常使用\emph{列举法}、\emph{描述法}或\emph{文氏图}(\href{http://mathworld.wolfram.com/VennDiagram.html}{Venn Diagram})表示集合:
例如,由元素 $a_1$,$a_2$,$\cdots$,$a_n$ 组成的集合 $A$ 可表示为
\[
A = \{ a_1, a_2, \cdots, a_n \} \text{;}
\]
由具有某种性质 $P$ 的元素 $b$ 的全体组成的集合 $B$ 可表示为
\[
B = \{ b | b \text{具有性质} P \} \text{.}
\]

\subsection{集合间的关系}

如果集合 $S$ 的元素都是集合 $L$ 的元素,则称集合 $S$ 是集合 $L$ 的\emph{子集}(\href{http://mathworld.wolfram.com/Subset.html}{Subset}),记作 $S\subset{}L$(读作``集合 $S$ \emph{包含于}集合 $L$'');
或称集合 $L$ 是集合 $S$ 的\emph{超集}(\href{http://mathworld.wolfram.com/Superset.html}{Superset}),记作 $L\supset{}S$(读作``集合 $L$ \emph{包含}集合 $S$'').

如果集合 $A$ 的元素都是集合 $B$ 的元素,且集合 $B$ 的元素也都是集合 $A$ 的元素,即集合 $A$ 与集合 $B$ 互相包含,则称集合 $A$ 与集合 $B$ \emph{相等},记作 $A=B$(读作``集合 $A$ \emph{等于}集合 $B$'').

如果集合 $S$ 包含于集合 $L$,且它们不相等,则称集合 $S$ 是集合 $L$ 的\emph{真子集}(\href{http://mathworld.wolfram.com/ProperSubset.html}{Proper Subset}),记作 $S\subsetneqq{}L$(读作``集合 $S$ \emph{真包含于}集合 $L$'');
或称集合 $L$ 是集合 $S$ 的\emph{真超集}(\href{http://mathworld.wolfram.com/ProperSuperset.html}{Proper Superset}),记作 $L\supsetneqq{}S$(读作``集合 $L$ \emph{真包含}集合 $S$'').

不拥有任何元素的集合称为\emph{空集}(\href{http://mathworld.wolfram.com/EmptySet.html}{Empty Set}),记作 $\varnothing$,规定空集是任何集合的子集.
有时,在指定上下文中可以将所有研究对象组成一个集合 $U$,所研究的其它集合都是集合 $U$ 的子集,此时我们称集合 $U$ 为\emph{全集}(\href{http://mathworld.wolfram.com/UniversalSet.html}{Universe}).

\subsection{并集、交集、差集、对称差集、补集和余集}

由所有属于集合 $A$ 或者属于集合 $B$ 的元素组成的集合,称为集合 $A$ 与集合 $B$ 的\emph{并集}(\href{http://mathworld.wolfram.com/Union.html}{Union},简称\emph{并}),记作 $A\cup{}B$,该运算也称为\emph{并}(\href{http://mathworld.wolfram.com/Cup.html}{Cup}),即
\[
A \cup B = \{ e | e \in A \text{或} e \in B \} \text{;}
\]
由所有既属于集合 $A$ 又属于集合 $B$ 的元素组成的集合,称为集合 $A$ 与集合 $B$ 的\emph{交集}(\href{http://mathworld.wolfram.com/Intersection.html}{Intersection},简称\emph{交}),记作 $A\cap{}B$,该运算也称为\emph{交}(\href{http://mathworld.wolfram.com/Cap.html}{Cap}),即
\[
A \cap B = \{ e | e \in A \text{且} e \in B \} \text{;}
\]
由所有属于集合 $A$ 但不属于集合 $B$ 的元素组成的集合,称为集合 $A$ 与集合 $B$ 的\emph{差集}(\href{http://mathworld.wolfram.com/SetDifference.html}{Set Difference},简称\emph{差}),记作 $A\setminus{}B$,该运算称为\emph{减}(\href{http://mathworld.wolfram.com/SetMinus.html}{Set Minus}),即
\[
A \setminus B = \{ e | e \in A \text{且} e \not\in B \} \text{;}
\]
由所有属于集合 $A$ 或者属于集合 $B$ 但不同时属于两集合的元素组成的集合,称为集合 $A$ 与集合 $B$ 的\emph{对称差集}(\href{http://mathworld.wolfram.com/SymmetricDifference.html}{Symmetric Difference}),记作 $A\ominus{}B$,即
\[
A \ominus B = (A \cup B) \setminus (A \cap B) \text{;}
\]
设有超集 $L$ 包含子集 $S$,则称超集 $L$ 与子集 $S$ 的差集,为超集 $L$ 中子集 $S$ 的\emph{补集}(\href{http://mathworld.wolfram.com/ComplementSet.html}{Complement Set}),记作 $\complement_LS$,即
\[
\begin{aligned}
\text{存在超集$L$和其子集$S$,} \complement_LS & = L \setminus S \\
                                     & = \{ e | e \in L \text{且} e \not\in S \} \text{;}
\end{aligned}
\]
特别的,若存在全集 $U$,则称全集 $U$ 中子集 $S$ 的补集,为集合 $S$ 的\emph{余集},记作 $S^c$,即
\[
\begin{aligned}
\text{存在全集$U$,} S^c &= \complement_US \\
                         &= U \setminus S \\
                         &= \{ e | e \in U \text{且} e \not\in S \} \text{.}
\end{aligned}
\]

\subsection{笛卡尔乘积}

由所有``属于集合 $A$ 的任一元素 $a$ 和属于集合 $B$ 的任一元素 $b$ 组成的有序对 $(a, b)$'' 组成的集合,称为集合 $A$ 与集合 $B$ 的\emph{笛卡尔乘积}(\href{http://mathworld.wolfram.com/CartesianProduct.html}{Cartesian Product}),记作 $A\times{}B$,即
\[
A \times B = \{ (a, b) | a \in A \text{且} b \in B \} \text{.}
\]

\newpage
\section{映射}

\subsection{映射概念}

设集合 $X$ 和集合 $Y$ 是两个非空集合,对于集合 $X$ 中任一元素 $x$,依照某种对应规律 $f$,恒有集合 $Y$ 中唯一确定元素 $y$ 与之对应,则称对应规律 $f$ 为一个从集合 $X$ 到集合 $Y$ 的映射(\href{http://mathworld.wolfram.com/Map.html}{Map}),记作
\[
\begin{aligned}
                f&: X \to Y \\
\text{或} \quad f&: x \mapsto y \text{,}
\end{aligned}
\]
其中:
\begin{itemize}
	\item 集合 $X$ 称为映射 $f$ 的\emph{定义域}(\href{http://mathworld.wolfram.com/Domain.html}{Domain}),记作 $D_f$;
	\item 集合 $Y$ 称为映射 $f$ 的\emph{陪域}(\href{http://mathworld.wolfram.com/Codomain.html}{Codomain}),记作 $C_f$;
	\item 元素 $x$ 称为在映射 $f$ 下元素 $y$ 的一个\emph{原像}(\href{http://mathworld.wolfram.com/Preimage.html}{Preimage});
	\item 元素 $y=f(x)$ 称为在映射 $f$ 下元素 $x$ 的\emph{像}(\href{http://mathworld.wolfram.com/Image.html}{Image});
	\item 集合 $f(X)=\{f(x)|x\in{}X\}$ 称为映射 $f$ 的\emph{值域}(\href{http://mathworld.wolfram.com/Range.html}{Range}),记作 $R_f$.
\end{itemize}

若对于定义域 $D_f$ 中任意原像 $d_1$、$d_2$ 不等,对应的像 $f(d_1)$、$f(d_2)$ 恒不等,则称映射 $f$ 是\emph{单射}(\href{http://mathworld.wolfram.com/Injection.html}{Injection});
若对于陪域 $C_f$ 中任一元素 $c$,都存在原像 $d$ 使得 $f(d)=c$,则称映射 $f$ 是\emph{满射}(\href{http://mathworld.wolfram.com/Surjection.html}{Surjection});
若映射 $f$ 既是单射又是满射,则称其为\emph{双射}(\href{http://mathworld.wolfram.com/Bijection.html}{Bijection}).

\subsection{逆映射}

设映射 $f$ 是一个从定义域 $D_f$ 到陪域 $C_f$ 的单射,则对于值域 $R_f$ 中任一像 $r$ 都存在唯一原像 $d$ 与之对应,于是可以定义一个从值域 $R_f$ 到定义域 $D_f$ 的新映射
\[
f^{-1}: r \mapsto d \text{,}
\]
这个映射称为映射 $f$ 的\emph{逆映射}(\href{http://mathworld.wolfram.com/InverseFunction.html}{Inverse Function}),其定义域 $D_{f^{-1}}=R_f$,值域 $R_{f^{-1}}=D_f$.

\subsection{复合映射}

设有两映射 $g=X\to{}Y_S$ 和 $f=Y_L\to{}Z$ 且 $Y_S\subset{}Y_L$,则对于集合 $X$ 中任一元素 $x$ 恒有集合 $Z$ 中唯一确定元素 $z=f(g(x))$ 与之对应,于是可以定义一个从集合 $X$ 到集合 $Z$ 的新映射
\[
f \circ g: x \mapsto f(g(x)) \text{,}
\]
这个映射称为映射 $g$ 和映射 $f$ 构成的\emph{复合映射}(\href{http://mathworld.wolfram.com/Composition.html}{Composition}).

\newpage
\section{二元运算}

从集合 $S$ 的笛卡尔平方 $S\times{}S$ 到集合 $S$ 的映射 $f:S\times{}S\to{}S$,称作集合 $S$ 上的\emph{二元运算}(\href{http://mathworld.wolfram.com/BinaryOperation.html}{Binary Operation});
若元素 $a$、$b$ 是集合 $S$ 的元素,通常将二元运算的像 $f(a, b)$ 记作 $a f b$.

%设集合 $S$ 上的二元运算 $*:S\times{}S\to{}S$,元素 $e_0$、$e_1$ 是集合 $S$ 的元素:
%\begin{itemize}
%	\item 若 $\forall a \in S \implies e_0 * a = e_0$,则称元素 $e_0$ 为二元运算 $*$ 的左零元;
%\end{itemize}
