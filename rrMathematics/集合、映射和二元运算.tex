\chapter{集合、映射和二元运算}

\section{集合}

\subsection{集合与元素间的关系}

一般的,\emph{集合}(\href{http://mathworld.wolfram.com/Set.html}{Set},简称\emph{集})是指具有某种特定性质的事物的总体,组成这个集合的事物称为该集合的\emph{元素}(\href{http://mathworld.wolfram.com/Element.html}{Element},简称\emph{元}).
通常用大写拉丁字母 $A$,$B$,$C$,$\cdots$ 表示集合,用小写拉丁字母 $a$,$b$,$c$,$\cdots$ 表示集合的元素.

如果元素 $e$ 是集合 $S$ 的元素,就说元素 $e$ \emph{属于}集合 $S$(记作 $e\in{}S$)或集合 $S$ \emph{拥有}元素 $e$(记作 $S\ni{}e$).
如果元素 $e$ 不是集合 $S$ 的元素,就说元素 $e$ \emph{不属于}集合 $S$(记作 $e\not\in{}S$) 或集合 $S$ \emph{不拥有}元素 $e$(记作 $S\not\ni{}e$).
一个集合,若它只拥有限个元素,则称为\emph{有限集};不是有限集的集合称为\emph{无限集}.

\begin{table}[h]
	\centering
	\begin{tabular}{l l l}
		\hline
		$e\in{}S$     & 元素 $e$ 属于集合 $S$   & \multirow{2}{*}{元素 $e$ 是集合 $S$ 的元素}   \\
		$S\ni{}e$     & 集合 $S$ 拥有元素 $e$   &                                               \\
		$e\not\in{}S$ & 元素 $e$ 不属于集合 $S$ & \multirow{2}{*}{元素 $e$ 不是集合 $S$ 的元素} \\
		$S\not\ni{}e$ & 集合 $S$ 不拥有元素 $e$ &                                               \\
		\hline
	\end{tabular}
\end{table}

通常使用\emph{列举法}、\emph{描述法}或\emph{文氏图}(\href{http://mathworld.wolfram.com/VennDiagram.html}{Venn Diagram})表示集合:
例如,由元素 $e_1$,$e_2$,$\cdots$,$e_n$ 组成的集合 $S_1$ 可表示为
\[ S_1 = \{ e_1, e_2, \cdots, e_n \} \text{;} \]
由具有某种性质 $P$ 的元素 $e$ 的全体组成的集合 $S_2$ 可表示为
\[ S_2 = \{ e | e \text{具有性质} P \} \text{.} \]

\subsection{集合与集合间的关系}

如何集合 $V$ 的元素都是集合 $W$ 的元素,则称集合 $V$ 是集合 $W$ 的\emph{子集}(\href{http://mathworld.wolfram.com/Subset.html}{Subset}),记作 $V\subset{}W$(读作集合 $V$ \emph{包含于}集合 $W$);
或称集合 $W$ 是集合 $V$ 的\emph{超集}(\href{http://mathworld.wolfram.com/Superset.html}{Superset}),记作 $W\supset{}V$(读作集合 $W$ \emph{包含}集合 $V$).

如果集合 $S_1$ 的元素都是集合 $S_2$ 的元素,且集合 $S_2$ 的元素也都是集合 $S_1$ 的元素,即集合 $S_1$ 与集合 $S_2$ 互相包含,则称集合 $S_1$ 与集合 $S_2$ \emph{相等},记作 $S_1=S_2$(读作集合 $S_1$ \emph{等于} $S_2$).

如果集合 $V$ 包含于集合 $W$,且它们不相等,则称集合 $V$ 是集合 $W$ 的\emph{真子集}(\href{http://mathworld.wolfram.com/ProperSubset.html}{Proper Subset}),记作 $V\subsetneqq{}W$(读作集合 $V$ \emph{真包含于}集合 $W$);
或称集合 $W$ 是集合 $V$ 的\emph{真超集}(\href{http://mathworld.wolfram.com/ProperSuperset.html}{Proper Superset}),记作 $W\supsetneqq{}V$(读作集合 $W$ \emph{真包含}集合 $V$).

\begin{table}[h]
	\centering
	\begin{tabular}{l l l}
		\hline
		$V\subset{}W$     & 集合 $V$ 包含于集合 $W$   & 集合 $V$ 是集合 $W$ 的子集   \\
		$W\supset{}V$     & 集合 $W$ 包含集合 $V$     & 集合 $W$ 是集合 $V$ 的超集   \\
		$S_1=S_2$         & 集合 $S_1$ 等于集合 $S_2$ & 集合 $S_1$ 与集合 $S_2$ 相等 \\
		$V\subsetneqq{}W$ & 集合 $V$ 真包含于集合 $W$ & 集合 $V$ 是集合 $W$ 的子集   \\
		$W\supsetneqq{}V$ & 集合 $W$ 真包含集合 $V$   & 集合 $W$ 是集合 $V$ 的超集   \\
		\hline
	\end{tabular}
\end{table}

不拥有任何元素的集合称为\emph{空集}(\href{http://mathworld.wolfram.com/EmptySet.html}{Empty Set}),记作 $\emptyset$,规定空集是任何集合的子集;
在指定上下文中,拥有所有元素的集合称作\emph{全集}(\href{http://mathworld.wolfram.com/UniversalSet.html}{Universe}),全集通常用 $U$ 表示.

\subsection{集合与集合间的运算}
